\documentclass{article}
\usepackage[utf8]{inputenc}
\usepackage[russian]{babel}
\begin{document}

\begin{tabular}{p{3cm}|p{4cm}|p{5cm}}
Выражение & Перевод & Возможные варианты/\textit{вопросы} \\
\hline
accumulated savings & текущие накопления \\
actual cost & реальная стоимость \\
banker's method & метод банкира & банковский? \\
batched rebuilding & порционная перестройка \\
bootstrapping & развёртка & раскрутка \\
complete cost & полная стоимость \\
debit & единица долга \\
execution trace & трассировка вычисления \\
force & вынудить \\
incremental computation & пошаговое вычисление \\
leftist heaps & левоориентированные кучи \\
node & узел & вершина \\
non-uniformly recursive types & гетерогенно рекурсивные типы \\
pairing heaps & парные кучи & спаривательные кучи, кучи со спариванием
\\
pennant & подвешенное дерево \\
persistence & устойчивость \\
physicist's method & метод физика & физический \\
raise an exception & возбудить исключение \\
relaxed heaps & расслабленные кучи & ослабленные? \\
release debit & высвободить долг & выплатить \\
rotate a queue & провернуть очередь \\
rotation & проворот \\
schedule & расписание & \textit{можно было бы говорить
  \emph{scheduling} `планирование', но не хочется плодить
  разнокоренные переводы} \\
shared cost & разделяемая стоимость \\
sharing & совместное использование \\
skew binary numbers & скошенные двоичные числа \\
splay heaps & расширяющиеся кучи & \textit{См. перевод Cormen-Leiserson-Rivest}\\
strict evaluation & энергичный порядок вычислений & строгий \\
suspend & задержать & подвесить, заморозить \\
suspension & задержка \\
tries & префиксные деревья & \textit{слово <<префиксные>> ъорошо
  подходит, когда речь идёт о строках, но плохо, когда о деревьях} \\
uniformly recursive types & гомогенно рекурсивные типы \\
unshared cost & нераздельная стоимость & неразделяемая \\
view & взгляд & перспектива, отражение \\
worst-case limit & жёсткая оценка, оценка для худшего случая &
\textit{было бы приятнее иметь одно выражение вместо двух, неочевидно
  синонимичных} \\

\end{tabular}

\end{document}
