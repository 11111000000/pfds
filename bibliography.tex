\begin{thebibliography}{XXXXXXX}
\bibitem[Ada93]{Adams1993} Stephen Adams. Efficient sets~--- a
  balancing act. \textit{Journal of Functional Programming},
  3(4):553-561, October 1993.
\bibitem[AFM$^+$95]{Ariola-etal1995} Zena M.~Ariola, Matthias
  Felleisen, John Maraist, Martin Odersky, and Philip Wadler. A
  call-by-need lambda calculus. In \textit{ACM Symposium on Principles
  of Programming Languages}, pages 233--246, January 1995.
\bibitem[And95]{Andersson1991} Arne Andersson. A note on searching in
  a binary search tree. \textit{Software---Practice and Experience},
  21(10):1125-1128, October 1991.
\bibitem[AVL62]{AdelsonVelskiiLandis1962} Г.~М.~Адельсон-Вельский
  и Е.~М.~Ландис. Один алгоритм организации
  информации. \textit{Доклады Академии Наук СССР}, 146:263--266, 1962.
\bibitem[Bac78]{Backus1978} John Backus. Can programming be liberated
  from the von Neumann style? A functional style and its algebra of
  programs. \textit{Communications of the ACM}, 21(8):613--641, August 1978.
\bibitem[BAG92]{BenAmramGalil1992} Amir Ben-Amram and Zvi Galil. On
  pointers versus addresses. \textit{Journal of the ACM},
  39(3):617-648, July 1992.
\bibitem[BC93]{BurtonCameron1993} F.~Warren Burton and
  Robert D.~Cameron. Pattern matching with abstract data
  types. \textit{Journal of Functional Programming}, 3(2):171-190,
  April 1993.
\bibitem[Bel57]{Bellman1957} Richard Bellman. \textit{Dynamic
    Programming.}\/ Princeton University Press, 1957.
\bibitem[BH89]{BjernerHolmstrom1989} Bjor Bjerner and S\"oren
  Holmstr\"om. A compositional approach to time analysis of first
  order lazy functional programs. In \textit{Conference on Functional
    Programming Languages and Computer Architecture}, pages 157--165,
  September 1989.
\bibitem[BO96]{BrodalOkasaki1996} Gerth St\o{}lting Brodal and Chris
  Okasaki. Optimal purely functional priority queues. \textit{Journal
    of Functional Programming}, 6(6):839--857, November 1996.
\bibitem[Bro78]{Brown1978} Mark R.~Brown. Implementation and analysis
  of binomial queue algorithms. \textit{SIAM Journal of Computing},
  7(3):298--319, August 1978.
\bibitem[Bro95]{Brodal1995} Gerth St\o{}lting Brodal. Fast meldable
  priority queues. In \textit{Workshop on Algorithms and Data
    Structures}, volume 995 of \textit{LNCS}, pages
  282--290. Springer-Verlag, August 1995.
\bibitem[Bro96]{Brodal1996} Gerth St\o{}lting Brodal. Worst-case
  priority queues. In \textit{ACM-SIAM Symposium on Discrete
    Algorithms}, pages 52--58, January 1996.
\bibitem[BST95]{BuchsbaumSundarTarjan1995} Adam L.~Buchsbaum, Rajamani
  Sundar, and Robert E.~Tarjan. Data-structural bootstrapping, linear
  path compression, and catenable heap-ordered double-ended
  queues. \textit{SIAM Journal on Computing}, 24(6):1190--1206,
  December 1995.
\bibitem[BT95]{BuchsbaumTarjan1995} Adam L.~Buchsbaum and
  Robert E.~Tarjan. Confluently persistent deques via data structural
  bootstrapping. \textit{Journal of Algorithms}, 18(3):513--547, May 1995.
\bibitem[Buc93]{Buchsbaum1993}
  Adam L.~Buchsbaum. \textit{Data-structural bootstrapping and
    catenable deques}. PhD thesis, Department of Computer Science,
  Princeton University, June 1993.
\bibitem[Bur82]{Burton1982} F.~Warren Burton. An efficient functional
  implementation of FIFO queues. \textit{Information Processing
    Letters}, 14(5):205--206, July 1982.
\bibitem[But83]{Butler1983} T.~W.~Butler. Computer response time and
  user performance. In \textit{Conference on Human Factors in
    Computing Systems}, pages 58--62, December 1983.
\bibitem[BW88]{BirdWadler1988} Richard S.~Bird and Philip
  Wadler. \textit{Introduction to Functional Programming}. Prentice
  Hall International, 1988.
\bibitem[CG93]{ChuangGoldberg1993} Tung-Ruey Chuang and Benjamin
  Goldberg. Real-time deques, multihead Turing machines, and purely
  functional programming. In \textit{Conference on Functional
    Programming Languages and Computer Architecture}, pages 289-298,
  June 1993.
\bibitem[CLR90]{CormenLeisersonRivest1990} Thomas H.~Cormen, Charles
  E.~Leiserson, and Ronald L.~Rivest. \textit{Introduction to
    Algorithms}. MIT Press, 1990. Русский перевод: Т.~Кормен,
  Ч.~Лейзерсон, Р.~Ривест. \textit{Алгоритмы: построение и
    анализ}. Москва, МЦНМО, 2001.
\bibitem[CM95]{ConnellyMorris1995} Richard H.~Connelly and F.~Lockwood
  Morris. A generalization of the trie data
  structure. \textit{Mathematical Structures in Computer Science},
  5(3):381--418, September 1995.
\bibitem[CMP88]{CarlssonMunroPoblete1988} Svante Carlsson, Ian Munro,
  and Patricio V.~Poblete. An implicit binomial queue with constant
  insertion time. In \textit{Scandinavian Workshop on Algorithm
    Theory}, volume 318 of \textit{LNCS}, pages
  1--13. Springer-Verlag, July 1988.
\bibitem[Cra72]{Crane1972} Clark Allan Crane. \textit{Linear lists and
  priority queues as balanced binary trees}. PhD thesis, Computer
Science Department, Stanford University, February 1972. Available as STAN-CS-72-259.
\bibitem[CS96]{ChoSahni1996} Seonghun Cho and Sartaj Sahni. Weight
  biased leftist trees and modified skip lists. In
  \textit{International Computing and Combinatorics Conference}, pages
  361--370, June 1996.
\bibitem[DGST88]{Driscoll-etal1988} James R.~Driscoll, Harold
  N.~Gabow, Ruth Shrairman, and Robert E.~Tarjan. Relaxed heaps: An
  alternative to Fibonacci heaps with applications to parallel
  computation. \textit{Communications of the ACM}, 31(11):1343-1354,
  November 1988.
\bibitem[Die82]{Dietz1982} Paul F.~Dietz. Maintaining order in a
  linked list. In \textit{ACM Symposium on Theory of Computing}, pages
  122--127, May 1982.
\bibitem[Die89]{Dietz1989} Paul F.~Dietz. Fully persistent arrays. In
  \textit{Workshop on Algorithms and Data Structures}, volume 382 of
  \textit{LNCS}, pages 67--74. Springer-Verlag, August 1989.
\bibitem[DR91]{DietzRaman1991} Paul F.~Dietz and Rajeev
  Raman. Persistence, amortization and randomization. In
  \textit{ACM-SIAM Symposium on Discrete Algorithms}, pages 78--88,
  January 1991.
\bibitem[DR93]{DietzRaman1993} Paul F.~Dietz and Rajeev Raman.
  Persistence, randomization and parallelization: On some
  combinatorial games and their applications. in \textit{Workshop on
    Algorithms and Data Structures}, volume 709 of \textit{LNCS},
  pages 289--301. Springer-Verlag, August 1993.
\bibitem[DS87]{DietzSleator1987} Paul F.~Dietz and Danial
  D.~Sleator. Two algorithms for maintaining order in a list. In
  \textit{ACM Symposium on Theory of Computing}, pages 365--372, May 1987.
\bibitem[DSST89]{Driscoll-etal1989} James R.~Driscoll, Neil Sarnak,
  Daniel D.~K.~Sleator, and Robert E.~Tarjan. Making data structures
  persistent. \textit{Journal of Computing and System Sciences},
  38(1):86--124, February 1989.
\bibitem[DST94]{DriscollSleatorTarjan1994} James R.~Driscoll, Daniel
  D.~K.~Sleator, and Robert E.~Tarjan. Fully persistent lists with
  catenation. \textit{Journal of the ACM}, 41(5):943--959, September 1994.
\bibitem[FB97]{FahndrichBoyland1997} Manuel F\"ahndrich and John
  Boyland. Statically checkable pattern abstractions. In \textit{ACM
    SIGPLAN International Conference on Functional Programming}, pages
  75--84, June 1997.
\bibitem[FMR72]{FischerMeyerRosenberg1972} Patrick C.~Fischer, Albert
  R.~Meyer and Arnold L.~Rosenberg. Real-time simulation of multihead
  tape units. \textit{Journal of the ACM}, 19(4):590--607, October 1972.
\bibitem[FSST86]{Fredman-etal1986} Michael L.~Fredman, Robert
  Sedgewick, Daniel D.~K.~Sleator, and Robert E.~Tarjan. The pairing
  heap: A new form of self-adjusting heap. \textit{Algorithmica},
  1(1):111--129, 1986.
\bibitem[FT87]{FredmanTarjan1987} Michael L.~Fredman and Robert
  E.~Tarjan. Fibonacci heaps and their uses in improved network
  optimization algorithms. \textit{Journal of the ACM},
  34(3):596--615, July 1987.
\bibitem[FW76]{FriedmanWise1976} Daniel P.~Friedman and David
  S.~Wise. CONS should not evaluate its arguments. In
  \textit{Automata, Languages and Programming}, pages 257--281, July 1976.
\bibitem[GMPR77]{Guibas-etal1977} Leo J.~Guibas, Edward M.~McCreight,
  Michael F.~Plass, and Janet R.~Roberts. A new representation for
  linear lists. In \textit{ACM Symposium on Theory of Computing},
  pages 49--60, May 1977.
\bibitem[Gri81]{Gries1981} David Gries. \textit{The Science of
    Programming}. Texts and Monographs in Computer
  Science. Springer-Verlag, New York, 1981.
\bibitem[GS78]{GuibasSedgewick1978} Leo J.~Guibas and Robert
  Sedgewick. A dichromatic framework for balanced trees. In
  \textit{IEEE Symposium on Foundations of Computer Science}, pages
  8--21, October 1978.
\bibitem[GT86]{GajewskaTarjan1986} Hania Gajewska and Robert
  E.~Tarjan. Deques with heap order. \textit{Information Processing
    Letters}, 22(4):197--200, April 1986.
\bibitem[Hen93]{Henglein1993} Fritz Henglein. Type inference with
  polymorphic recursion. \textit{ACM Transactions on Programming
    Languages and Systems}, 15(2):253--289, April 1993.
\bibitem[HJ94]{HudakJones1994} Paul Hudak and Mark P.~Jones. Haskell
  vs. Ada vs. C$++$ vs. \ldots An experiment in software prototyping
  productivity, 1994.
\bibitem[HM76]{HendersonMorris1976} Peter Henderson and James
  H.~Morris, Jr. A lazy evaluator. In \textit{ACM Symposium on
    Principles of Programming Languages}, pages 95--103, January 1976.
\bibitem[HM81]{HoodMelville1981} Robert Hood and Robert
  Melville. Real-time queue operations in pure
  Lisp. \textit{Information Processing Letters}, 13(2): 50--53,
  November 1981.
\bibitem[Hoo82]{Hood1982} Robert Hood. \textit{The Efficient
    Implementation of Very-High-Level Programming Language
    Constructs.}\/ PhD thesis, Department of Computer Science, Cornell
  University, August 1982. (Cornell TR 82-503).
\bibitem[Hoo92]{Hoogerwoord1992} Rob R.~Hoogerwoord. A symmetric set
  of efficient list operations. \textit{Journal of Functional
    Programming}, 2(4):505--513, October 1992.
\bibitem[HU73]{HopcroftUllman1973} John E.~Hopcroft and Jeffrey
  D.~Ullman. Set merging algorithms. \textit{SIAM Journal on
    Computing}, 2(4):294--303, December 1973.
\bibitem[Hug85]{Hughes1985} John Hughes. Lazy memo functions. In
  \textit{Conference on Functional Programming Languages and Computer
    Architecture}, volume 201 of \textit{LNCS}, pages
  129--146. Springer-Verlag, September 1985.
\bibitem[Hug86]{Hughes1986} John Hughes. A novel representation of
  lists and its application to the function
  ``reverse''. \textit{Information Processing Letters},
  22(3):141--144, March 1986.
\bibitem[Hug89]{Hughes1989} John Hughes. Why functional programming
  matters. \textit{The Computer Journal}, 32(2):98--107, April 1989.
\bibitem[Joh86]{Jones1986} Douglas W.~Jones. An empirical comparison
  of priority-queue and event-set
  implementations. \textit{Communications of the ACM}, 29(4):300--311,
  April 1986.
\bibitem[Jos89]{Josephs1989} Mark B.~Josephs. The semantics of lazy
  functional languages. \textit{Theoretical Computer Science},
  68(1):105--111, October 1989.
\bibitem[KD96]{KaldewaijDielissen1996} Anne Kaldewaij and Victor
  J.~Dielissen. Leaf trees. \textit{Science of Computer Programming},
  26(1--3):149--165, May 1996.
\bibitem[Kin94]{King1994} David J.~King. Functional binomial
  queues. In \textit{Glasgow Workshop on Functional Programming},
  pages 141--150, September 1994.
\bibitem[KL93]{KhoongLeong1993} Chan Meng Khoong and Hon Wai
  Leong. Double-ended binomial queues. In \textit{International
    Symposium on Algorithms and Computation}, volume 762 of
  \textit{LNCS}, pages 128--137. Springer-Verlag, December 1993.
\bibitem[Knu73a]{Knuth1973a} Donald E.~Knuth. \textit{Searching and
    Sorting}, volume 3 of \textit{The Art of Computer
    Programming}. Addison-Wesley, 1973. Русский перевод: Дональд
  Э.~Кнут. \textit{Искусство программирования. Том 3: Сортировка и
    поиск.}\/ Вильямс, 2012.
\bibitem[Knu73b]{Knuth1973b} Donald E.~Knuth. \textit{Seminumerical
    Algorithms}, volume 2 of \textit{The Art of Computer
    Programming}. Addison-Wesley, 1973. Русский перевод: Дональд
  Э.~Кнут. \textit{Искусство программирования. Том 2: Получисленные
    алгоритмы.}\/ Вильямс, 2011.
\bibitem[KT95]{KaplanTarjan1995} Haim Kaplan and Robert
  E.~Tarjan. Persistent lists with catenation via recursive
  slow-down. In \textit{ACM Symposium on Theory of Computing}, pages
  93--102, May 1995.
\bibitem[KT96a]{KaplanTarjan1996a} Haim Kaplan and Robert
  E.~Tarjan. Purely functional lists with catenation via recursive
  slow-down. Draft revision of \cite{KaplanTarjan1995}, August 1996.
\bibitem[KT96b]{KaplanTarjan1996b} Haim Kaplan and Robert
  E.~Tarjan. Purely functional representation of catenable sorted
  lists. In \textit{ACM Symposium on Theory of Computing}, pages
  202--211, May 1996.
\bibitem[KTU93]{KfouryTiurynUrzyczyn1993} Assaf J.~Kfoury, Jerzy
  Tiuryn, and Pawel Urzyczyn. Type reconstruction in the presence of
  polymorphic recursion. \textit{ACM Transactions on Programming
    Languages and Systems}, 15(2):290--311, April 1993.
\bibitem[Lan65]{Landin1965} P.~J.~Landin. A correspondence between
  ALGOL 60 and Church's lambda-notation: Part
  I. \textit{Communications of the ACM}, 8(2):89--101, February 1965.
\bibitem[Lau93]{Launchbury1993} John Launchbury. A natural semantics
  for lazy evaluation. In \textit{ACM Symposium on Principles of
    Programming Languages}, pages 144--154, January 1993.
\bibitem[Lia92]{Liao1992} Andrew M.~Liao. Three priority queue
  applications revisited. \textit{Algorithmica}, 7(4):415--427, 1992.
\bibitem[LS81]{LeongSeiferas1981} Benton L.~Leong and Joel
  I.~Seiferas. New real-time simulations of multihead tape
  units. \textit{Journal of the ACM}, 28(1):166--180, January 1981.
\bibitem[MEP96]{MoffatEddyPetersson1996} Alistair Moffat, Gary Eddy,
  and Ola Petersson. Splaysort: Fast, versatile,
  practical. \textit{Software---Practice and Experience},
  26(7):781--797, July 1996.
\bibitem[Mic68]{Michie1968} Donald Michie. ``Memo'' functions and
  machine learning. \textit{Nature}, 218:19--22, April 1968.
\bibitem[MS91]{MoretShapiro1991} Bernard M.~E.~Moret and Henry
  D.~Shapiro. An empirical analysis of algorithms for constructing a
  minimum spanning tree. In \textit{Workshop on Algorithms and Data
    Structures}, volume 519 of \textit{LNCS}, pages
  400--411. Springer-Verlag, August 1991.
\bibitem[MT94]{MacQueenTofte1994} David B.~MacQueen and Mads Tofte. A
  semantics for higher-order functors. In \textit{European Symposium
    on Programming}, pages 409--423, April 1994.
\bibitem[MTHM97]{Milner-etal1997} Robin Milner, Mads Tofte, Robert
  Harper, and David MacQueen. \textit{The Definition of Standard ML
    (Revised)}. The MIT Press, Cambridge, Massachusetts, 1997.
\bibitem[Myc84]{Mycroft1984} Alan Mycroft. Polymorphic type schemes
  and recursive definitions. In \textit{International Symposium on
    Programming}, volume 167 of \textit{LNCS}, pages
  217--228. Springer-Verlag, April 1984.
\bibitem[Mye82]{Myers1982} Eugene W.~Myers. AVL dags. Technical Report
  TR82-9, Department of Computer Science, University of Arizona, 1982.
\bibitem[Mye83]{Myers1983} Eugene W.~Myers. An applicative
  random-access stack. \textit{Information Processing Letters},
  17(5):241--248, December 1983.
\bibitem[Mye84]{Myers1984} Eugene W.~Myers. Efficient applicative data
  types. In \textit{ACM Symposium on Principles of Programming
    Languages}, pages 66--75, January 1984.
\bibitem[NPP95]{NunezPalaoPena1995} Manuel N\'u\~nez, Pedro Palao, and
  Ricardo Pe\~na. A second year course on data structures based on
  functional programming. In \textit{Functional Programming Languages
    in Education}, volume 1022 of \textit{LNCS}, pages
  65--84. Springer-Verlag, December 1995.
\bibitem[Oka95a]{Okasaki1995a} Chris Okasaki. Amortization, lazy
  evaluation, and persistence: Lists with catenation via lazy
  linking. In \textit{IEEE Symposium on Foundations of Computer
    Science}, pages 646--654, October 1995.
\bibitem[Oka95b]{Okasaki1995b} Chris Okasaki. Purely functional
  random-access lists. In \textit{Conference on Functional Programming
  Languages and Computer Architecture}, pages 86--95, June 1995.
\bibitem[Oka95c]{Okasaki1995c} Chris Okasaki. Simple and efficient
  purely functional queues and deques. \textit{Journal of Functional
    Programming}, 5(4):583--592, October 1995.
\bibitem[Oka96a]{Okasaki1996a} Chris Okasaki. \textit{Purely
    Functional Data Structures.}\/ PhD thesis, School of Computer
  Science, Carnegie Mellon University, September 1996.
\bibitem[Oka96b]{Okasaki1996b} Chris Okasaki. The role of lazy
  evaluation in amortized data structures. In \textit{ACM SIGPLAN
    International Conference on Functional Programming}, pages 62--72,
  May 1996.
\bibitem[Oka97]{Okasaki1997} Chris Okasaki. Catenable double-ended
  queues. In \textit{ACM SIGPLAN International Conference on
    Functional Programming}, pages 64--74, June 1997.
\bibitem[OLT94]{OkasakiLeeTarditi1994} Chris Okasaki, Peter Lee, and
  David Tarditi. Call-by-need and continuation-passing
  style. \textit{Lisp and Symbolic Computation}, 7(1):57--81, January 1994.
\bibitem[Ove83]{Overmars1983} Mark H.~Overmars. \textit{The Design of
    Dynamic Data Structures}, volume 156 of
  \textit{LNCS}. Springer-Verlag, 1983.
\bibitem[Pau96]{Paulson1996} Laurence C.~Paulson. \textit{ML for the
    Working Programmer.}\/ Cambridge University Press, 2nd edition, 1996.
\bibitem[Pet87]{Peterson1987} Gery L.~Peterson. A balanced tree scheme
  for meldable heaps with updates. Technical Report GIT-ICS-87-23,
  School of Information and Computer Science, Georgia Institute of
  Technology, 1987.
\bibitem[Pip96]{Pippenger1996} Nicholas Pippinger. Pure versus impure
  Lisp. In \textit{ACM Symposium on Principles of Programming
    Languages}, pages 104--109, January 1996.
\bibitem[PPN96]{PalaoGostanzaPenaNunez1996} Pedro Palao Gostanza, Ricardo
  Pe\~na, and Manuel Nu\~nez. A new look at pattern matching in
  abstract data types. In \textit{ACM SIGPLAN International Conference
  on Functional Programming}, pages 110--121, May 1996.
\bibitem[Ram92]{Raman1992} Rajeev Raman. \textit{Eliminating
    Amortization: On Data Structures with Guaranteed Response
    Times.}\/ PhD thesis, Department of Computer Sciences, University
  of Rochester, October 1992.
\bibitem[Rea92]{Reade1992} Chris M.~P.~Reade. Balanced trees with
  removals: an exercise in rewriting and proof. \textit{Science of
    Computer Programming}, 18(2):181--204, April 1992.
\bibitem[San90]{Sands1990} David Sands. Complexity analysis for a lazy
  higher-order language. In \textit{European Symposium on
    Programming}, volume 432 of \textit{LNCS}, pages
  361--376. Springer-Verlag, May 1990.
\bibitem[San95]{Sands1995} David Sands. A na\"\i{}ve time analysis and
  its theory of cost equivalence. \textit{Journal of Logic and
    Computation}, 5(4):495--541, August 1995.
\bibitem[Sar86]{Sarnak1986} Neil Sarnak. \textit{Persistent Data
    Structures.}\/ PhD thesis, Department of Computer Sciences, New
  York university, 1986.
\bibitem[Sch92]{Schoenmakers1992} Berry Schoenmakers. \textit{Data
    Structures and Amortized Complexity in a Functional Setting.}\/
  PhD thesis, Eindhoven University of Technology, September 1992.
\bibitem[Sch93]{Schoenmakers1993} Berry Schoenmakers. A systematic
  analysis of splaying. \textit{Information Processing Letters},
  45(1):41--50, January 1993.
\bibitem[Sch97]{Schwenke1997} Martin Schwenke. High-level refinement
  of random access data structures. In \textit{Formal Methods
    Pacific}, pages 317--318, July 1997.
\bibitem[SS90]{SackStrothotte1990} J\"org-R\"udiger Sack and Thomas
  Strothotte. A characterization of heaps and its
  applications. \textit{Information and Computation}, 86(1):69--86,
  May 1990.
\bibitem[ST85]{SleatorTarjan1985} Daniel D.~K.~Sleator and Robert
  E.~Tarjan. Self-adjusting binary search trees. \textit{Journal of
    the ACM}, 32(3):652--686, July 1985.
\bibitem[ST86a]{SarnakTarjan1986a} Neil Sarnak and Robert
  E.~Tarjan. Planar point location using persistent search
  trees. \textit{Communications of the ACM}, 29(7):669--679, July 1986.
\bibitem[ST86b]{SleatorTarjan1986b} Daniel D.~K.~Sleator and Robert E.~Tarjan.
  Self-adjusting heaps. \textit{SIAM Journal on Computing},
  15(1):52--69, February 1986.
\bibitem[Sta88]{Stankovic1988} John A.~Stankovic. Misconceptions about
  real-time computing: A serious problem for next-generation
  systems. \textit{Computer}, 21(10):10--19, October 1988.
\bibitem[Sto70]{Stoss1970} Hans-J\"org Sto\ss{}. K-band simulation von
  k-Kopf-Turing\-machinen. \textit{Computing}, 6(3):309--317, 1970.
\bibitem[SV87]{StaskoVitter1987} John T.~Stasko and Jeffrey
  S.~Vitter. Pairing heaps: experiments and
  analysis. \textit{Communications of the ACM}, 30(3):234--249, March 1987.
\bibitem[Tar83]{Tarjan1983} Robert E.~Tarjan. \textit{Data Structures
    and Network Algorithms}, volume 44 of \textit{CBMS Regional
    Conference Series in Applied Mathematics.}\/ Society for
  Industrial and Applied Mathematics, Philadelphia, 1983.
\bibitem[Tar85]{Tarjan1985} Robert E.~Tarjan. Amortized computational
  complexity. \textit{SIAM Journal on Algebraic and Discrete Methods},
  6(2):306--318, April 1985.
\bibitem[TvL84]{TarjanvanLeeuwen1984} Robert E.~Tarjan and Jan van
  Leeuwen. Worst-case analysis of set union
  algorithms. \textit{Journal of the ACM}, 31(2):245--281, April 1984.
\bibitem[Ull94]{Ullman1994} Jeffrey D.~Ullman. \textit{Elements of ML
    Programming.}\/ Prentice Hall, Englewood Cliffs, New Jersey, 1994.
\bibitem[Vui74]{Vuillemin1974} Jean Vuillemin. Correct and optimal
  implementations of recursion in a simple programming
  language. \textit{Journal of Computer and System Sciences},
  9(3):332--354, December 1974.
\bibitem[Vui78]{Vuillemin1978} Jean Vuillemin. A data structure for
  manipulating priority queues. \textit{Communications of the ACM},
  21(4):309--315, April 1978.
\bibitem[Wad71]{Wadsworth1971} Christopher
  P.~Wadsworth. \textit{Semantics and Pragmatics of the Lambda
    Calculus.}\/ PhD thesis, University of Oxford, September 1971.
\bibitem[Wad87]{Wadler1987}Philip Wadler. Views: A way for
  pattern-matching to cohabit with data abstraction. In \textit{ACM
    Symposium on Principles of Programming Languages}, pages 307--313,
  January 1987.
\bibitem[Wad88]{Wadler1988} Philip Wadler. Strictness analysis aids
  time analysis. In \textit{ACM Symposium on Principles of Programming
  Languages}, pages 119--132, January 1988.
\bibitem[WV86]{vanWykVitter1986} Christopher van Wyk and Jeffrey Scott
  Vitter. The complexity of hashing with lazy
  deletion. \textit{Algorithmica}, 1(1):17--29, 1986.
\end{thebibliography}

%%% Local Variables: 
%%% mode: latex
%%% TeX-master: "pfds"
%%% End: 
