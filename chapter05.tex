\chapter{Основы амотризации}
\label{ch:5}

В последние пятнадцать лет амортизация стала мощным инструментом при
построении и анализе структур данных. Реализации с амортизированными
характеристиками производительности часто оказываются проще и быстрее,
чем реализации со сравнимыми жесткими характеристиками. В этой главе
мы даем обзор основных методов амортизации и иллюстрируем эти идеи
через простую реализацию очередей FIFO и несколько реализаций кучи.

К сожалению, простой подход к амортизации, рассматриваемый в этой
главе, конфликтует с идеей устойчивости~--- эти структуры, будучи
используемы как устойчивые, могут быть весьма неэффективны. Однако на
практике многие приложения устойчивости не требуют, и часто для таких
приложений реализации, представленные в этой главе, могут быть
замечательным выбором. В следующей главе мы увидим, как можно
подружить понятия амортизации и устойчивости при помощи ленивого
вычисления.

\section{Методы амортизированного анализа}
\label{sc:5-1}

%%% Local Variables: 
%%% mode: latex
%%% TeX-master: "pfds"
%%% End: 
