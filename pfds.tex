\documentclass{book}
\usepackage[utf8]{inputenc}
\usepackage[russian]{babel}
\usepackage{amsmath}
\usepackage{amssymb}
\usepackage{listings}

\usepackage[arrow,curve,matrix,frame]{xy}

\lstset{language=ml,mathescape=true}

\newcommand{\term}[2]{\textit{#1} (#2)}

\newcommand{\concat}{\ensuremath{+\!\!\!+\,}}

\newtheorem{remark}{\textbf{Замечание}}[chapter]
\newtheorem{exercise}{\textbf{Упражнение}}[chapter]

\newtheorem{hint}{\textbf{Указание разработчикам}}[chapter]

\newtheorem{theorem}{\textbf{Теорема}}[chapter]
\newtheorem{lemma}[theorem]{\textbf{Лемма}}

\author{Крис Окасаки}
\title{Чисто функциональные структуры данных}

\begin{document}
\maketitle

\chapter*{Предисловие}

Я впервые познакомился с языком Стандартный ML в 1989 году. Мне всегда
нравилось программировать эффективные реализации структур данных,
и я немедленно занялся переводом некоторых своих любимых программ
на Стандартный ML. Для некоторых структур перевод оказался достаточно
простым и, к моему большому удовольствию, получался код значительно более краткий
и ясный, чем предыдущие версии, написанные мной на C, Pascal или
Ada.  Однако не всегда результат оказывался столь приятным. Раз за
разом мне приходилось использовать разрушающее присваивание, которое в
Стандартном ML не приветствуется, а во многих других функциональных
языках вообще запрещено.  Я пытался обращаться к литературе, но
нашел лишь несколько разрозненных статей.  Понемногу я стал понимать,
что столкнулся с неисследованной областью, и начал искать новые
способы решения задач.

Сейчас, восемь лет спустя, мой поиск продолжается. Всё ещё есть много
примеров структур данных, которые я просто не знаю как эффективно
реализовать на функциональном языке. Однако за это время я получил
множество уроков о том, что в функциональных языках
\textit{работает}.  Эта книга является попыткой записать выученные
уроки, и я надеюсь, что она послужит справочником для функциональных
программистов, а также как текст для тех, кто хочет больше узнать о
структурах данных в функциональном окружении.

\textbf{Стандартный ML.} Несмотря на то, что структуры данных из этой
книги можно реализовать практически на любом функциональном языке, я
во всех примерах буду использовать Стандартный ML.  У этого языка
имеются следующие преимущества для моих целей: (1)  аппликативный
порядок вычислений, что значительно упрощает рассуждения о том,
сколько времени потребует тот или иной алгоритм, и (2) замечательная
система модулей, идеально подходящая для выражения абстрактных типов
данных.  Однако пользователи других языков, например, Haskell или
Lisp, смогут без труда адаптировать мои примеры к своим вычислительным
окружениям. (В приложении я привожу переводы большинства примеров на
Haskell.) Даже программисты на C или Java должны быть способны
реализовать эти структуры данных, хотя в случае C отсутствие
автоматической сборки мусора иногда будет доставлять неприятности.

Читателям, незнакомым со Стандартным ML, я рекомендую в качестве
введения книги \textit{ML для программиста-практика} Полсона
\cite{Paulson96} или \textit{Элементы программирования на ML}
Ульмана \cite{Ullman94}.

\textbf{Прочие предварительные требования.} Эта книга не рассчитана
служить первоначальным общим введением в структуры данных. Я
предполагаю, что читателю достаточно знакомы основные абстрактные
структуры данных~--- стеки, очереди, кучи (приоритетные очереди) и
конечные отображения (словари).  Кроме того, я предполагаю знакомство
с основами анализа алгоритмов, особенно с нотацией <<большого O>>
(напр., $O(n \log n)$). Обычно эти вопросы рассматриваются во втором
курсе для студентов, изучающих информатику.

\textbf{Благодарности.} Мое понимание функциональных структур данных
чрезвычайно обогатилось в результате дискуссий со многими
специалистами на протяжении многих лет.  Мне бы особенно хотелось
поблагодарить Питера Ли, Генри Бейкера, Герта Бродала, Боба Харпера,
Хаима Каплана, Грэма Мосса, Саймона Пейтон Джонса и Боба Тарждана.



%%% Local Variables:
%%% mode: latex
%%% TeX-master: "pfds"
%%% End:


\chapter{Введение}
\label{ch:1}

Когда программисту на C для решения определенной задачи требуется
эффективная структура данных, часто он или она могут просто найти
подходящее решение в одном из многих учебников или справочников. К
сожалению, для программистов на функциональных языках вроде
Стандартного ML или Haskell такая роскошь недоступна.  Хотя большинство
справочников стараются быть независимы от языка, независимость эта
получается только в смысле Генри Форда: программисты свободны выбрать
любой язык, если язык этот императивный.\footnote{%
  Генри Форд однажды сказал о цветах автомобилей Модели T:
  <<[Покупатели] могут выбрать любой цвет, при условии, что он черный.>>
}
Чтобы несколько исправить этот дисбаланс, в этой книге я рассматриваю
структуры данных с функциональной точки зрения. В примерах программ я
использую Стандартный ML, однако эти программы нетрудно перевести на
другие функциональные языки, например, Haskell или Lisp. Версии наших
программ на Haskell можно найти в Приложении~\ref{app:A}.

\section{Функциональные и императивные структуры данных}

Методологические преимущества функциональных языков хорошо известны
\cite{Backus1978,Hughes1989,HudakJones1994}, но тем не менее
большинство программ по-прежнему пишутся на императивных языках вроде
C. Кажущееся противоречие легко объяснить тем, что исторически
функциональные языки проигрывали в скорости своим более традиционным
аналогам, однако этот разрыв сейчас сужается.  По широкому фронту
задач был достигнут впечатляющий прогресс, начиная от базовой техники
построения компиляторов и заканчивая глубоким анализом и оптимизацией
программ.  Однако одну особенность функционального программирования не
исправить никакими ухищрениями со стороны авторов компиляторов~---
использование слабых или несоответствующих задаче структур данных. К
сожалению, имеющаяся литература содержит относительно мало рецептов
помощи в этой области.

Почему оказывается, что функциональные структуры данных труднее
спроектировать и реализовать, чем императивные? Здесь две основные
проблемы. Во-первых, с точки зрения проектирования и реализации
эффективных структур данных, запрет функционального программирования
на деструктивное обновление (т.,~е., присваивание) является
существенным препятствием, подобно запрету для повара использовать
ножи. Как и ножи, деструктивные обновления при неправильном
употреблении опасны, но, будучи пущены в дело должным образом,
чрезвычайно эффективны.  Императивные структуры данных часто
существенным образом полагаются на присваивание, так что в
функциональных программах приходится искать другие подходы.

Второе затруднение состоит в том, что от функциональных структур
ожидается большая гибкость, чем от их императивных аналогов.  В
частности, когда мы производим обновление императивной структуры
данных, мы, как правило, принимаем как данность, что старая версия
данных более недоступна, в то время как при обновлении функциональной
структуры мы ожидаем, что как старая, так и новая версия доступны для
дальнейшей обработки. Структура данных, поддерживающая несколько
версий, называется \term{устойчивой}{persistent}, в то время как
структура данных, позволяющая иметь лишь одну версию в каждый момент
времени, называется \term{эфемерной}{ephemeral}
\cite{DSST1989}. Функциональные языки программирования обладают тем
интересным свойством, что \emph{все} структуры данных в них
автоматически устойчивы. Императивные структуры данных, как правило,
эфемерны. В тех случаях, когда требуется устойчивая структура,
императивные программисты не удивляются, что она получается более
сложной и, возможно, даже асимптотически менее эффективной, чем
эквивалентная эфемерная структура.

Более того, теоретики установили нижние границы, которые показывают,
что в некоторых ситуациях функциональные языки по своей природе менее
эффективны, чем императивные \cite{BAG1992, Pippenger1996}.  В свете
перечисленного, функциональные структуры данных иногда кажутся
похожими на танцующего медведя, о котором говорится: <<удивительно не
то, что он танцует как-то особенно хорошо, а то, что он вообще
танцует!>> Однако на практике ситуация совсем не так безнадежна. Как
мы увидим, часто оказывается возможным построить функциональные
структуры данных, асимптотически столь же эффективные, как лучшие
императивные решения.

\section{Аппликативное и ленивое вычисление}

Большинство (последовательных) функциональных языков программирования
можно отнести либо к \term{аппликативным}{strict}, либо к
\term{ленивым}{lazy}, в зависимости от порядка вычислений.  Какой из
этих порядков предпочтительнее~--- тема, обсуждаемая функциональными
программистами подчас с религиозным жаром.  Различие между двумя
порядками вычисления наиболее ярко проявляется в подходах к вычислению
аргументов функции. В аппликативных языках аргументы вычисляются
прежде тела функции. В ленивых языках вычисление аргументов управляется
потребностью; исходно они передаются в функцию в невычисленном виде, и
вычисляются только тогда, когда (и если!) их значение нужно
для продолжения работы.  Кроме того, после однократного вычисления
значение аргумента кэшируется, так что если оно потребуется снова, его
можно получить из памяти, а не перевычислять заново.  Такое
кэширование называется \term{мемоизация}{memoization}
\cite{Michie1968}.

Каждый из этих порядков имеет свои достоинства и недостатки, но
аппликативное вычисление явно удобнее по крайней мере в одном
отношении: с ним проще рассуждать об асимптотичееской сложности
вычислений.  В аппликативных языках то, какие именно подвыражения
будут вычислены и когда, ясно по большей части уже из синтаксиса.
Таким образом рассуждения о времени выполнения каждой данной программы
относительно просты.  В то же время в ленивых языках даже эксперты
часто испытывают сложности при ответе на вопрос, когда будет вычислено
данное подвыражение и будет ли вычислено вообще.  Программисты на
таких языках часто вынуждены притворяться, что язык на самом деле
аппликативен, чтобы получить хотя бы грубые оценки времени работы.

Оба порядка вычисления влияют на проектирование и анализ структур
данных. Как мы увидим, аппликативные языки могут описать структуры с
жесткой оценкой времени выполнения в худшем случае, но не с амортизированной
оценкой, а в ленивых языках описываются амортизированные структуры
данных, но не рассчитанные на худший случай. Чтобы описывать обе
разновидности структур, требуется язык, поддерживающий оба
порядка вычислений. Мы получаем такой язык, расширяя Стандартный ML
примитивами для ленивого вычисления, как описано в Главе~\ref{ch:4}.

\section{Терминология}

Любой разговор о структурах данных содержит опасность возникновения путаницы,
поскольку у термина \term{структура данных}{data structure} есть по
крайней мере четыре различных связанных между собой значения.

\begin{itemize}
\item \emph{Абстрактный тип данных} (то есть,
  \emph{тип и набор функций над этим типом}). Для этого значения мы
  будем пользоваться словом \term{абстракция}{abstraction}.
\item \emph{Конкретная реализация абстрактного типа данных}. Для этого
  значения мы используем слово
  \term{реализация}{implementation}. Однако от реализации мы не требуем
  воплощения в коде~--- достаточно детального проекта.
\item \emph{Экземпляр типа данных, например, конкретный список или
    дерево}. Для такого экземпляра мы будем использовать слово
  \term{объект}{object} или \term{версия}{version}. Впрочем,
  для конкретных типов часто бывает свой термин. Например, стеки и
  очереди мы будем называть просто стеками и очередями.
\item \emph{Сущность, сохраняющая свою идентичность при
    изменениях}. Например, в интерпретаторе, построенном на основе
  стека, мы часто говорим о <<стеке>>, как если бы это был один
  объект, а не различные версии в различные моменты времени. Для этого
  значения мы будем использовать выражение \term{устойчивая
    сущность}{persistent identity}.  Нужда в этом возникает прежде
  всего при разговоре об устойчивых структурах данных; когда мы
  говорим о различных версиях одной и той же структуры, мы имеем в
  виду, что они все имеют одну и ту же устойчивую сущность.
\end{itemize}
Грубо говоря, абстракциям в Стандартном ML соответствуют сигнатуры,
реализациям~--- структуры или функторы, а объектам или версиям~---
значения. Хорошего аналога понятию устойчивой сущности в Стандартном
ML нет.\footnote{%
  Устойчивая сущность эфемерной структуры данных может быть
  реализована как ссылочная ячейка, но для моделирования устойчивой
  сущности устойчивой структуры данных такого подхода недостаточно.
}

Термин \term{операция}{operation} перегружен подобным же образом; он
обозначает и функции, предоставляемые абстрактным типом данных, и
конкретные применения этих функций. Мы пользуемся словом
\emph{операция} только во втором значении, а для первого употребляем
слова \term{функция}{function} или \term{оператор}{operator}.

\section{Наш подход}

Вместо того, чтобы каталогизировать структуры данных, подходящие для каждой
возможной задачи (безнадежное предприятие!), мы сосредоточим внимание на нескольких
общих методиках проектирования эффективных функциональных структур
данных, и каждую такую методику будем иллюстрировать одной или
несколькими реализациями базовых абстракций, таких, как
последовательность, куча (очередь с приоритетами) или структуры для
поиска.  Когда читатель овладел этими методиками, он сможет с
легкостью их приспособить к собственным нуждам, или даже
спроектировать новые структуры с нуля.

\section{Обзор книги}

Книга состоит из трех частей. Первая (Главы~\ref{ch:2} и \ref{ch:3})
служит введением в функциональные структуры данных.
\begin{itemize}
\item В Главе~\ref{ch:2} обсуждается, как функциональные структуры
  данных добиваются устойчивости.
\item Глава~\ref{ch:3} описывает три хорошо известных структуры
  данных~--- кучи со смещением влево,
%% Термин!!!
  биномиальные кучи и красно-черные деревья,~--- и показывает, как их
  можно реализовать на Стандартном ML.
\end{itemize}
Вторая часть (Главы~\ref{ch:4}--\ref{ch:7}) посвящена соотношению
между ленивым вычислением и амортизацией.
\begin{itemize}
\item В Главе~\ref{ch:4} кратко рассматриваются основные понятия
  ленивого вычисления и вводится синтаксис, которым мы пользуемся для
  описания ленивых вычислений в Стандартном ML.
\item Глава~\ref{ch:5} служит введением в основные методы
  амортизации. Объясняется, почему эти методы не работают при
  анализе устойчивых структур данных.
\item Глава~\ref{ch:6} описывает связующую роль, которую ленивое
  вычисление играет при сочетании амортизации и устойчивости, и дает
  два метода анализа амортизированной стоимости структур данных,
  реализованных через ленивое вычисление.
\item В Главе~\ref{ch:7} демонстрируется, какую выразительную мощь дает
  сочетание аппликативного и ленивого вычисления в одном языке.
  Мы показываем, как во многих случаях можно получить структуру данных
  с жесткими характеристиками производительности из структуры с
  амортизированными характеристиками, если систематически запускать
  преждевременное вычисление ленивых компонент структуры.
\end{itemize}
В третьей части книги (Главы~\ref{ch:8}--\ref{ch:11}) исследуется
несколько общих методик построения функциональных структур данных.
\begin{itemize}
\item В Главе~\ref{ch:8} описывается \term{ленивая перестройка}{lazy
    rebuilding}, вариант идеи \term{глобальной перестройки}{global
    rebuilding} \cite{Overmars1983}.  Ленивая перестройка значительно
  проще глобальной, но в результате получаются структуры с
  амортизированными, а не с жесткими характеристиками.  Сочетание
  ленивой перестройки с методиками планирования из Главы~\ref{ch:7}
  часто позволяет восстановить жесткие характеристики.
\item В Главе~\ref{ch:9} исследуются \term{числовые
    представления}{numerical representations}~--- представления
  данных, построенные по аналогии с представлениями чисел (как
  правило, двоичных чисел). В этой модели нахождение эффективных
  процедур вставки и изъятия соответствует выбору таких вариантов
  двоичных чисел, в которых сложение или вычитание занимает
  константное время.
\item Глава~\ref{ch:10} рассматривает \term{развертку структур
    данных}{data-structural bootstrapping} \cite{Buchsbaum1993}. Эта
  методика существует в трех вариантах: \term{структурная
    декомпозиция}{structural decomposition}, когда решения без
  ограничений строятся на основе ограниченных решений,
  \term{структурная абстракция}{structural abstraction}, когда
  эффективные решения строятся на основе неэффективных, и
  \term{развёртка до составных типов}{bootstrapping to aggregate
    types}, когда реализации с атомарными элементами развёртываются до
  реализаций с составными элементами.
\item В Главе~\ref{ch:11} описывается \term{неявное рекурсивное
    замедление}{implicit recursive slowdown}, ленивый вариант метода
  \term{рекурсивного замедления}{recursive slowdown} Каплана и
  Тарджана \cite{KaplanTarjan1995}.  Подобно ленивой перестройке,
  неявное рекурсивное замедление значительно проще обычного
  рекурсивного замедления, но вместо жестких характеристик дает лишь
  амортизированные. Как и в случае ленивой перестройки, часто жесткие
  характеристики можно восстановить через планирование.
\end{itemize}

Наконец, Приложение~\ref{app:A} включает в себя перевод большинства
программных реализаций этой книги на Haskell.

%%% Local Variables:
%%% mode: latex
%%% TeX-master: "pfds"
%%% End:

\chapter{Устойчивость}
\label{ch:2}

Отличительной особенностью функциональных структур данных является то,
что они всегда \term{устойчивы}{persistent}~--- обновление
функциональной структуры не уничтожает старую версию, а создает
новую, которая с ней сосуществует. Устойчивость достигается путем
\emph{копирования} затронутых узлов структуры данных, и все изменения
проводятся на копии, а не на оригинале. Поскольку узлы никогда
напрямую не модифицируются, все незатронутые узлы могут
\term{совместно использоваться}{be shared} между старой и новой версией структуры
данных без опасения, что изменения одной версии непроизвольно окажутся
видны другой.

В этой главе мы исследуем подробности копирования и совместного использования для
двух простых структур данных: списков и двоичных деревьев.

\section{Списки}
\label{sc:2.1}

Мы начинаем с простых связанных списков, часто встречающихся в
императивном программировании и вездесущих в функциональном.  Основные
функции, поддерживаемые списками, в сущности те же, что и для
абстракции стека, описанной в виде сигнатуры на Стандартном ML на
Рис.~\ref{fig:2.1}.  Списки и стеки можно тривиально реализовать либо
с помощью встроенного типа <<список>> (Рис.~\ref{fig:2.2}), либо как
отдельный тип (Рис.~\ref{fig:2.3}).

\begin{remark}
Сигнатура на Рис.~\ref{fig:2.1} использует терминологию списков
(\texttt{cons}, \texttt{head}, \texttt{tail}), а не стеков
(\texttt{push}, \texttt{top}, \texttt{pop}), потому что мы
рассматриваем списки как частный случай общего класса
последовательностей. Другими примерами этого класса являются
\emph{очереди}, \emph{двусторонние очереди} и \emph{списки с
  конкатенацией}. Для функций во всех этих абстракциях мы используем
одинаковые соглашения по именованию, чтобы можно было заменять одну
реализацию другой с минимальными трудностями.
\end{remark}

\begin{figure}
  \centering

  \caption{Сигнатура для стеков.}
  \label{fig:2.1}
\end{figure}
\begin{figure}
  \centering

  \caption{Реализация стека с помощью встроенного типа списков.}
  \label{fig:2.2}
\end{figure}
\begin{figure}
  \centering

  \caption{Реализация стека в виде отдельного типа.}
  \label{fig:2.3}
\end{figure}

К этой сигнатуре мы могли бы добавить ещё одну часто встречающуюся
операцию на списках: $\concat$, которая конкатенирует (т. е.,
соединяет) два списка. В императивной среде эту функцию нетрудно
поддержать за время $O(1)$, если сохранять указатели и на первый, и на
последний элемент списка.  Тогда $\concat$ просто изменяет последнюю
ячейку первого списка так, чтобы она указывала на первую ячейку
второго списка.  Результат этой операции графически показан на
Рис.~\ref{fig:2.4}. Обратите внимание, что эта операция
\emph{уничтожает} оба своих аргумента~--- после выполнения
\lstinline!xs $\concat$ ys! ни \lstinline!xs!, ни \lstinline!ys! использовать
больше нельзя.

\begin{figure}
  \centering
  (до)\\
  (после)\\
  \caption{Выполнение \lstinline!xs $\concat$ ys! в императивной среде. Эта операция уничтожает списки-аргументы \lstinline!xs! и \lstinline!ys!.}
  \label{fig:2.4}
\end{figure}

В функциональном окружении мы не можем деструктивно модифицировать
последнюю ячейку первого списка. Вместо этого мы копируем эту ячейку и
модифицируем хвостовой указатель в ячейке-копии. Затем мы копируем
предпоследнюю ячейку и модифицируем ее хвостовой указатель, указывая
на копию последней ячейки.  Такое копирование продолжается, пока
не окажется скопирован  весь список. Процесс в общем виде можно
реализовать как
\begin{lstlisting}
  fun xs$\concat$ys = if isEmpty xs then ys else cons (head xs, tail xs$\concat$ys)
\end{lstlisting}
Если у нас есть доступ к реализации нашей структуры (например, в виде
встроенных списков Стандартного ML), мы можем переписать эту функцию
через сопоставление с образцом:
\begin{lstlisting}
  fun []$\concat$ys = ys
    | (x :: xs)$\concat$ys = x :: (xs$\concat$ys)
\end{lstlisting}
На Рис.~\ref{fig:2.5} изображен результат конкатенации двух
списков. Обратите внимание, что после выполнения операции мы можем
продолжать использовать два исходных списка, \lstinline!xs! и
\lstinline!ys!. Таким образом, мы добиваемся устойчивости, но за счет
копирования ценой $O(n)$.\footnote{%
  В Главах~\ref{ch:10} и \ref{ch:11} мы увидим, как можно поддержать
  $\concat$ за время $O(1)$ без потери устойчивости.
}

\begin{figure}
  \centering
  (до)\\
  (после)\\
  \caption{Выполнение \lstinline!zs = xs$\concat$ys! в функциональной среде. Заметим, что списки-аргументы \lstinline!xs! и \lstinline!ys! не затронуты операцией.}
  \label{fig:2.5}
\end{figure}

Несмотря на большой объем копирования, заметим, что второй список, \lstinline!ys!, нам
копировать не пришлось. Эти узлы теперь общие между
\lstinline!ys! и \lstinline!zs!. Ещё одна функция, иллюстрирующая
парные понятия копирования и общности подструктур~---
\lstinline!update!, изменяющая значение узла списка по данному
индексу. Эту функцию можно реализовать как
\begin{lstlisting}
  fun update ([], i, y) = raise Subscript
    | update (x::xs, 0, y) = y::xs
    | update (x::xs, i, y) = x::update(xs, i-1, y)
\end{lstlisting}
Здесь мы не копируем весь список-аргумент. Копировать приходится
только сам узел, подлежащий модификации (узел $i$) и узлы,
содержащие прямые или косвенные указатели на $i$.  Другими словами,
чтобы изменить один узел, мы копируем все узлы на пути от корня
к изменяемому. Все узлы, не находящиеся на этом пути, используются как
исходной, так и обновленной версиями. На Рис.~\ref{fig:2.6} показан
результат изменения третьего узла в пятиэлементном списке: первые
три узла копируются, а последние два используются совместно.

\begin{figure}
  \centering

  \caption{Выполнение \lstinline!ys = update(xs, 2, 7)!. Обратите
    внимание на совместное использование структуры списками \lstinline!xs! и \lstinline!ys!.}
  \label{fig:2.6}
\end{figure}

\begin{remark}
  Такой стиль программирования очень сильно упрощается при наличии
  автоматической сборки мусора. Очень важно освободить память от тех
  копий, которые больше не нужны, но многочисленные совместно используемые
  узлы делают ручную сборку мусора нетривиальной задачей.
\end{remark}

\begin{exercise}\label{ex:2.1}
  Напишите функцию \lstinline!suffixes! типа
  \lstinline!$\alpha$ list $\to$ $\alpha$ list list!, которая принимает как
  аргумент список \lstinline!xs! и возвращает список всех его
  суффиксов в убывающем порядке длины. Например,
  \begin{lstlisting}
    suffixes [1,2,3,4] = [[1,2,3,4],[2,3,4],[3,4],[4],[]]
  \end{lstlisting}
  Покажите, что список суффиксов можно породить за время $O(n)$ и
  занять при этом $O(n)$ памяти.
\end{exercise}

\section{Двоичные деревья поиска}
\label{sc:2.2}

Если узел структуры содержит более одного указателя, оказываются
возможны более сложные сценарии совместного использования памяти. Хорошим примером
совместного использования такого вида служат двоичные деревья поиска.

Двоичные деревья поиска~--- это двоичные деревья, в которых элементы
хранятся во внутренних узлах в \term{симметричном}{symmetric}
порядке, то есть, элемент в каждом узле больше любого элемента в
левом поддереве этого узла и меньше любого элемента в правом
поддереве. В Стандартном ML мы представляем двоичные деревья поиска
при помощи следующего типа:
\begin{lstlisting}
  datatype Tree = E | T of Tree $\times$ Elem $\times$ Tree
\end{lstlisting}
где \lstinline!Elem!~--- какой-либо фиксированный полностью упорядоченный
тип элементов.

\begin{remark}
  Двоичные деревья поиска не являются полиморфными относительно типа
  элементов, поскольку в качестве элементов не может выступать любой
  тип~--- подходят только типы, снабженные отношением полного
  порядка. Однако это не значит, что для каждого типа элементов мы
  должны заново реализовывать деревья двоичного поиска. Вместо этого
  мы делаем тип элементов и прилагающиеся к нему функции сравнения
  параметрами \term{функтора}{functor}, реализующего двоичные деревья
  поиска (см. Рис.~\ref{fig:2.9}).
\end{remark}

Мы используем это представление для реализации множеств. Однако оно
легко адаптируется и для других абстракций (например, конечных
отображений) или поддержки более изысканных функций (скажем, найти
$i$-й по порядку элемент), если добавить в конструктор \lstinline!T!
дополнительные поля.

На Рис.~\ref{fig:2.7} показана минимальная сигнатура для множеств. Она
содержит значение <<пустое множество>>, а также функции добавления
нового элемента и проверки на членство.  В более практической
реализации, вероятно, будут присутствовать и многие другие функции,
например, для удаления элемента или перечисления всех элементов.

\begin{figure}
  \centering

  \caption{Сигнатура для множеств.}
  \label{fig:2.7}
\end{figure}

Функция \lstinline!member! ищет в дереве, сравнивая запрошенный
элемент с находящимся в корне дерева. Если запрошенный элемент меньше
корневого, мы рекурсивно ищем в левом поддереве. Если он больше,
рекурсивно ищем в правом поддереве. Наконец, в оставшемся случае
запрошенный элемент равен корневому, и мы возвращаем значение
<<истина>>. Если мы когда-либо натыкаемся на пустое дерево, значит,
запрашиваемый элемент не является членом множества, и мы возвращаем
значение <<ложь>>.  Эта стратегия реализуется так:
\begin{lstlisting}
  fun member(x,E) = false
    | member(x,T(a,y,b)) =
       if x < y then member(x,a)
       else if x > y then member(x,b)
       else true
\end{lstlisting}
\begin{remark}
  Простоты ради, мы предполагаем, что функции сравнения называются $<$
  и $>$. Однако если эти функции передаются в качестве параметров
  функтора, как на Рис.~\ref{fig:2.9}, часто оказывается удобнее
  называть их именами вроде \lstinline!lt! или \lstinline!leq!, а
  символы $<$ и $>$ оставить для сравнения целых и других элементарных
  типов.
\end{remark}

Функция \lstinline!insert! проводит поиск в дереве по той же стратегии,
что и \lstinline!member!, но только по пути она копирует каждый
элемент. Когда наконец оказывается достигнут пустой узел, он
заменяется на узел, содержащий новый элемент.
\begin{lstlisting}
  fun insert(x,E) = T(E,x,E)
    | insert(x, s as T(a,y,b)) =
       if x < y then T(insert(x,a),y,b)
       else if x > y then T(a,y,insert(x,b))
       else s
\end{lstlisting}
На Рис.~\ref{fig:2.8} показана типичная вставка. Каждый скопированный
узел использует одно из поддеревьев совместно с исходным деревом; речь о том поддереве,
которое не оказалось на пути поиска. Для большинства деревьев путь
поиска содержит лишь небольшую долю узлов в дереве. Громадное
большинство узлов находятся в совместно используемых поддеревьях.

\begin{figure}
  \centering

  \caption{Выполнение \lstinline!ys = insert("e", xs)!. Как и прежде,
    обратите внимание на совместное использвание структуры деревьями \lstinline!xs! и \lstinline!ys!.}
  \label{fig:2.8}
\end{figure}

На Рис.~\ref{fig:2.9} показано, как двоичные деревья поиска можно
реализовать в виде функтора на Стандартном ML. Функтор принимает тип
элементов и связанные с ним функции сравнения как параметры. Поскольку
часто те же самые параметры будут использоваться и другими функторами
(см., например, Упражнение~\ref{ex:2.6}), мы упаковываем их в
структуру с сигнатурой \lstinline!Ordered!.

\begin{figure}
  \centering
  (* Полностью упорядоченный тип и его функции сравнения *)
  \caption{Реализация двоичных деревьев поиска в виде функтора на Стандартном ML.}
  \label{fig:2.9}
\end{figure}

\begin{exercise}\textbf{Андерсон \cite{Andersson1991}}\label{ex:2.2}
В худшем случае \lstinline!member! производит $2d$ сравнений, где
$d$~--- глубина дерева. Перепишите ее так, чтобы она делала не более
$d+1$ сравнений, сохраняя элемент, который \emph{может} оказаться
равным запрашиваемому (например, последний элемент, для которого
операция $<$ вернула значение <<истина>> или $\le$~--- <<ложь>>, и
производя проверку на равенство только по достижении дна дерева.
\end{exercise}

\begin{exercise}\label{ex:2.3}
  Вставка уже существующего элемента в двоичное дерево поиска копирует
  весь путь поиска, хотя скопированные узлы неотличимы от
  исходных. Перепишите \lstinline!insert! так, чтобы она избегала
  копирования с помощью исключений. Установите только один обработчик
  исключений для всей операции поиска, а не по обработчику на итерацию.
\end{exercise}

\begin{exercise}\label{ex:2.4}
  Совместите улучшения из предыдущих двух упражнений, и получите
  версию \lstinline!insert!, которая не делает ненужного копирования и
  использует не более $d+1$ сравнений.
\end{exercise}

\begin{exercise}\label{ex:2.5}
  Совместное использование может быть полезно и внутри одного объекта, не
  обязательно между двумя различными.  Например, если два поддерева
  одного дерева идентичны, их можно представить одним и тем же
  деревом.
  \begin{enumerate}
  \item Используя эту идею, напишите функцию \lstinline!complete! типа
    \lstinline!Elem $\times$ Int $\to$ Tree!, такую, что
    \lstinline!complete(x,d)! создает полное двоичное дерево глубины
    \lstinline!d!, где в каждом узле содержится \lstinline!x!.
    (Разумеется, такая функция бессмысленна для абстракции множества,
    но она может оказаться полезной для какой-либо другой абстракции,
    например, мультимножества.) Функция должна работать за время $O(d)$.
  \item Расширьте свою функцию, чтобы она строила сбалансированные
    деревья произвольного размера. Эти деревья не всегда будут полны,
    но они должны быть как можно более сбалансированными: для любого
    узла размеры поддеревьев должны различаться не более чем на
    единицу. Функция должна работать за время $O(\log n)$. (Подсказка:
    воспользуйтесь вспомогательной функцией \lstinline!create2!,
    которая, получая размер $m$, создает пару деревьев~--- одно размера
    $m$, а другое размера $m+1$)
  \end{enumerate}
\end{exercise}

\begin{exercise}\label{ex:2.6}
  Измените функтор \lstinline!UnbalancedSet! так, чтобы он служил
  реализацией не множеств, а \term{конечных отображений}{finite maps}. На
  Рис.~\ref{fig:2.10} приведена минимальная сигнатура для конечных
  отображений. (Заметим, что исключение \lstinline!NotFound! не
  является встроенным в Стандартный ML~--- Вам придется его определить
  самостоятельно. Это исключение можно было бы сделать частью
  сигнатуры \lstinline!FiniteMap!,  чтобы каждая реализация
  определяла собственное исключение \lstinline!NotFound!, но удобнее,
  если все конечные отображения будут использовать одно и то же
  исключение.)
\end{exercise}

\begin{figure}
  \centering

  (* Если ключ не найден, возбудить \lstinline!NotFound! *)

  \caption{Сигнатура для конечных отображений.}
  \label{fig:2.10}
\end{figure}

\section{Примечания}
\label{sc:2.3}

Майерс \cite{Myers1982,Myers1984} использовал копирование и совместное использование
при реализации двоичных деревьев поиска (в его случае это были
AVL-деревья).  Для общего метода реализации устойчивых структур данных
путем копирования затронутых узлов
Сарнак и Тарджан \cite{SarnakTarjan1986a} выбрали термин
\term{копирование путей}{path copying}. Существуют также другие методы
реализации устойчивых структур данных, предложенные Дрисколлом,
Сарнаком, Слитором и Тарджаном \cite{Driscoll-etal1989} и Дитцем
\cite{Dietz1989}, но эти методы не являются чисто функциональными.

%%% Local Variables:
%%% mode: latex
%%% TeX-master: "pfds"
%%% End:

\chapter{Некоторые известные структуры данных в функциональном
  окружении}
\label{ch:3}

Хотя реализовать в функциональной среде многие императивные структуры
данных трудно или невозможно, есть и такие, которые реализуются без
особых усилий.  В этой главе мы рассматриваем три структуры данных,
которым обычно учат в императивном контексте. Первая из них,
левоориентированные кучи, просто устроена и в том, и в другом
окружении. Однако две других, биномиальные очереди и красно-чёрные
деревья, часто считаются сложными для понимания, поскольку
их императивные реализации быстро превращаются в мешанину манипуляций
с указателями.  Напротив, функциональные реализации этих структур
данных абстрагируются от действий с указателями и прямо отражают
высокоуровневые представления. Дополнительное преимущество
функциональной реализации этих структур состоит в том, что мы
бесплатно получаем устойчивость.

\section{Левоориентированные кучи}
\label{sc:3.1}

Как правило, множества и конечные отображения поддерживают эффективный
доступ к произвольным элементам. Однако иногда требуется эффективный
доступ только к \emph{минимальному} элементу.  Структура данных,
поддерживающая такой режим доступа, называется \term{очередь с
  приоритетами}{priority queue} или \term{куча}{heap}.  Чтобы избежать
путаницы с очередями FIFO, мы будем использовать второй из этих
терминов. На Рис.~\ref{fig:3.1} приведена простая сигнатура для кучи.

\begin{figure}
\begin{lstlisting}
signature HEAD = sig
  structure Elem: ORDERED
  type Heap

  val empty     : Heap
  val isEmpty   : Heap $\rightarrow$ bool
  val insert    : Elem.T $\times$ Heap $\rightarrow$ Heap
  val merge     : Heap $\times$ Heap $\rightarrow$ Heap

  val findMin   : Heap $\rightarrow$ Elem.T /* raises EMPTY if heap is empty */
  val deleteMin : Heap $\rightarrow$ Elem.T /* raises EMPTY if heap is empty */
end
\end{lstlisting}
% TODO: understand what is wrong with comments
\caption{Сигнатура для кучи (очереди с приоритетами).} \label{fig:3.1}
\end{figure}

\begin{remark}
  Сравнивая сигнатуру кучи с сигнатурой множества
  (Рис.~\ref{fig:2.7}), мы видим, что для кучи отношение порядка на
  элементах включено в сигнатуру, а для множества нет.  Это различие
  вытекает из того, что отношение порядка играет важную роль в
  семантике кучи, а в семантике множества не играет.  С другой
  стороны, можно утверждать, что в семантике множества большую роль
  играет отношение \emph{равенства}, и оно должно быть включено в
  сигнатуру.
\end{remark}

Часто кучи реализуются через деревья \term{с порядком
  кучи}{heap-ordered}, т.~е., в которых элемент при каждой вершине не
больше элементов в поддеревьях. При таком упорядочении минимальный
элемент дерева всегда находится в корне.

Левоориентированные кучи \cite{Crane1972, Knuth1973a} представляют
собой двоичные деревья с порядком кучи, обладающие свойством
\term{левоориентированности}{leftist property}: ранг любого левого поддерева
не меньше ранга его сестринской правой вершины.  Ранг узла
определяется как длина его \term{правой периферии}{right spine}
(т.~е., самого правого пути от данного узла до пустого).  Простым
следствием свойства левоориентированности является то, что правая
периферия любого узла~--- кратчайший путь от него к пустому узлу.

\begin{exercise}\label{ex:3.1}
  Докажите, что правая периферия левоориентированной кучи размера $n$
  всегда содержит не более $\lfloor \log(n+1) \rfloor$ элементов. (В
  этой книге все логарифмы, если не указано обратного, берутся по
  основанию 2.)
\end{exercise}

Если у нас есть некоторая структура упорядоченных элементов
\lstinline!Elem!, мы можем представить левоориентированные кучи как
двоичные деревья, снабженные информацией о ранге.
\begin{lstlisting}
  datatype Heap = E | T of int $\times$ Elem.T $\times$ Heap $\times$ Heap
\end{lstlisting}
Заметим, что элементы правой периферии левоориентированной кучи (да и
любого дерева с порядком кучи) расположены в порядке возрастания.
Главная идея левоориентированной кучи заключается в том, что для
слияния двух куч достаточно слить их правые периферии как
упорядоченные списки, а затем вдоль полученного пути обменивать
местами поддеревья при вершинах, чтобы восстановить свойство
левоориентированности.  Это можно реализовать следующим образом:
\begin{lstlisting}
  fun merge (h, E) = h
    | merge (E, h) = h
    | merge (h$_1$ as T(_, x, a$_1$, b$_1$), h$_2$ as T(_, y, a$_2$, b$_2$)) =
       if Elem.leq (x, y) then makeT (x, a$_1$, merge (b$_1$, h$_2$))
       else makeT (y, a$_2$, merge (h$_1$, b$_2$))
\end{lstlisting}
где \lstinline!makeT!~--- вспомогательная функция, вычисляющая ранг
вершины \lstinline!T! и, если необходимо, меняющая местами ее
поддеревья.
\begin{lstlisting}
  fun rank (E) = 0
    | rank (T (r, _, _, _)) = r
  fun makeT (x, a, b) = if rank a $\ge$ rank b then T (rank b + 1, x, a, b)
                                           else T (rank a + 1, x, b, a)
\end{lstlisting}
Поскольку длина правой периферии любой вершины в худшем случае
логарифмическая, \lstinline!merge! выполняется за время $O(\log n)$.

Теперь, когда у нас есть эффективная функция \lstinline!merge!,
оставшиеся функции не представляют труда: \lstinline!insert! создает
одноэлементную кучу и сливает ее с существующей, \lstinline!findMin!
возвращает корневой элемент, а \lstinline!deleteMin! отбрасывает
корневой элемент и сливает его поддеревья.
\begin{lstlisting}
  fun insert (x, h) = merge (T (1, x, E, E), h)
  fun findMin (T (_, x, a, b)) = x
  fun deleteMin (T (_, x, a, b)) = merge (a, b)
\end{lstlisting}
Поскольку \lstinline!merge! выполняется за время $O(\log n)$, столько
же занимают и \lstinline!insert! с \lstinline!deleteMin!. Очевидно,
что \lstinline!findMin! выполняется за $O(1)$. Полная реализация
левоориентированных куч приведена на Рис.~\ref{fig:3.2} в виде
функтора, принимающего в качестве параметра структуру упорядоченных
элементов.

\begin{remark}
  Чтобы не перегружать примеры мелкими деталями, мы обычно в
  фрагментах кода пропускаем варианты, ведущие к ошибкам. Например,
  приведенные выше фрагменты не показывают поведение
  \lstinline!findMin! и \lstinline!deleteMin! на пустых кучах.  Когда
  дело доходит до полной реализации, как на Рис.~\ref{fig:3.2}, мы
  всегда включаем в нее разбор ошибок.
\end{remark}

\begin{figure}
\begin{lstlisting}
functor LeftistHeap(Element: ORDERED) : Heap = struct
  structure Elem = Element

  datatype Heap = E | T of int $\times$ Elem.T $\times$ Heap $\times$ Heap

  fun rank E = 0
    | rank (T(r,_,_,_)) = r
  fun makeT (x,a,b) = if rank a $\geq$ rank b then T(rank b+1, x,a,b)
                       else T(rank a + 1, x,b,a)

  val empty = E
  fun isEmpty E = true
    | isEmpty _ = false

  fun merge (h,E) = h
    | merge (E,h) = h
    | merge ($h_1$ as T(_,x,$a_1$,$b_1$), h_2 as T(_,y,$a_2$,$b_2$)) =
       if Elem.leq (x,y) then makeT(x,$a_1$, merge($b_1$,$h_2$)
       else makeT(y,$a_2$,merge($h_1$,$b_2$))

  fun insert (x,h) = merge (T(1,x,E,E),h)
  fun findMin E = raise EMPTY
    | findMin (T(_,x,a,b)) = x
  fun deleteMin E = raise EMPTY
    | deleteMin (T(_,x,a,b)) = merge(a,b)
end
\end{lstlisting}


  \caption{Левоориентированные кучи.}
  \label{fig:3.2}
\end{figure}

\begin{exercise}\label{ex:3.2}
  Определите \lstinline!insert! напрямую, а не через обращение к \lstinline!merge!.
\end{exercise}

\begin{exercise}\label{ex:3.3}
  Реализуйте функцию \lstinline!fromList! типа \lstinline!Elem.T list $\to$ Heap!,
  порождающую левоориентированную кучу из неупорядоченного списка
  элементов путем преобразования каждого элемента в одноэлементную
  кучу, а затем слияния получившихся куч, пока не останется
  одна. Вместо того, чтобы сливать кучи проходом слева направо или
  справа налево при помощи \lstinline!foldr! или \lstinline!foldl!,
  слейте кучи за $\lceil \log n \rceil$ проходов, где на каждом
  проходе сливаются пары соседних куч. Покажите, что
  \lstinline!fromList! требует всего $O(n)$ времени.
\end{exercise}

\begin{exercise}\label{ex:3.4}
  \textbf{(Чо и Сахни \cite{ChoSahni1996})} Левоориентированные кучи
  со сдвинутым весом~--- альтернатива левоориентированным кучам, где
  вместо свойства левоориентированности используется свойство
  \term{левоориентированности, сдвинутой по весу}{weight-biased leftist
    property}: размер любого левого поддерева всегда не меньше размера
  соответствующего правого поддерева.
  \begin{enumerate}
  \item Докажите, что правая периферия левоориентированной кучи со
    сдвинутым весом содержит не более $\lfloor \log(n+1) \rfloor$ элементов.
  \item Измените реализацию на Рис.~\ref{fig:3.2}, чтобы получились
    левоориентированные кучи со сдвинутым весом.
  \item Функция \lstinline!merge! сейчас выполняется в два прохода:
    сверху вниз, с вызовами \lstinline!merge!, и снизу вверх, с
    вызовами вспомогательной функции \lstinline!makeT!. Измените
    \lstinline!merge! для левоориентированных куч со сдвинутым весом
    так, чтобы она работала за один проход сверху вниз.
  \item Каковы преимущества однопроходной версии \lstinline!merge! в
    условиях ленивого вычисления? В условиях параллельного вычисления?
  \end{enumerate}
\end{exercise}

\section{Биномиальные кучи}
\label{sc:3.2}

Биномиальные очереди \cite{Vuillemin1978, Brown1978}, которые мы,
чтобы избежать путаницы с очередями FIFO, будем называть \term{ биномиальными
кучами}{binomial heaps}~--- ещё одна распространенная реализация
куч. Биномиальные кучи устроены сложнее, чем левоориентированные, и, на
первый взгляд, не возмещают эту сложность никакими
преимуществами. Однако в последующих главах мы увидим, как в различных
вариантах биномиальных куч можно заставить \lstinline!insert! и
\lstinline!merge! выполняться за время $O(1)$.

Биномиальные кучи строятся из более простых объектов, называемых
биномиальными деревьями. Биномиальные деревья индуктивно определяются
так:
\begin{itemize}
\item Биномиальное дерево ранга 1 представляет собой одиночный узел.
\item Биномиальное дерево ранга $r+1$ получается путем
  \term{связывания}{linking} двух биномиальных деревьев ранга $r$, так
  что одно из них становится самым левым потомком второго.
\end{itemize}
Из этого определения видно, что биномиальное дерево ранга $r$ содержит
ровно $2^r$ элементов.  Существует второе, эквивалентное первому,
определение биномиальных деревьев, которым иногда удобнее
пользоваться: биномиальное дерево ранга $r$ представляет собой узел
с $r$ потомками $t_1\ldots t_r$, где каждое $t_i$ является
биномиальным деревом ранга $r-i$.  На Рис.~\ref{fig:3.3} показаны
биномиальные деревья рангов от 0 до 3.

\begin{figure}[h]
  \centering
  \begin{tikzpicture}[thick,scale=0.5, every node/.style={scale=0.5},grow via three points={%
one child at (0,-1.5) and two children at (0,-1.5) and (-0.8,-1.5)}
]
    \tikzstyle{marrs}=[very thick,-latex]
    \tikzstyle{tnode}=[circle, fill=black, inner sep=1.5mm]
    \def\rstep{5cm}
    
    \huge
    
    \node[tnode] (0, 0) {};
            child { node[tnode] {} }
            child { node[tnode] {} };
    
    \begin{scope}
        \draw (0, 1) node {Ранг 0};
        \node[tnode] (0, 0) {};
    \end{scope}
    
    \begin{scope}[xshift=\rstep]
        \draw (0, 1) node {Ранг 1};
        \node[tnode] {}
            child {node[tnode] {} };
    \end{scope}
    
    \begin{scope}[xshift=2 * \rstep]
        \draw (0, 1) node {Ранг 2};
        \node[tnode] {}
            child {node[tnode] {} }
            child {node[tnode] {} 
                    child {node[tnode] {} }};
            
            
    \end{scope}
    
    \begin{scope}[xshift=3 * \rstep]
        \draw (0, 1) node {Ранг 3};
        \node[tnode] {}
            child {node[tnode] {} }
            child {node[tnode] {} 
                child {node[tnode] {} }}
            child {node[tnode] {} 
                child {node[tnode] {} 
                    child {node[tnode] {} }}
                child {node[tnode] {} }};
            
            
    \end{scope}
    
    
\end{tikzpicture}
  \caption{Биномиальные деревья рангов 0--3.}
  \label{fig:3.3}
\end{figure}

Мы представляем вершину биномиального дерева в виде элемента и списка
его потомков. Для удобства мы также помечаем каждый узел его рангом.
\begin{lstlisting}
  datatype Tree = Node of int $\times$ Elem.T $\times$ Tree list
\end{lstlisting}
Каждый список потомков хранится в убывающем порядке рангов, а элементы
хранятся с порядком кучи.  Чтобы сохранять этот порядок, мы всегда
привязываем дерево с большим корнем к дереву с меньшим корнем.
\begin{lstlisting}
  fun link (t$_1$ as Node (r, x$_1$, c$_1$), t$_2$ as Node (_, x$_2$, c$_2$)) =
        if Elem.leq (x$_1$, x$_2$) then Node (r+1, x$_1$, t$_2$ :: c$_1$)
        else Node (r+1, x$_2$, t$_1$ :: c$_2$
\end{lstlisting}
Связываем мы всегда деревья одного ранга.

Теперь определяем биномиальную кучу как коллекцию биномиальных
деревьев, каждое из которых имеет порядок кучи, и никакие два дерева
не совпадают по рангу. Мы представляем эту коллекцию в виде списка
деревьев в возрастающем порядке ранга.
\begin{lstlisting}
  Type Heap = Tree list
\end{lstlisting}
Поскольку каждое биномиальное дерево содержит $2^r$ элементов, и
никакие два дерева по рангу не совпадают, деревья размера $n$ в
точности соответствуют единицам в двоичном представлении
$n$. Например, число 21 в двоичном виде выглядит как 10101, и поэтому
биномиальная куча размера 21 содержит одно дерево ранга 0, одно ранга
2, и одно ранга 4 (размерами, соответственно, 1, 4 и 16). Заметим, что
так же, как двоичное представление $n$ содержит не более $\lfloor log
(n+1)\rfloor$ единиц, биномиальная куча размера $n$ содержит не более
$\lfloor log(n+1) \rfloor$ деревьев.

Теперь мы готовы описать функции, действующие на биномиальных
деревьях. Начинаем мы с \lstinline!insert! и \lstinline!merge!,
которые определяются примерно аналогично сложению двоичных чисел. (Мы
укрепим эту аналогию в Главе~\ref{ch:9}.) Чтобы внести элемент в кучу,
мы сначала создаем одноэлементное дерево (т.~е., биномиальное дерево
ранга 0), затем поднимаемся по списку существующих деревьев в порядке
возрастания рангов, связывая при этом одноранговые деревья. Каждое
связывание соответствует переносу в двоичной арифметике.
\begin{lstlisting}
  fun rank (Node (r, x, c)) = r
  fun insTree (t,[]) = [t]
    | insTree (t, ts as t' :: ts') =
       if rank t < rank t' then t :: ts else insTree (link (t, t'), ts')
  fun insert (x, ts) = insTree (Node (0, x, []), ts)
\end{lstlisting}
В худшем случае, при вставке в кучу размера $n = 2^k -1$, требуется
$k$ связываний и $O(k) = O(\log n)$ времени.

При слиянии двух куч мы проходим через оба списка деревьев в порядке
возрастания ранга и связываем по пути деревья равного ранга. Как и
прежде, каждое связывание соответствует переносу в двоичной
арифметике.
\begin{lstlisting}
  fun merge (ts$_1$, []) = ts$_1$
    | merge ([], ts$_2$) = ts$_2$
    | merge (ts$_1$ as t$_1$ :: ts'$_1$, ts$_2$ as t$_2$ :: ts'$_2$) =
       if rank t$_1$ < rank t$_2$ then t$_1$ :: merge (ts'$_1$, ts$_2$)
       else if rank t$_2$ < rank t$_1$ then merge (ts$_1$, ts'$_2$)
       else insTree (link (t$_1$, t$_2$), merge (ts'$_1$, ts'$_2$))
\end{lstlisting}

Функции \lstinline!findMin! и \lstinline!deleteMin! вызывают
вспомогательную функцию \lstinline!removeMinTree!, которая находит
дерево с минимальным корнем, исключает его из списка и возвращает как
это дерево, так и список оставшихся деревьев.
\begin{lstlisting}
  fun removeMinTree [t] = (t, [])
    | removeMinTree (t :: ts) =
        let val (t', ts') = removeMinTree ts
        in if Elem.leq (root t, root t') then (t, ts) else (t', t :: ts') end
\end{lstlisting}
Функция \lstinline!findMin! просто возвращает корень найденного дерева
\begin{lstlisting}
  fun findMin ts = let val (t, _) = removeMinTree ts in root t end
\end{lstlisting}
Функция \lstinline!deleteMin! устроена немного похитрее. Отбросив
корень найденного дерева, мы ещё должны вернуть его потомков в список
остальных деревьев. Заметим, что список потомков \emph{почти} уже
соответствует определению биномиальной кучи. Это коллекция
биномиальных деревьев с неповторяющимися рангами, но только
отсортирована она не по возрастанию, а по убыванию ранга. Таким
образом, обратив список потомков, мы преобразуем его в биномиальную
кучу, а затем сливаем с оставшимися деревьями.
\begin{lstlisting}
  fun deleteMin ts = let val (Node (_, x, ts$_1$), ts$_2$) = removeMinTree ts
                     in merge (rev ts$_1$, ts$_2$) end
\end{lstlisting}
Полная реализация биномиальных куч приведена на
Рис.~\ref{fig:3.4}. Все четыре основные операции в худшем случае
требуют $O(\log n)$ времени.

\begin{figure}
\begin{lstlisting}
functor BinomialHeap(Element: ORDERED) : Heap = struct
  structure Elem = Element
  datatype Tree = Node of int $\times$ Elem.T $\times$ Tree list
  datatype Heap = Tree list

  val empty = []
  val isEmpty ts = null ts

  fun rank (Node(r,x,c)) = r
  fun root (Node(r,x,c)) = x

  fun link ($t_1$ as Node ($r_1$,$x_1$,$c_1$), $t_2$ as Node ($r_2$,$x_2$,$c_2$)) =
    if Elem.leq ($x_1$,$x_2$) then Node(r+1, $x_1$, $t_2$::$c_1$)
    else Node(r+1, $x_2$, $t_1$::$c_2$)

  fun insTree (t,[]) = [t]
    | insTree (t, ts as t'::ts') =
       if rank t < rank t' then t::ts else insTree(link(t,t'), ts')

  fun insert (x,ts) = insTree(Node(0,x,[]), ts)

  fun merge ($ts_1$, []) $=$  $ts_1$
    | merge ([], $ts_2$) $=$  $ts_2$
    | merge ($ts_1$ as $t_1$::$ts_1'$, $ts_2$ as $t_2$::$ts_2'$) =
       if rank $t_1$ < rank $t_2$ then $t_1$ :: merge($ts_1$,$ts_2'$)
       else if rank $t_2$ < rank $t_1$ then $t_2$ :: merge($ts_2$, $ts_1'$)
       else insTree(link($t_1$,$t_2$), merge($ts_1'$,$ts_2'$)

  fun removeMinTree [] = raise EMPTY
    | removeMinTree [t] = (t,[])
    | removeMinTree (t::ts) =
      let val ($t'$, $ts'$) $=$ removeMinTree ts
      in if Elem.leq (root t, root t') then (t,ts) else (t', t::ts') end

  fun findMin ts = let val (t,_) = removeMinTree ts in root t end

  fun deleteMin ts =
    let val (Node(_, x,$ts_1$), $ts_2$) $=$ removeMinTree ts
    in merge (rev $ts_1$,$ts_2$) end
end
\end{lstlisting}
\centering

  \caption{Биномиальные кучи.}
  \label{fig:3.4}
\end{figure}

\begin{exercise}\label{ex:3.5}
  Определите \lstinline!findMin! напрямую, без обращения к \lstinline!removeMinTree!.
\end{exercise}

\begin{exercise}\label{ex:3.6}
  Большая часть аннотаций ранга в нашем представлении биномиальных куч
  излишня, потому что мы и так знаем, что дети узла ранга $r$ имеют
  ранги $r-1, \ldots, 0$. Таким образом, можно исключить
  поле-аннотацию ранга из узлов, а вместо этого помечать ранг корня
  каждого дерева, т.~е.,
  \begin{lstlisting}
    datatype Tree = Node of Elem $\times$ Tree list
    type Heap = (int $\times$ Tree) list
  \end{lstlisting}
  Реализуйте биномиальные кучи в таком представлении.
\end{exercise}

\begin{exercise}\label{ex:3.7}
  Одно из основных преимуществ левоориентированных куч над
  биномиальными заключается в том, что \lstinline!findMin! занимает в
  них $O(1)$ времени, а не $O(\log n)$. Следующая заготовка функтора
  улучшает время \lstinline!findMin! до $O(1)$, сохраняя минимальный
  элемент отдельно от остальной кучи.
  \begin{lstlisting}
    functor ExplicitMin (H : Heap) : Heap =
    struct
          structure Elem = H.Elem
          datatype Heap = E | NE of Elem.t $\times$ H.Heap
          ...
    end
  \end{lstlisting}
  Заметим, что этот функтор не ограничен биномиальными кучами, а
  принимает любую реализацию куч в качестве параметра. Закончите этот
  функтор так, чтобы \lstinline!findMin! требовал время $O(1)$, а
  функции \lstinline!insert!, \lstinline!merge! и
  \lstinline!deleteMin! каждая по $O(\log n)$. Предполагается, что
  нижележащая реализация \lstinline!H! для всех операций занимает
  $O(\log n)$.
\end{exercise}

\section{Красно-чёрные деревья}
\label{sc:3.3}

В разделе~\ref{sc:2.2} мы описали двоичные деревья поиска. Такие
деревья хорошо ведут себя на случайных или неупорядоченных данных,
однако на упорядоченных данных их производительность резко падает, и
каждая операция может занимать до $O(n)$  времени.  Решение этой
проблемы состоит в том, чтобы каждое дерево поддерживать в
приблизительно сбалансированном состоянии. Тогда каждая операция
выполняется не хуже, чем за время $O(\log n)$.  Одним из наиболее
популярных семейств сбалансированных двоичных деревьев поиска являются
красно-чёрные \cite{GuibasSedgewick1978}.

Красно-чёрное дерево представляет собой двоичное дерево поиска, в
котором каждый узел окрашен либо красным, либо чёрным. Мы добавляем
поле цвета в тип двоичных деревьев поиска из раздела~\ref{sc:2.2}.
\begin{lstlisting}
  datatype Color = R | B
  datatype Tree = E | T of Color $\times$ Tree $\times$ Elem $\times$ Tree
\end{lstlisting}
Все пустые узлы считаются чёрными, поэтому пустой конструктор
\lstinline!E! в поле цвета не нуждается.

Мы требуем, чтобы всякое красно-чёрное дерево соблюдало два
инварианта:
\begin{itemize}
\item \textbf{Инвариант 1.} У красного узла не может быть красного ребёнка.
\item \textbf{Инвариант 2.} Каждый путь от корня дерева до пустого
  узла содержит одинаковое количество чёрных узлов.
\end{itemize}
Вместе эти два инварианта гарантируют, что самый длинный возможный
путь по красно-чёрному дереву, где красные и чёрные узлы чередуются,
не более чем вдвое длиннее самого короткого, состоящего только из
чёрных узлов.

\begin{exercise}\label{ex:3.8}
  Докажите, что максимальная глубина узла в красно-чёрном дереве
  размера $n$ не превышает $2 \lfloor \log (n+1) \rfloor$.
\end{exercise}

Функция \lstinline!member! для красно-чёрных деревьев не обращает
внимания на цвета. За исключением заглушки в варианте для конструктора
\lstinline!T!, она не отличается от функции \lstinline!member! для
несбалансированных деревьев.
\begin{lstlisting}
  fun member (x, E) = false
    | member (x, T (_, a, y, b) =
       if x < y then member (x, a)
       else if x > y then member (x, b)
       else true
\end{lstlisting}
Функция \lstinline!insert! более интересна, поскольку она должна
поддерживать два инварианта баланса.
\begin{lstlisting}
  fun insert (x, s) =
        let fun ins E = T (R, E, x, E)
              | ins (s as T (color, a, y, b)) =
                  if x < y then balance (color, ins a, y, b)
                  else if x > y then balance (color, a, y, ins b)
                  else s
            val T (_, a, y, b) = ins s  (* $\mbox{гарантированно непустое}$ *)
        in T (B, a, y, b)
\end{lstlisting}
Эта функция содержит три существенных изменения по сравнению с \lstinline!insert! для
несбалансированных деревьев поиска. Во-первых, когда мы создаем новый
узел в ветке \lstinline!ins E!, мы сначала окрашиваем его в красный
цвет. Во-вторых, независимо от цвета, возвращаемого \lstinline!ins!,
в окончательном результате мы корень окрашиваем чёрным. Наконец, в
ветках \lstinline!x < y! и \lstinline!x > y! мы вызовы конструктора
\lstinline!T! заменяем на обращения к функции
\lstinline!balance!. Функция \lstinline!balance! действует подобно
конструктору \lstinline!T!, но только она переупорядочивает свои
аргументы, чтобы обеспечить выполнение инвариантов баланса.

Если новый узел окрашен красным, мы сохраняем Инвариант 2, но в
случае, если отец нового узла тоже красный, нарушается Инвариант 1. Мы
временно позволяем существовать одному такому нарушению, и переносим
его снизу вверх по мере перебалансирования. Функция
\lstinline!balance! обнаруживает и исправляет красно-красные нарушения,
когда обрабатывает чёрного родителя красного узла с красным
ребёнком. Такая чёрно-красно-красная цепочка может возникнуть в
четырёх различных конфигурациях, в зависимости от того, левым или
правым ребёнком является каждая из красных вершин. Однако в каждом из
этих случаев решение одно и то же: нужно преобразовать
чёрно-красно-красный путь в красную вершину с двумя чёрными детьми,
как показано на Рис.~\ref{fig:3.5}.  Это преобразование можно записать
так:
\begin{lstlisting}
  fun balance (B,T (R,T (R,a,x,b),y,c),z,d) = T (R, T (B,a,x,b),T (B,c,z,d))
    | balance (B,T (R,a,x,T (R,b,y,c)),z,d) = T (R, T (B,a,x,b),T (B,c,z,d))
    | balance (B,a,x,T (R,T (R,b,y,c),z,d)) = T (R, T (B,a,x,b),T (B,c,z,d))
    | balance (B,a,x,T (R,b,y,T (R,c,z,d))) = T (R, T (B,a,x,b),T (B,c,z,d))
    | balance body = T body
\end{lstlisting}
Нетрудно проверить, что в получающемся поддереве будут соблюдены оба
инварианта красно-чёрного баланса.

\begin{figure}[h]
  \centering
  \begin{tikzpicture}[thick,scale=0.5, every node/.style={scale=0.5},level distance=2.5cm, sibling distance=2cm]
    \tikzstyle{tblack}=[circle, line width=1mm, draw=black]
    \tikzstyle{tred}=[circle, draw=black]
    \def\xstep{7cm}
    \def\ystep{10cm}
    
    \huge
    
    \begin{scope}[xshift=7cm, yshift=7cm]
        \def\inse{3.5mm}
     %   \draw (-1, -2.5) rectangle (5, 1);
        
        \node[tblack, inner sep=\inse] at (0,0) {};
        \node[tred, inner sep=\inse] at (0,-1.6) {};
        \node[right=1pt] at (1,0) { -- черный};
        \node[right=1pt] at (1,-1.6) { -- красный};
    \end{scope}
    
    \begin{scope}[yshift=\ystep]
        \node[tblack] {z}
            child { node[tred] {x}
                child { node {a} }
                child { node[tred] {y}
                    child {node {b}}
                    child {node {c}}
                }
            }
            child { node {d} };
    \end{scope}
    
    \begin{scope}[xshift=\xstep]
        \node[tblack] {x}
            child { node {a} }
            child { node[tred] {y}
                child { node {b} }
                child { node[tred] {z}
                    child {node {c}}
                    child {node {d}}
                }
            };
    \end{scope}
    
    \begin{scope}[xshift=-\xstep]
        \node[tblack] {z}
            child { node[tred] {y}
                child { node[tred] {x}
                    child {node {a}}
                    child {node {b}}
                }
                child { node {c} }
            }
            child { node {d} };
    \end{scope}
    
    \begin{scope}[yshift=-\ystep]
        \node[tblack] {x}
            child { node {a} }
            child { node[tred] {z}
                child { node[tred] {y}
                    child {node {b}}
                    child {node {c}}
                }
                child { node {d} }
            };
    \end{scope}
    
    \begin{scope}[yshift=-1.5cm]
        \tikzstyle{level 1}=[sibling distance=3cm]
        \tikzstyle{level 2}=[sibling distance=2cm]
        \node[tred] {y}
            child { node[tblack] {x}
                child { node {a} }
                child { node {b} }
            }
            child { node[tblack] {z}
                child { node {c} }
                child { node {d} }
            };
    \end{scope}
    \Huge
    \draw (0, 0.5cm) node[rotate=-90] {$\Rightarrow$};
    \draw (0, -8cm) node[rotate=90] {$\Rightarrow$};
    \draw (-4cm, -4cm) node[rotate=0] {$\Rightarrow$};
    \draw (4cm, -4cm) node[rotate=180] {$\Rightarrow$};


    
\end{tikzpicture}

  \caption{Избавление от красных узлов с красными родителями.}
  \label{fig:3.5}
\end{figure}

\begin{remark}
  Заметим, что в первых четырех строках \lstinline!balance! правые
  части одинаковы. В некоторых реализациях Стандартного ML, в
  частности, в Нью-Джерсийском Стандартном ML (Standard ML of New
  Jersey), поддерживается расширение, называемое
  \term{или-образцы}{or-patterns}, позволяющее слить несколько
  вариантов с одинаковыми правыми сторонами в один
  \cite{FahndrichBoyland1997}. С использованием или-образцов можно
  переписать функцию \lstinline!balance! так:
  \begin{lstlisting}
    fun balance ( (B,T (R,T (R,a,x,b),y,c),z,d)
                | (B,T (R,a,x,T (R,b,y,c)),z,d)
                | (B,a,x,T (R,T (R,b,y,c),z,d))
                | (B,a,x,T (R,b,y,T (R,c,z,d))) ) = T (R, T (B,a,x,b),T (B,c,z,d))
      | balance body = T body
  \end{lstlisting}
\end{remark}

После балансировки некоторого поддерева красный корень этого поддерева
может оказаться ребёнком ещё одного красного узла. Таким образом,
балансировка продолжается до самого корня дерева. На самом верху
дерева мы можем получить красную вершину с красным ребёнком, но без
чёрного родителя. С этим вариантом мы справляемся, всегда перекрашивая корень
в чёрное.

Реализация красно-чёрных деревьев полностью приведена на Рис.~\ref{fig:3.6}.

\begin{figure}
\begin{lstlisting}
functor RedBlackSet(Element: ORDERED) : SET =
  type Elem = Element.T
  datatype Color = R | B
  datatype Tree = E | T of Color $\times$ Tree $\times$ Elem $\times$ Tree
  type Set = Tree

  val empty = E
  fun member (x, E) = false
    | member (x, T (_, a, y, b) =
       if Element.lt (x,y) then member (x, a)
       else if Element.lt (y,x) then member (x, b)
       else true

  fun balance (B,T (R,T (R,a,x,b),y,c),z,d) = T (R, T (B,a,x,b),T (B,c,z,d))
    | balance (B,T (R,a,x,T (R,b,y,c)),z,d) = T (R, T (B,a,x,b),T (B,c,z,d))
    | balance (B,a,x,T (R,T (R,b,y,c),z,d)) = T (R, T (B,a,x,b),T (B,c,z,d))
    | balance (B,a,x,T (R,b,y,T (R,c,z,d))) = T (R, T (B,a,x,b),T (B,c,z,d))
    | balance body = T body

  fun insert (x, s) =
        let fun ins E = T (R, E, x, E)
              | ins (s as T (color, a, y, b)) =
                  if Element.lt (x,y) then balance (color, ins a, y, b)
                  else if Element.lt (y,x) then balance (color, a, y, ins b)
                  else s
            val T (_, a, y, b) = ins s  /* $\mbox{гарантированно непустое}$ */
        in T (B, a, y, b)
end
\end{lstlisting}
% TODO: опять курсив там где комменты


  \caption{Красно-чёрные деревья.}
  \label{fig:3.6}
\end{figure}

\begin{hint}
  Даже без дополнительных оптимизаций наша реализация сбалансированных
  двоичных деревьев поиска~--- одна из самых быстрых среди
  имеющихся. С оптимизациями вроде описанных в
  Упражнениях~\ref{ex:2.2} и \ref{ex:3.10} она просто летает!
\end{hint}

\begin{remark}
  Одна из причин, почему наша реализация выглядит настолько проще, чем
  типичное описание красно-чёрных деревьев (напр., Глава~14 в
  книге~\cite{CormenLeisersonRivest1990}), состоит в том, что мы
  используем несколько другие преобразования перебалансировки. В
  императивных реализациях обычно наши четыре проблематичных случая
  разбиваются на восемь, в зависимости от цвета узла, соседствующего с
  красной вершиной с красным ребёнком.  Знание цвета этого узла в
  некоторых случаях позволяет совершить меньше присваиваний, а в
  некоторых других завершить балансировку раньше. Однако в
  функциональной среде мы в любом случае копируем все эти вершины, и
  таким образом, не можем ни сократить число присваиваний, ни
  прекратить копирование раньше времени, так что для использования
  более сложных преобразований нет причины.
\end{remark}

\begin{exercise}\label{ex:3.9}
  Напишите функцию \lstinline!fromOrdList! типа \lstinline!Elem list $\to$ Tree!,
  преобразующую отсортированный список без повторений в красно-чёрное
  дерево. Функция должна выполняться за время $O(n)$.
\end{exercise}

\begin{exercise}\label{ex:3.10}
  Приведенная нами функция \lstinline!balance! производит несколько
  ненужных проверок. Например, когда функция \lstinline!ins!
  рекурсивно вызывается для левого ребёнка, не требуется проверять
  красно-красные нарушения на правом ребёнке.
  \begin{enumerate}
  \item Разбейте \lstinline!balance! на две функции
    \lstinline!lbalance! и \lstinline!rbalance!, которые проверяют,
    соответственно, нарушения инварианта в левом и правом
    ребёнке. Замените обращения к \lstinline!balance! внутри
    \lstinline!ins! на вызовы \lstinline!lbalance! либо \lstinline!rbalance!.
  \item Ту же самую логику можно распространить ещё на шаг и убрать
    одну из проверок для внуков. Перепишите \lstinline!ins! так, чтобы
    она никогда не проверяла цвет узлов, не находящихся на пути поиска.
  \end{enumerate}
\end{exercise}

\section{Примечания}
\label{sc:3.4}

Нуньес, Палао и Пенья \cite{NunezPalaoPena1995} и Кинг \cite{King1994}
описывают подобные нашим реализации, соответственно,
левоориентированных куч и биномиальных куч на Haskell.  Красно-чёрные
деревья до сих пор не были описаны в литературе по функциональному
программированию, в отличие от некоторых других вариантов
сбалансированных деревьев поиска, таких как AVL-деревья
\cite{Myers1982, Myers1984, BirdWadler1988, NunezPalaoPena1995},
2-3-деревья \cite{Reade1992} и деревья, сбалансированные по весу
\cite{Adams1993}.

Левоориентированные кучи были изобретены Кнутом \cite{Knuth1973a} как
упрощение структуры данных, введенной Крейном
\cite{Crane1972}. Вуаллемин \cite{Vuillemin1978} изобрел биномиальные
кучи; Браун \cite{Brown1978} исследовал многие свойства этой изящной
структуры данных. Гуибас и Седжвик \cite{GuibasSedgewick1978}
предложили красно-чёрные деревья в качестве обобщающего описания для
многих других разновидностей сбалансированных деревьев.

%%% Local Variables:
%%% mode: latex
%%% TeX-master: "pfds"
%%% End:

\chapter{Ленивое вычисление}
\label{ch:4}

Ленивое вычисление является основной стратегий вычисления во многих
функциональных языках программирования (но не в Стандартном ML). У
этой стратегии есть два существенных свойства. Во-первых, вычисление
всякого выражения задерживается, или \term{подвешивается}{suspend},
пока не потребуется его результат. Во-вторых, когда задержанное
выражение вычисляется в первый раз, результат вычисления запоминается
(\term{мемоизируется}{memoize}), так что, если он потребуется снова,
можно его просто извлечь из памяти, а не вычислять заново. Оба этих свойства
ленивого вычисления оказываются алгоритмически полезными.

В этой главе мы вводим удобные обозначения для ленивых вычислений и, в
качестве иллюстрации, строим при помощи этой нотации простую
библиотеку потоков. В последующих главах мы будем активно пользоваться
как ленивыми вычислениями вообще, так и потоками в частности.

\section{$\$$-запись}
\label{sc:4.1}

К сожалению, определение Стандартного ML \cite{Milner-etal1997} не
включает поддержки ленивого вычисления, так что каждая реализация
может предоставлять свой собственный набор элементарных операций.
Мы представляем здесь один такой набор,
называемый $\$$-записью.  Перевод программ, использующих
$\$$-запись, в другие варианты примитивов ленивого вычисления не
должен представлять трудности.

В $\$$-записи мы вводим новый тип \lstinline!$\alpha$ susp!,
представляющий задержки (задержанные вычисления). У этого типа имеется один
одноместный конструктор $\$$. В первом приближении 
\lstinline!$\alpha$ susp! и $\$$ ведут себя так, как будто они введены при помощи
обыкновенного объявления типа
\begin{lstlisting}
  datatype $\alpha$ susp = $\$$ of $\alpha$
\end{lstlisting}
Новая задержка типа \lstinline!$\tau$ susp! создается
при помощи конструкции \lstinline!$\$e$!, где $e$~---
выражение типа $\tau$. Подобным же образом, содержимое задержки можно
извлечь через сопоставление с образцом
$\$p$. Если образец $p$ сопоставляется со значениями типа $\tau$, то
$\$p$ сопоставляется с задержками типа
\lstinline!$\tau$ susp!.

Основное различие между $\$$ и обыкновенными конструкторами состоит в
том, что $\$$ не вычисляет свой аргумент немедленно.  Вместо этого он
запоминает информацию, необходимую для того, чтобы вычислить
выражение-аргумент позже. (Как правило, эта информация состоит из
указателя на код, а также значений свободных переменных выражения.)
Выражение-аргумент не вычисляется до тех пор, когда (и если) оно не
сопоставится с образцом вида $\$p$.  В этот момент выражение
вычисляется, а его результат запоминается. Затем результат
сопоставляется с образцом $p$. Если задержанное выражение потом
сопоставляется с другим образцом вида $\$p'$, запомненное значение
извлекается и сопоставляется с образцом $p'$.

Кроме того, конструктор $\$$ отличается от прочих конструкторов
синтаксически. Во-первых, его область действия распространяется
направо как можно дальше. Таким образом, например, выражение
\lstinline!$\$$f x! равнозначно \lstinline!$\$$(f x)!, а не 
\lstinline!($\$$f) x!; образец \lstinline!$\$$Cons (x, xs)! обозначает
то же, что \lstinline!$\$$(Cons (x, xs))!, а не 
\lstinline!($\$$ Cons) (x, xs)!. Во-вторых, $\$$ не является правильно
построенным выражением сам по себе~--- он всегда должен сочетаться с
аргументом.

В качестве примера $\$$--записи рассмотрим следующий фрагмент
программы: 
\begin{lstlisting}
  val s = $\$$primes 1000000	(* $\mbox{быстро}$ *)
  ...
  val $\$$x = s			(* $\mbox{медленно}$ *)
  ...
  val $\$$y = s			(* $\mbox{быстро}$ *)
  ...
\end{lstlisting}
Программа вычисляет миллионное простое число. Первая строка, которая
просто создает новую задержку, выполняется очень
быстро.  Вторая строка выполняет задержанное вычисление и
находит простое число. В зависимости от алгоритма
поиска простых чисел, она может потребовать значительного
времени.  Третья строка обращается к мемоизированному значению и также
выполняется очень быстро.

В качестве второго примера рассмотрим фрагмент
\begin{lstlisting}
  let val s = $\$$primes 1000000
  in 15 end
\end{lstlisting}
В этой программе содержимое задержки никогда не
требуется, и, значит, выражение \lstinline!primes 1000000! не
выполняется.

Хотя все примеры ленивого вычисления в этой книге можно было бы
выразить только через выражения и образцы со знаком $\$$, удобно
оказывается ввести два элемента синтаксического сахара. Первый из них~---
оператор \lstinline!force! (<<вынудить>>), определяемый как
\begin{lstlisting}
  fun force ($\$$x) = x
\end{lstlisting}
Он полезен, чтобы извлечь содержимое задержки посредине
выражения, где было бы неудобно вставлять конструкцию сопоставления с
образцом.

Второй элемент синтаксического сахара полезен при написании некоторых
разновидностей ленивых функций. Рассмотрим, например, следующую
функцию для сложения задержанных целых:
\begin{lstlisting}
  fun plus ($\$$m, $\$$n) = $\$$m+n
\end{lstlisting}
Несмотря на то, что определение функции выглядит совершенно разумно,
скорее всего, это не та функция, которую мы хотели написать. Проблема
состоит в том, что оба ее задержанных аргумента выполняются слишком
рано.  Они вынуждаются в момент применения функции
\lstinline!plus!, а не тогда, когда требуется выполнить задержку,
создаваемую ей.  Один из способов получить нужное
поведение~--- явным образом задержать сопоставление с образцом
\begin{lstlisting}
  fun plus (x, y) = $\$$case (x, y) of ($\$$m, $\$$n) $\Rightarrow$ m+n
\end{lstlisting}
Однако подобные конструкции встречаются достаточно часто, чтобы
имело смысл ввести для них синтаксический сахар
\begin{lstlisting}
  fun lazy f p = e
\end{lstlisting}
что равносильно
\begin{lstlisting}
  fun f x = $\$$case x of p $\Rightarrow$ force e
\end{lstlisting}
При помощи дополнительного \lstinline!force! мы добиваемся того, что
ключевое слово \lstinline!lazy! никак не влияет на тип функции (если
предположить, что он уже был \lstinline!$\alpha$ susp!), так что эту
аннотацию можно добавлять и убирать, никак не меняя остальной
текст. Теперь требуемую нам функцию для сложения задержанных целых
можно написать просто как
\begin{lstlisting}
  fun lazy plus ($\$$m, $\$$n) = $\$$m+n
\end{lstlisting}
Раскрытие синтаксического сахара дает
\begin{lstlisting}
  fun plus (x, y) = $\$$case (x, y) of ($\$$m, $\$$n) $\Rightarrow$ force ($\$$m+n)
\end{lstlisting}
что совпадает с ранее вручную написанной версией с
точностью до дополнительных \lstinline!force! и $\$$ вокруг
\lstinline!m+n!. Хороший компилятор уберет эти \lstinline!force! и
$\$$ при оптимизации, поскольку для любого $e$ выражения $e$ и 
\lstinline!force ($\$e$)! эквивалентны.

В функции \lstinline!plus! аннотация \lstinline!lazy! используется для
задержки сопоставления с образцом, чтобы $\$$-образцы не были
сопоставлены раньше времени. Однако аннотация \lstinline!lazy! полезна
также, когда правая сторона определения функции возвращает задержку
в результате вычисления, которое может оказаться долгим и 
сложным.  В такой ситуации использование \lstinline!lazy! сдвигает
выполнение дорогого вычисления от того момента, когда функция
применяется к аргументу, на тот, когда вынуждается возвращаемая ею
задержка. В следующем разделе мы увидим несколько
примеров такого использования \lstinline!lazy!.

Синтаксис и семантика $\$$-записи формально определены в
\cite{Okasaki1996a}.

\section{Потоки}
\label{sc:4.2}

В качестве расширенного примера ленивых вычислений и $\$$-записи в
Стандартном ML мы представляем простой пакет для работы с
потоками. Потоки будут использоваться в нескольких структурах данных
из последующих глав.
Потоки (известные также как ленивые списки) подобны обыкновенным
спискам, за исключением того, что каждая их ячейка задерживается. Тип
потоков выглядит так:
\begin{lstlisting}
  datatype $\alpha$ StreamCell = Nil | Cons of $\alpha$ $\times$ $\alpha$ Stream
  withtype $\alpha$ Stream = $\alpha$ StreamCell susp
\end{lstlisting}
Простой поток, содержащий элементы 1, 2 и 3, можно записать как
\begin{lstlisting}
  $\$$Cons (1, $\$$Cons (2, $\$$Cons (3, $\$$Nil)))
\end{lstlisting}

Полезно сравнить потоки с задержанными списками типа
\lstinline!$\alpha$ list susp!. Вычисления, представленные последними,
по существу {\em монолитны}~--- единожды начав вычислять задержанный
список, мы вычисляем его до конца. Напротив, вычисления,
представленные потоками, часто {\em пошаговы}~--- при обращении к
потоку проводится только та часть вычисления, которая порождает его
первый элемент, а остальное задерживается. Такое поведение часто
встречается в типах, которые, подобно потокам, содержат вложенные
задержки.

Чтобы яснее прочувствовать эту разницу в поведении, рассмотрим функцию
конкатенации, записываемую \lstinline!s $\concat$ t!. Для задержанных
списков ее можно записать как
\begin{lstlisting}
  fun s $\concat$ t = $\$$(force s @ force t)
\end{lstlisting}
что равносильно
\begin{lstlisting}
  fun lazy ($\$$xs) $\concat$ ($\$$ys) = $\$$(xs @ ys)
\end{lstlisting}
Задержка, порождаемая этой функцией, вынуждает оба аргумента, а затем
конкатенирует полученные списки и возвращает результат целиком. Таким
образом, задержка монолитна. Можно также сказать, что монолитна вся
функция. Для потоков функция записывается как
\begin{lstlisting}
  fun lazy ($\$$Nil) $\concat$ t = t
         | ($\$$Cons (x, s)) $\concat$ t = $\$$Cons (x, s $\concat$ t)
\end{lstlisting}
Эта функция немедленно возвращает задержку, которая, будучи запущена,
требует первую ячейку первого потока, сопоставляя ее с
$\$$-образцом. Если эта ячейка представляет собой \lstinline!Cons!, мы
строим результат из \lstinline!x! и \lstinline!s $\concat$ t!. 
Вследствие аннотации \lstinline!lazy! рекурсивный вызов просто
порождает ещё одну задержку, не производя никакой дополнительной
работы. Следовательно, эта функция описывает пошаговое вычисление:
порождается первая ячейка результата, а остальное задерживается. Мы
также говорим, что пошаговой является сама функция.

Ещё одна пошаговая функция~--- \lstinline!take!, извлекающая первые
$n$ элементов потока.
\begin{lstlisting}
  fun lazy take (0, s) = $\$$Nil
         | take (n, $\$$Nil) = $\$$Nil
         | take (n, $\$$Cons (x, s)) = $\$$Cons (x, take (n-1, s))
\end{lstlisting}
Как и в случае с $\concat$, рекурсивный вызов \lstinline!take!
немедленно возвращает задержку, а не выполняет оставшуюся часть кода
функции.

Рассмотрим, однако, функцию, уничтожающую первые $n$ элементов потока,
которую можно записать как
\begin{lstlisting}
  fun lazy drop (0, s) = s
         | drop (n, $\$$Nil) = $\$$Nil
         | drop (n, $\$$Cons (x, s)) = drop (n-1, s)
\end{lstlisting}
или, более эффективно, как
\begin{lstlisting}
  fun lazy drop (n, s) = let fun drop' (0, s) = s
                               | drop' (n, $\$$Nil) = $\$$Nil
                               | drop' (n-1, $\$$Cons (x, s)) = drop' (n-1, s)
                         in drop' (n, s) end
\end{lstlisting}
Эта функция монолитна, поскольку рекурсивные вызовы \lstinline!drop'!
никогда не задерживаются~--- вычисление первой же ячейки результата
требует выполнения всей функции целиком. Здесь аннотация
\lstinline!lazy! используется, чтобы задержать исходный вызов
\lstinline!drop'!, а не сопоставление с образцом.

\begin{exercise}\label{ex:4.1}
  Покажите, используя эквивалентность \lstinline!force ($\$e$)! и $e$,
  что два определения \lstinline!drop! эквивалентны. 
\end{exercise}

Ещё одна часто используемая монолитная функция над потоками~---
\lstinline!reverse!.
\begin{lstlisting}
  fun lazy reverse s =
        let fun reverse' ($\$$Nil, r) = r
              | reverse' ($\$$Cons (x, s), r) = reverse' (s, $\$$Cons (x, r))
        in reverse' (s, $\$$Nil) end
\end{lstlisting}
Здесь рекурсивные вызовы \lstinline!reverse'! никогда не
задерживаются. Обратите внимание, однако, что каждый такой вызов
создает задержку вида \lstinline!$\$$Cons (x, r)!. Может показаться,
что \lstinline!reverse! на самом деле не производит всю работу за один
раз. Однако задержки такого вида, где тело содержит лишь
несколько конструкторов и переменных, называются
\term{тривиальными}{trivial}. Тривиальные задержки создаются не из
каких-то алгоритмических соображений, а для того, чтобы удовлетворить
систему типов. Можно считать, что тело тривиальной задержки
выполняется в момент ее создания.  На самом деле, при минимальной
оптимизации компилятором подобные задержки создаются уже в
мемоизированном виде. В любом случае, вынуждение тривиальной
задержки никогда не занимает больше, чем $O(1)$ времени.

Несмотря на распространенность монолитных функций над потоками вроде
\lstinline!drop! и \lstinline!reverse!, смыслом существования потоков
являются пошаговые функции вроде $\concat$ и \lstinline!take!. Каждая
задержка несет с собой небольшие, но существенные расходы, поэтому для
максимальной эффективности ленивость следует использовать только тогда,
когда для этого есть серьезные основания. Если все операции над
ленивыми списками в каком-то приложении монолитны, то в этом
приложении лучше пользоваться обыкновенными ленивыми списками, а не
потоками.

На Рис.~\ref{fig:4.1} потоковые функции собраны в единый модуль на
Стандартном ML. Заметим, что в модуле не экспортируются, как можно
было бы ожидать,  функции вроде \lstinline!isEmpty! и
\lstinline!cons!. Вместо этого мы намеренно выставляем для обозрения
внутреннее представление, чтобы поддержать для потоков сопоставление с
образцом. 

\begin{figure}
  \centering
  
  (* конкатенация потоков *)
  
  \caption{Небольшой пакет потоков.}
  \label{fig:4.1}
\end{figure}

\begin{exercise}\label{ex:4.2}
  Реализуйте сортировку вставками для потоков. Покажите, что
  извлечение первых $k$ элементов \lstinline!sort xs! требует лишь 
  $O (n \cdot k)$ времени, где $n$~--- длина \lstinline!xs!, а не
  $O(n^2)$, как можно было бы ожидать от сортировки вставками.
\end{exercise}

\section{Примечания}
\label{sc:4.3}

\textbf{Ленивое вычисление.} Ленивое вычисление было изобретено
Уодсвортом \cite{Wadsworth1971} как оптимизация нормального порядка
редукции в лямбда-исчислении. Позже Вуаллемин \cite{Vuillemin1974}
показал, что при некоторым образом ограниченных условиях ленивое
вычисление является оптимальной стратегией вычисления. Формальная
семантика ленивого вычисления подробно исследовалась в
\cite{Josephs1989, Launchbury1993, OkasakiLeeTarditi1994, Ariola-etal1995}.

\noindent
\textbf{Потоки.} Потоки изобрел Ландин \cite{Landin1965}, но без
мемоизации. Фридман и Уайз \cite{FriedmanWise1976} и Хендерсон и
Моррис \cite{HendersonMorris1976} расширили потоки Ландина
мемоизацией.

\noindent
\textbf{Мемоизация.} Термин <<мемоизация>> придумал Мичи
\cite{Michie1968}, чтобы называть так кэширование пар
аргумент-результат у функции. Поле аргумента можно отбросить при мемоизации
задержек, если рассматривать задержки как нульместные функции, то
есть функции с нулем аргументов. Позднее Хьюз \cite{Hughes1985}
применил мемоизацию в исходном смысле Мичи к функциональным
программам.

\noindent
\textbf{Алгоритмика.} Обе компоненты ленивых вычислений~--- задержка
вычисления и мемоизация результатов,~--- имеют долгую историю в науке
построения алгоритмов, хотя и не всегда в сочетании друг с
другом. Идея задержки вычислений, которые могут оказаться дорогими
(часто это уничтожение элементов) с пользой используется в
хэш-таблицах \cite{vanWykVitter1986}, очередях с приоритетами
\cite{SleatorTarjan1986b, FredmanTarjan1987} и деревьях поиска
\cite{Driscoll-etal1989}. В свою очередь, мемоизация является основой
таких методик, как динамическое программирование \cite{Bellman1957} и
сжатие путей \cite{HopcroftUllman1973, TarjanvanLeeuwen1984}.

%%% Local Variables: 
%%% mode: latex
%%% TeX-master: "pfds"
%%% End: 

\chapter{Основы амортизации}
\label{ch:5}

За последние пятнадцать лет амортизация стала мощным инструментом в
построении и анализе структур данных. Реализации с амортизированными
характеристиками производительности часто оказываются проще и быстрее,
чем реализации со сравнимыми жёсткими характеристиками. В этой главе
мы даем обзор основных методов амортизации и иллюстрируем эти идеи
через простую реализацию очередей FIFO и несколько реализаций кучи.

К сожалению, простой подход к амортизации, рассматриваемый в этой
главе, конфликтует с идеей устойчивости~--- эти структуры, будучи
используемы как устойчивые, могут быть весьма неэффективны. Однако на
практике многие приложения устойчивости не требуют, и часто для таких
приложений реализации, представленные в этой главе, могут быть
замечательным выбором. В следующей главе мы увидим, как можно
совместить понятия амортизации и устойчивости при помощи ленивого
вычисления.

\section{Методы амортизированного анализа}
\label{sc:5.1}

Понятие амортизации возникает из следующего наблюдения.  Имея
последовательность операций, мы можем интересоваться временем, которое
отнимает вся эта последовательность, однако при этом нам может быть
безразлично время каждой отдельной операции. Например, имея $n$
операций, мы можем желать, чтобы время всей последовательности было
ограничено показателем $O(n)$, не настаивая, чтобы каждая из этих
операций происходила за время $O(1)$. Нас может устраивать, чтобы
некоторые из операций занимали $O(\log n)$ или даже $O(n)$, при
условии, что общая стоимость всей последовательности будет
$O(n)$. Такая дополнительная степень свободы открывает широкое
пространство возможностей при проектировании, и часто позволяет найти
более простые и быстрые решения, чем варианты с аналогичными жёсткими
ограничениями.

Чтобы доказать, что соблюдается амортизированное ограничение, нужно
определить амортизированную стоимость для каждой операции, и доказать,
что для любой последовательности операций общая амортизированная
стоимость является верхней границей общей реальной стоимости, т.~е.,
$$
\sum_{i=1}^m a_i \ge \sum_{i=1}^m t_i
$$
где $a_i$~--- амортизированная стоимость операции $i$, $t_i$~--- ее
реальная стоимость, а $m$~--- общее число операций. Обычно
доказывается несколько более сильный результат: что на любой
промежуточной стадии в последовательности операций общая текущая
амортизированная стоимость является верхней границей для общей текущей
реальной стоимости, т.~е.,
$$
\sum_{i=1}^j a_i \ge \sum_{i=1}^j t_i
$$
для любого $j$. Разница между общей текущей амортизированной стоимостью
и общей текущей реальной стоимостью называется
\term{текущие накопления}{accumulated savings}. Таким образом, общая
текущая амортизированная стоимость является верхней границей для
общей текущей реальной стоимости тогда и только тогда, когда текущие
накопления неотрицательны.

Амортизация позволяет некоторым операциям быть дороже, чем их
амортизированная стоимость. Такие операции называются
\term{дорогими}{expensive}. Операции, для которых амортизированная
стоимость превышает реальную, называются
\term{дешевыми}{cheap}. Дорогие операции уменьшают текущие накопления,
а дешевые их увеличивают. Главное при доказательстве
амортизированных характеристик стоимости~--- показать, что дорогие
операции случаются только тогда, когда текущих накоплений хватает,
чтобы покрыть их дополнительную стоимость.

Тарджан \cite{Tarjan1985} описывает два метода для анализа
амортизированных структур данных: \term{метод банкира}{banker's
  method} и \term{метод физика}{physicist's method}. В методе банкира
текущие накопления представляются как \term{кредит}{credits},
привязанный к определенным ячейкам структуры данных. Этот кредит
используется, чтобы расплатиться за будущие операции доступа к этим
ячейкам.  Амортизированная стоимость операции определяется как ее
реальная стоимость плюс размер кредита, выделяемого этой операцией,
минус размер кредита, который она расходует, т.~е.,
$$
a_i = t_i + c_i - \bar{c}_i
$$
где $c_i$~--- размер кредита, выделяемого операцией $i$, а $\bar{c}_i$~---
размер кредита, расходуемого операцией $i$. Каждая единица кредита
должна быть выделена, прежде чем израсходована, и нельзя расходовать
кредит дважды. Таким образом, $\sum c_i \ge \sum \bar{c}_i$, а
следовательно, как и требуется, $\sum a_i \ge \sum t_i$. Как правило,
доказательства с использованием метода банкира определяют
\term{инвариант кредита}{credit invariant}, регулирующий распределение
кредита так, чтобы при всякой дорогой операции достаточное его
количество было выделено в нужных ячейках структуры для покрытия
стоимости операции.

В методе физика определяется функция $\Phi$, отображающая всякий
объект $d$ на действительное число, называемое его
\term{потенциалом}{potential}.  Потенциал обычно выбирается так, чтобы
изначально равняться нулю и оставаться неотрицательным. В таком случае
потенциал представляет нижнюю границу текущих накоплений.

Пусть объект $d_i$ будет результатом операции $i$ и аргументом
операции $i+1$. Тогда амортизированная стоимость операции $i$
определяется как сумма реальной стоимости и изменения потенциалов между
$d_{i-1}$ и $d_i$, т.~е.,
$$
a_i = t_i + \Phi(d_i) - \Phi(d_{i-1})
$$
Текущая реальная стоимость последовательности операций равна
$$
\begin{array}{rcl}
\sum_{i=1}^j t_i & = & \sum_{i=0}^j (a_i + \Phi(d_{i-1}) - \Phi(d_i))\\
\\
  & = & \sum_{i=1}^j a_i + \sum_{i=1}^j (\Phi(d_{i-1}) - \Phi(d_i)) \\
\\
  & = & \sum_{i=1}^j a_i + \Phi(d_0) - \Phi(d_j)
\end{array}
$$
Суммы вроде $\sum_{i=1}^j (\Phi(d_{i-1}) + \Phi(d_i))$, где
чередующиеся отрицательные и положительные члены взаимно уничтожаются,
называются \term{телескопическими последовательностями}{telescoping
  series}. Если $\Phi$ выбран таким образом, что
$\Phi(d_0)$ равен нулю, а $\Phi(d_j)$ неотрицателен, мы имеем
$\Phi(d_j) \ge \Phi(d_0)$, так что, как и требуется, текущая общая
амортизированная стоимость является верхней границей для текущей общей
реальной стоимости.

\begin{remark}
  Такое описание метода физика несколько упрощает
  картину. Часто при анализе оказывается трудно втиснуть реальное
  положение дел в указанные рамки. Например, что делать с функциями,
  которые порождают или возвращают более одного объекта? Однако даже
  упрощенное описание достаточно для демонстрации основных идей.
\end{remark}

Ясно, что два метода анализа весьма похожи. Можно преобразовать метод
банкира в метод физика, если игнорировать распределение по ячейкам, и
считать, что потенциал равен общему количеству единиц кредита в
объекте, как указано в инварианте кредита. Подобным образом, можно
преобразовать метод физика в метод банкира, если расположить весь
кредит в корне объекта. Возможно, несколько удивляет то, что знание о
расположении ячеек не дает никакой дополнительной мощности в
доказательстве, но методы на самом деле эквивалентны \cite{Tarjan1985,
  Schoenmakers1992}. Чаще всего метод физика оказывается проще, но
иногда бывает удобно принять во внимание распределение по ячейкам.

Заметим, что кредит и потенциал являются лишь средствами анализа; ни
то, ни другое не присутствует в тексте программы (разве что, возможно,
в комментариях).

\section{Очереди}
\label{sc:5.2}

Мы демонстрируем методы банкира и физика через анализ простой
функциональной реализации FIFO-очередей, чья сигнатура приведена на
Рис.~\ref{fig:5.1}.

\begin{figure}

  \centering

  (* возбуждает исключение \lstinline!Empty!, если очередь пуста *)

  (* возбуждает исключение \lstinline!Empty!, если очередь пуста *)

  \caption{Сигнатура для очередей. (Этимологическое замечание:
    \lstinline!snoc! представляет собой перевернутое слово
    \lstinline!cons! и означает <<добавить справа>>.)}
  \label{fig:5.1}  
\end{figure}

Самая распространенная чисто функциональная реализация очередей
представляет собой пару списков, \lstinline!f! и \lstinline!r!, где
\lstinline!f! содержит головные элементы очереди в правильном порядке,
а \lstinline!r! состоит из хвостовых элементов в обратном порядке.
Например, очередь, содержащая целые числа 1\ldots 6, может быть
представлена списками \lstinline!f=[1,2,3]! и
\lstinline!r=[6,5,4]!. Это представление можно описать следующим
типом:
\begin{lstlisting}
  type $\alpha$ Queue = $\alpha$ list $\times$ $\alpha$ list
\end{lstlisting}
В этом представлении голова очереди~--- первый элемент \lstinline!f!,
так что функции \lstinline!head! и \lstinline!tail!
возвращают и отбрасывают этот элемент, соответственно.
\begin{lstlisting}
  fun head (x :: f, r) = x
  fun tail (x :: f, r) = f
\end{lstlisting}
Подобным образом, хвостом очереди является первый элемент
\lstinline!r!, так что \lstinline!snoc! добавляет к \lstinline!r!
новый элемент.
\begin{lstlisting}
  fun snoc ((f,r), x) = (f, x :: r)
\end{lstlisting}
Элементы добавляются к \lstinline!r! и убираются из \lstinline!f!, так
что они должны как-то переезжать из одного списка в другой. Этот
переезд осуществляется путем обращения \lstinline!r! и установки его
на место \lstinline!f! всякий раз, когда в противном случае
\lstinline!f! оказался бы пустым. Одновременно \lstinline!r!
устанавливается в \lstinline![]!. Наша цель~--- поддерживать
инвариант, что список \lstinline!f! может быть пустым только в том
случае, когда список \lstinline!r! также пуст (т.~е., пуста вся
очередь). Заметим, что если бы \lstinline!f! был пустым при непустом
\lstinline!r!, то первый элемент очереди находился бы в конце
\lstinline!r!, и доступ к нему занимал бы $O(n)$ времени. Поддерживая
инвариант, мы гарантируем, что функция \lstinline!head! всегда может
найти голову очереди за $O(1)$ времени.

Теперь \lstinline!snoc! и \lstinline!tail! должны распознавать
ситуацию, которая может привести к нарушению инварианта, и
соответствующим образом менять свое поведение.
\begin{lstlisting}
  fun snoc (([], _), x) = ([x], [])
    | snoc ((f,r), x) = (f,  x :: r)
  fun tail ([x], r) = (rev r, [])
    | tail (x :: f, r) = (f, r)
\end{lstlisting}
Заметим, что в первой ветке \lstinline!snoc! используется
образец-заглушка. В этом случае поле \lstinline!r! проверять не нужно,
поскольку из инварианта мы знаем, что если список \lstinline!f! равен
\lstinline![]!, то \lstinline!r! также пуст.

Чуть более изящный способ записать эти функции~--- вынести те части
\lstinline!snoc! и \lstinline!tail!, которые поддерживают инвариант, в
отдельную функцию \lstinline!checkf!. Она заменяет \lstinline!f! на
\lstinline!rev r!, если \lstinline!f! пуст, а в противном случае
ничего не делает.
\begin{lstlisting}
  fun checkf ([], r) = (rev r, [])
    | checkf q = q

  fun snoc ((f,r), x) = checkf (f, x :: r)
  fun tail (x :: f, r) = checkf (f, r)
\end{lstlisting}
Полный код реализации показан на Рис.~\ref{fig:5.2}. Функции
\lstinline!snoc! и \lstinline!head! всегда завершаются за время
$O(1)$, но \lstinline!tail! в худшем случае отнимает $O(n)$
времени. Однако, используя либо метод банкира, либо метод физика, мы
можем показать, что как \lstinline!snoc!, так и \lstinline!tail!
занимают амортизированное время $O(1)$.

\begin{figure}
  \centering
  
  \caption{Распространенная реализация чисто функциональной очереди.}
  \label{fig:5.2}
\end{figure}

В методе банкира мы поддерживаем инвариант, что каждый элемент в
хвостовом списке связан с одной единицей кредита. Каждый вызов
\lstinline!snoc! для непустой очереди занимает один реальный шаг и
выделяет одну единицу кредита для элемента хвостового списка; таким
образом, общая амортизированная стоимость равна двум. Вызов
\lstinline!tail!, не обращающий хвостовой список, занимает один шаг,
не выделяет и не тратит никакого кредита, и, таким образом, имеет
амортизированную стоимость 1. Наконец, вызов \lstinline!tail!,
обращающий хвостовой список, занимает $m+1$ реальный шаг, где $m$~---
длина хвостового списка, и тратит $m$ единиц кредита, содержащиеся в
этом списке, так что амортизированная стоимость получается $m + 1 - m
= 1$.

В методе физика мы определяем функцию потенциала $\Phi$ как длину
хвостового списка. Тогда всякий \lstinline!snoc! к непустой очереди
занимает один реальный шаг и увеличивает потенциал на единицу, так что
амортизированная стоимость равна двум. Вызов \lstinline!tail! без
обращения хвостовой очереди занимает один реальный шаг и не изменяет
потенциал, так что амортизированная стоимость равна одному. Наконец,
вызов \lstinline!tail! с обращением очереди занимает $m+1$ реальный
шаг, но при этом устанавливает хвостовой список равным \lstinline![]!,
уменьшая таким образом потенциал на $m$, так что амортизированная
стоимость равна $m + 1 - m = 1$.

В этом простом примере доказательства почти одинаковы. Но даже при
этом метод физика оказывается чуть проще по следующей причине.
Используя метод банкира, мы должны сначала выбрать инвариант кредита,
а затем для каждой функции решить, когда она должна выделять или
расходовать кредит. Инвариант кредита подсказывает нам, как это
сделать, но решение все же не принимается автоматически. Например,
должен ли \lstinline!snoc! выделить одну единицу кредита и израсходовать
ноль, или выделить две и одну израсходовать? Общий результат
оказывается один и тот же, так что дополнительная свобода оказывается
лишь дополнительным возможным источником путаницы. С другой стороны, в
методе физика от нас требуется принять только одно решение~--- выбрать
функцию потенциала. После этого анализ сводится к простым вычислениям;
никакой свободы выбора не остается.

\begin{hint}
  Эта реализация очередей идеальна в приложениях, где не требуется
  устойчивости и где приемлемы амортизированные показатели
  производительности.
\end{hint}

\begin{exercise}\label{ex:5.1}
  \textbf{Хогерворд \cite{Hoogerwoord1992}.}  Идея этих очередей легко
  может быть расширена на абстракцию \term{двусторонней очереди}{double-ended
    queue}, или \term{дека}{deque}, где чтение и запись разрешены с
  обоих концов очереди (см. Рис.~\ref{fig:5.3}). Инвариант делается
  симметричным относительно \lstinline!f! и \lstinline!r!: если
  очередь содержит более одного элемента, оба списка должны быть
  непустыми. Когда один из списков становится пустым, мы делим другой
  пополам и одну из половин обращаем.

  \begin{enumerate}
  \item Реализуйте эту версию деков.
  \item Докажите, что каждая операция занимает $O(1)$ амортизированного
    времени, используя функцию потенциала $\Phi(f,r) = abs(|f| -
    |r|)$, где $abs$~--- функция модуля.
  \end{enumerate}
\end{exercise}

\begin{figure}
  \centering

  (* вставка, просмотр и уничтожение головного элемента *)\\
  (* возбуждает исключение \lstinline!Empty!, если очередь пуста *)\\
  (* возбуждает исключение \lstinline!Empty!, если очередь пуста *)\\

  (* вставка, просмотр и уничтожение хвостового элемента *)\\
  (* возбуждает исключение \lstinline!Empty!, если очередь пуста *)\\
  (* возбуждает исключение \lstinline!Empty!, если очередь пуста *)\\

  \caption{Сигнатура двусторонней очереди.}
  \label{fig:5.3}
\end{figure}

\section{Биномиальные кучи}
\label{sc:5.3}

В Разделе~\ref{sc:3.2} мы показали, что вставка в биномиальную кучу
проходит в худшем случае за время $O(\log n)$. Здесь мы доказываем,
что на самом деле амортизированное ограничение на время вставки
составляет $O(1)$.

Мы пользуемся методом физика. Определим потенциал биномиальной кучи
как число деревьев в ней.  Заметим, что это число равно количеству
единиц в двоичном представлении $n$, числа элементов в куче.  Вызов
\lstinline!insert! занимает $k+1$ шаг, где $k$~--- число обращений к
\lstinline!link!. Если изначально в куче было $t$ деревьев, то после
вставки окажется $t - k + 1$ деревьев. Таким образом, изменение
потенциала составляет $(t - k + 1) - t = 1 - k$, а амортизированная
стоимость вставки $(k + 1) - (1 - k) = 2$.

\begin{exercise}\label{ex:5.2}
  Повторите доказательство с использованием метода банкира.
\end{exercise}

Для полноты картины нам нужно показать, что амортизированная стоимость
операций \lstinline!merge! и \lstinline!deleteMin! по-прежнему
составляет $O(\log n)$. \lstinline!deleteMin! не доставляет здесь
никаких трудностей, но в случае \lstinline!merge! требуется небольшое
расширение метода физика. До сих пор мы определяли амортизированную
стоимость операции как
$$
a = t + \Phi(d_{\mbox{\textit{вых}}}) - \Phi(d_{\mbox{\textit{вх}}})
$$
где $d_{\mbox{\textit{вх}}}$~--- структура на входе операции, а $d_{\mbox{\textit{вых}}}$~---
структура на выходе. Однако если операция принимает либо возвращает
более одного объекта, это определение требуется обобщить до
$$
a = t + \sum_{d \in \mbox{\textit{Вых}}} \Phi(d) - \sum_{d \in \mbox{\textit{Вх}}} \Phi(d)
$$
где $\mbox{\textit{Вх}}$~--- множество входов, а $\mbox{\textit{Вых}}$~--- множество выходов. В этом
правиле мы рассматриваем только входы и выходы анализируемого типа.

\begin{exercise}\label{ex:5.3}
  Докажите, что амортизированная стоимость операций \lstinline!merge!
  и \lstinline!deleteMin! по-прежнему составляет $O(\log n)$.
\end{exercise}

\section{Расширяющиеся кучи}
\label{sc:5.4}

\term{Расширяющиеся деревья}{splay trees} \cite{SleatorTarjan85}~--- возможно, самая известная
и успешно применяемая амортизированная структура данных. Расширяющиеся
деревья являются ближайшими родственниками двоичных сбалансированных
деревьев поиска, но они не хранят никакую информацию о балансе
явно. Вместо этого каждая операция перестраивает дерево при помощи
некоторых простых преобразований, которые имеют тенденцию увеличивать
сбалансированность. Несмотря на то, что каждая конкретная операция
может занимать до $O(n)$ времени, амортизированная стоимость ее, как
мы покажем, не превышает $O(\log n)$.

Важное различие между расширяющимися деревьями и сбалансированными
двоичными деревьями поиска вроде красно-чёрных деревьев из
Раздела~\ref{sc:3.3} состоит в том, что расширяющиеся деревья
перестраиваются даже во время запросов (таких, как \lstinline!member!),
а не только во время обновлений (таких, как \lstinline!insert!). Это
свойство мешает использованию расширяющихся деревьев для реализации
абстракций вроде множеств или конечных отображений в чисто
функциональном окружении, поскольку приходилось бы возвращать в
запросе новое дерево наряду с ответом на запрос\footnote{%
В Стандартном ML можно было бы хранить корень расширяющегося дерева в
ссылочной ячейке и обновлять значение по ссылке при каждом запросе, но
такое решение не является чисто функциональным.
}.
Однако в некоторых абстракциях операции-запросы достаточно ограничены,
чтобы эту проблему можно было обойти. Хорошим примером служит
абстракция кучи, поскольку здесь единственным интересным запросом
является \lstinline!findMin!. Как мы увидим, расширяющиеся деревья дают
нам отличную реализацию кучи.

Представление расширяющихся деревьев идентично представлению
несбалансированных двоичных деревьев поиска.
\begin{lstlisting}
  datatype Tree = E | T of Tree $\times$ Elem.T $\times$ Tree
\end{lstlisting}
Однако в отличие от несбалансированных двоичных деревьев поиска из
Раздела~\ref{sc:2.2}, мы позволяем дереву содержать повторяющиеся
элементы. Эта разница не является фундаментальным различием расширяющихся
деревьев и несбалансированных двоичных деревьев поиска; она просто
отражает отличие абстракции множества от абстракции кучи.

Рассмотрим следующую стратегию реализации для \lstinline!insert!:
разобьем существующее дерево на два поддерева, чтобы одно содержало все
элементы, меньше или равные новому, а второе все элементы, большие
нового. Затем породим новый узел из нового элемента и двух этих
поддеревьев. В отличие от вставки в обыкновенное двоичное дерево
поиска, эта процедура добавляет элемент как корень дерева, а не как
новый лист. Код для \lstinline!insert! выглядит просто как
\begin{lstlisting}
  fun insert (x, t) = T (smaller (x, t), x, bigger (x, t))
\end{lstlisting}
где \lstinline!smaller! выделяет дерево из элементов, меньше или равных
\lstinline!x!, а \lstinline!bigger! дерево из элементов, больших
\lstinline!x!. По аналогии с фазой разделения быстрой сортировки,
назовем новый элемент \term{границей}{pivot}.

Можно наивно реализовать \lstinline!bigger! как
\begin{lstlisting}
  fun bigger (pivot, E) = E
    | bigger (pivot, T (a, x, b)) =
        if x $\le$ pivot then bigger (pivot, b)
        else T (bigger (pivot, a), x, b)
\end{lstlisting}
однако при таком решении не делается никакой попытки перестроить
дерево, добиваясь лучшего баланса.  Вместо этого мы применяем простую
эвристику для перестройки: каждый раз, пройдя по двум левым ветвям
подряд, мы проворачиваем два пройденных узла.
\begin{lstlisting}
  fun bigger (pivot, E) = E
    | bigger (pivot, T a, x, b)) =
        if x $\le$ pivot then bigger (pivot, b)
        else case a of
               E $\Rightarrow$ T (E, x, b)
             | T (a$_1$, y, a$_2$) $\Rightarrow$
                  if y $\le$ pivot then T (bigger (pivot, a$_2$), x, b)
                  else T (bigger (pivot. a$_1$), y, T (a$_2$, x, b))
\end{lstlisting}
На Рис.~\ref{fig:5.4} показано, как \lstinline!bigger! действует на
сильно несбалансированное дерево. Несмотря на то, что результат
по-прежнему не является сбалансированным в обычном смысле, новое
дерево намного сбалансированнее исходного; глубина каждого узла
уменьшилась примерно наполовину, от $d$ до $\lfloor d/2 \rfloor$ или
$\lfloor d/2 \rfloor + 1$. Разумеется, мы не всегда можем уполовинить
глубину каждого узла в дереве, но мы можем уполовинить глубину каждого
узла, лежащего на пути поиска. В сущности, в этом и состоит принцип
расширяющихся деревьев: нужно перестраивать путь поиска так, чтобы
глубина каждого лежащего на пути узла уменьшалась примерно вполовину.

\begin{figure}
  \centering
  
  \caption{Вызов функции \lstinline!bigger! с граничным элементом 0.}
  \label{fig:5.4}
\end{figure}

\begin{exercise}\label{ex:5.4}
  Реализуйте операцию \lstinline!smaller!. Не забудьте, что
  \lstinline!smaller! должна сохранять элементы, равные границе (однако
   устраивать отдельную проверку на равенство не следует!).
\end{exercise}

Заметим, что \lstinline!smaller! и \lstinline!bigger! вегда проходят
по одному и тому же пути поиска. Вместо того, чтобы повторять это
прохождение дважды, можно соединить \lstinline!smaller! и
\lstinline!bigger! в единую функцию с названием \lstinline!partition!,
которая вернет оба результата в виде пары.  Написание этой функции не
представляет труда, но несколько утомительно.
\begin{lstlisting}
  fun partition (pivot, E) = (E, E)
    | partition (pivot, t as T (a, x, b)) =
       if x $\le$ pivot then
         case b of
           E $\Rightarrow$ (t, E)
         | T (b$_1$, y, b$_2$) $\Rightarrow$
              if y $\le$ pivot then
                  let val (small, big) = partition (pivot, b$_2$)
                  in (T (T (a, x, b$_1$), y, small), big) end
              else
                  let val (small, big) = partition (pivot, b$_1$)
                  in (T (a, x, small), T (big, y, b$_2$)) end
       else
         case a of
           E $\Rightarrow$ (E, t)
         | T (a$_1$, y, a$_2$) $\Rightarrow$
              if y $\le$ pivot then
                  let val (small, big) = partition (pivot, a$_2$)
                  in (T (a$_1$, y, small), T (big, x, b)) end
              else
                  let val (small, big) = partition (pivot, a$_1$)
                  in (small, T (big, y, T (a$_2$, x, b))) end         
\end{lstlisting}

\begin{remark}
  Эта функция не является точным эквивалентом \lstinline!smaller! и
  \lstinline!bigger! из-за расхождения фаз: \lstinline!partition!
  всегда обрабатывает узлы парами, а \lstinline!smaller! и
  \lstinline!bigger! иногда проходят по одному узлу.  Поэтому иногда
  \lstinline!smaller! и \lstinline!bigger! оборачивают не те же самые
  узлы, что \lstinline!partition!. Однако ни к каким важным
  последствиям это расхождение не приводит.
\end{remark}

Рассмотрим теперь \lstinline!findMin! и
\lstinline!deleteMin!. Минимальный элемент расширяющегося дерева
хранится в самой левой его вершине типа \lstinline!T!. Найти эту
вершину несложно.
\begin{lstlisting}
  fun findMin (T (E, x, b)) = x
    | findMin (T (a, x, b)) = findMin a
\end{lstlisting}
Функция \lstinline!deleteMin! должна уничтожить самый левый узел и
одновременно перестроить дерево таким же образом, как это делает
\lstinline!bigger!. Поскольку мы всегда рассматриваем только левую
ветвь, сравнения не нужны.
\begin{lstlisting}
  fun deleteMin (T (E, x, b)) = b
    | deleteMin (T (T (E, x, b), y, c)) = T (b, y, c)
    | deleteMin (T (T (a, x, b), y, c)) = T (deleteMin a, x, T (b, y, c))
\end{lstlisting}
На Рис.~\ref{fig:5.5} реализация расширяющихся
деревьев приведена целиком. Для полноты мы включили в нее функцию слияния
\lstinline!merge!, хотя она довольно неэффективна и для многих входов
занимает $O(n)$ времени.

\begin{figure}
  \centering
  
  \caption{Реализация кучи через расширяющиеся деревья.}
  \label{fig:5.5}
\end{figure}

Теперь мы хотим показать, что \lstinline!insert! выполняется за время
$O(\log n)$. Пусть $\#t$ обозначает размер дерева $t$ плюс
один. Заметим, что если $t = \lstinline!T($a$, $x$, $b$)!$, то $\#t =
\#a + \#b$. Пусть потенциал вершины $\phi(t)$ равен $\log(\# t)$, а
потенциал всего дерева равен сумме потенциалов его вершин. Нам
требуется следующее элементарное утверждение, касающееся логарифмов:
\begin{lemma}\label{lm:5.1}
  Для всех положительных $x, y, z$, таких, что $y + z \le x$, 
  $$
  1 + \log y + \log z < 2 \log x
  $$

  \noindent
  \textit{Доказательство.} Без потери общности предположим, что $y \le  z$.
  Тогда $y \le x/2$ и $z \le x$, так что $1 + \log y \le \log x$ и
  $\log z < \log x$
\end{lemma}

Пусть $\mathcal{T}(t)$ обозначает реальную стоимость вызова
\lstinline!partition! для дерева $t$, что определяется как число
рекурсивных вызовов \lstinline!partition!. Пусть $\mathcal{A}(t)$~---
амортизированная стоимость такого вызова, определяемая как
$$
\mathcal{A}(t) = \mathcal{T}(t) + \Phi(a) + \Phi(b) - \Phi(t)
$$
где $a$ и $b$~--- возвращаемые функцией \lstinline!partition!
поддеревья.

\begin{theorem}\label{th:5.2}
  $\mathcal{A}(t) \le 1 + 2\phi(t) = 1 + 2\log(\#t)$

  \noindent\textit{Доказательство.} Требуется рассмотреть два
  нетривиальных случая, называемые зиг-зиг и зиг-заг, в зависимости
  от того, проходит ли вызов \lstinline!partition! по двум левым
  ветвям (или, симметрично, по двум правым), либо по левой ветке, а
  затем правой (или, симметрично, по правой, а затем по левой).

  Для случая зиг-зиг предположим, что исходное и результирующее дерево
  имеют формы
$$
\begin{xy}
  \xymatrix@C=0.5em@R=2ex{
    & s & = & x \ar@{-}[dl]\ar@{-}[dr] & & & & & & & & y \ar@{-}[dl]\ar@{-}[dr] & = & s'\\
    t & = & y \ar@{-}[dl]\ar@{-}[dr] & & d & & \Rightarrow & & a & || & b & & x \ar@{-}[dl]\ar@{-}[dr] & = & t' \\
    & u & & c & & & & & & & & c &  & d \\
  }
\end{xy}
$$
где $a$ и $b$ являются результатами вызова \lstinline!partition (pivot, u)!. Тогда
$$
\begin{array}{ll}
  & \mathcal{A}(s) \\
= & \qquad\{\mbox{ по определению $\mathcal{A}$ }\} \\
  & \mathcal{T}(s) + \Phi(a) + \Phi(s') - \Phi(s) \\
= & \qquad\{\mbox{ $\mathcal{T}(s) = 1 + \mathcal{T}(u)$ }\} \\
  & 1 + \mathcal{T}(u) + \Phi(a) + \Phi(s') - \Phi(s) \\
= & \qquad\{\mbox{ $\mathcal{T}(u) = \mathcal{A}(u) - \Phi(a) - \Phi(b) + \Phi(u)$ }\} \\
  & 1 + \mathcal{A}(u) - \Phi(a) - \Phi(b) + \Phi(u) + \Phi(a) + \Phi(s') - \Phi(s) \\
= & \qquad\{\mbox{ раскрываем $\Phi(s)$ и $\Phi(s')$, упрощаем }\} \\
  & 1 + \mathcal{A}(u) + \phi(s') + \phi(t') - \phi(s) - \phi(t) \\
\le & \qquad\{\mbox{ по предположению индукции, $\mathcal{A}(u) \le 1 + 2\phi(u)$ } \} \\
  & 2 + 2\phi(u) + \phi(s') + \phi(t') - \phi(s) - \phi(t) \\
< & \qquad \{\mbox{$\phi(u) < \phi(t)$, а $\phi(s') \le \phi(s)$}\} \\
  & 2 + \phi(u) + \phi(t') \\
< & \qquad \{\mbox{ $\#u + \#t' < \#s$, а также Лемма~\ref{lm:5.1} }\} \\
  & 1 + 2\phi(s) \\
\end{array}
$$
Доказательство случая зиг-заг мы оставляем как упражнение для читателя.

\begin{exercise}\label{ex:5.5}
  Докажите случай зиг-заг.
\end{exercise}

Дополнительная стоимость операции \lstinline!insert! по сравнению с
\lstinline!partition! составляет один реальный шаг плюс разница
потенциалов между двумя поддеревьями-результатами
\lstinline!partition! и деревом-окончательным результатом
\lstinline!insert!. Это изменение потенциала равно просто $\phi$ от
нового корня. Поскольку амортизированная стоимость
\lstinline!partition! ограничена $1 + 2\log(\#t)$, амортизированная
стоимость \lstinline!insert! ограничена
$2 + 2\log(\#t) + \log(\#t + 1) \approx 2 + 3\log(\#t)$.

\end{theorem}

\begin{exercise}\label{ex:5.6}
  Докажите, что стоимость \lstinline!deleteMin! также составляет
  $O(\log n)$.
\end{exercise}

Какова ситуация с \lstinline!findMin!? Если дерево сильно
несбалансированно, \lstinline!findMin! может занять до $O(n)$
времени. Причем поскольку \lstinline!findMin! не проводит никакой
перестройки и, следовательно, никак не изменяет потенциал,
амортизировать эту стоимость негде! Однако раз время
\lstinline!findMin! пропорционально времени \lstinline!deleteMin!,
мы можем увеличить стоимость, взимаемую за \lstinline!deleteMin!,
вдвое, и один раз на каждый ее вызов бесплатно звать
\lstinline!findMin!. Этого достаточно для тех приложений, которые
всегда зовут \lstinline!findMin! и \lstinline!deleteMin!
вместе. Однако в некоторых приложениях \lstinline!findMin! может
вызываться по несколько раз на каждый вызов \lstinline!deleteMin!. Для
этих приложений мы не будем напрямую вызывать функтор
\lstinline!SplayHeap!, а будем его использовать в комбинации с
функтором \lstinline!ExplicitMin! из
Упражнения~\ref{ex:3.7}. Напомним, что задачей функтора
\lstinline!ExplicitMin! было обеспечить выполнение \lstinline!findMin!
за время $O(1)$. Функции \lstinline!insert! и \lstinline!deleteMin!
по-прежнему будут выполняться за время $O(\log n)$.

\begin{hint}
  Расширяющиеся деревья, дополняемые при необходимости функтором
  \lstinline!ExplicitMin!,~--- самая быстрая из известных реализаций
  кучи для большинства приложений, не требующих устойчивости данных и
  не вызывающих функцию \lstinline!merge!.
\end{hint}

Особенно приятным свойством расширяющихся деревьев является то, что
они естественным образом подстраиваются под любой порядок,
присутствующий во входных данных. Например, при использовании
расширяющихся деревьев для сортировки уже сортированного заранее
списка тратится всего $O(n)$ времени, а не $O(n \log n)$
\cite{MoffatEddyPetersson1996}. Тем же свойством обладают
левоориентированные кучи, но только для уменьшающихся
последовательностей. Расширяющиеся кучи отлично себя ведут как на
растущих, так и на уменьшающихся последовательностях, а также на
последовательностях, отсортированных лишь частично.

\begin{exercise}\label{ex:5.7}
  Напишите функцию сортировки, которая складывает элементы в
  расширяющееся дерево, а затем обходит его по порядку, выводя
  элементы в список. Покажите, что на уже отсортированном списке она
  работает за время всего $O(n)$.
\end{exercise}

\section{Парные кучи}
\label{sc:5.5}

\term{Парные кучи}{pairing heaps} \cite{Fredmaneta1986}~--- одна из тех структур, которые
сводят специалистов с ума. С одной стороны, их легко реализовать и они
весьма хорошо показали себя на практике. С другой стороны, провести их
полный анализ не удается уже более 10 лет!

Парные кучи представляют собой упорядоченные по принципу кучи деревья
с переменной степенью ветвления; их можно определить следующим типом
данных:
\begin{lstlisting}
  datatype Heap = E | T of Elem.T $\times$ Heap list
\end{lstlisting}
Мы считаем правильными только такие деревья, где \lstinline!E! никогда
не встречается в качестве ребенка вершины \lstinline!T!.

Поскольку деревья упорядочены по принципу кучи, функция
\lstinline!findMin! тривиальна:
\begin{lstlisting}
  fun findMin (T (x, hs)) = x
\end{lstlisting}
Функции \lstinline!merge! и \lstinline!insert! ненамного
сложнее. \lstinline!merge! добавляет то дерево, чей корень больше, в
качестве первого ребенка того дерева, чей корень
меньше. \lstinline!insert! сначала создает новое дерево с одним
элементом, а затем зовет \lstinline!merge!.
\begin{lstlisting}
  fun merge (h, E) = h
    | merge (E, h) = h
    | merge (h$_1$ as T (x, hs$_1$), h$_2$ as T (y, hs$_2$)) =
       if Elem.leq (x, y) then T (x, h$_2$ :: hs$_1$) else T (y, h$_1$ :: hs$_2$)
  fun insert (x, h) = merge (T(x, []), h)
\end{lstlisting}
Парные деревья называются именно так благодаря операции
\lstinline!deleteMin!. Эта операция отбрасывает корень, а затем
сливает деревья в два прохода. Первый проход сливает деревья парами
слева направо (т.~е., первое дерево сливается со вторым, третье с
четвертым и т.~д.). При втором проходе получившиеся деревья сливаются
справа налево. Эти два прохода можно кратко выразить так:
\begin{lstlisting}
  fun mergePairs [] = E
    | mergePairs [h] = h
    | mergePairs (h$_1$ :: h$_2$ :: hs) = merge (merge (h$_1$, h$_2$), mergePairs hs)
\end{lstlisting}
После этого \lstinline!deleteMin! выглядит совсем просто:
\begin{lstlisting}
  fun deleteMin (T (x, hs)) = mergePairs hs
\end{lstlisting}
Полная реализация приведена на Рис.~\ref{fig:5.6}

\begin{figure}
  \centering
  
  \caption{Парные кучи.}
  \label{fig:5.6}
\end{figure}

Легко видеть, что \lstinline!findMin!, \lstinline!insert! и
\lstinline!merge! занимают каждая по $O(1)$ времени. Однако в худшем
случае \lstinline!deleteMin! может отнять до $O(n)$. По аналогии с
расширяющимися деревьями (см. Упражнение~\ref{ex:5.8}) мы можем
показать, что \lstinline!insert!, \lstinline!merge! и
\lstinline!deleteMin! каждая отнимает по $O(\log n)$ амортизированного
времени. Существует предположение, что \lstinline!insert! и
\lstinline!merge! на самом деле работают за амортизированное время
$O(1)$ \cite{Fredmaneta1986}, но его до сих пор никому не удалось ни
доказать, ни опровергнуть.

\begin{hint}
  В приложениях, где не требуется функция \lstinline!merge!, парные
  кучи работают почти так же быстро, как расширяющиеся кучи, а если
  \lstinline!merge! требуется, то они значительно быстрее.  Подобно
  расширяющимся кучам, их следует применять только в тех приложениях,
  где устойчивость не требуется.
\end{hint}

\begin{exercise}\label{ex:5.8}
  Часто проще оказывается работать с двоичными деревьями, чем с деревьями с
  произвольным ветвлением. К счастью, любое дерево с произвольным
  ветвлением легко представить в виде двоичного. Достаточно
  преобразовать каждый узел со списком детей в двоичный узел, где левый ребенок
  представляет самого левого ребенка исходного узла, а правый
  потомок представляет его сестринский узел непосредственно
  справа. Если отсутствуют либо левый узел, либо правый сосед, то
  соответствующий узел двоичного дерева оказывается пустым. (Заметим,
  что таким образом в двоичном представлении правый потомок корневого
  узла всегда оказывается пуст.) Применив такое преобразование к парной
  куче, мы получаем полуупорядоченные двоичные деревья, где элемент в
  каждом узле не больше любого элемента в своем левом дочернем
  поддереве. 
  \begin{enumerate}
  \item Напишите функцию \lstinline!toBinary!, преобразующую парные
    кучи из исходного представления в тип
    \begin{lstlisting}
      datatype BinTree = E' | T' of Elem.T $\times$ BinTree $\times$ BinTree
    \end{lstlisting}
  \item Заново реализуйте парные кучи, используя это новое представление.
  \item Модифицируйте анализ расширяющихся деревьев и докажите, что
    \lstinline!deleteMin! и \lstinline!merge! работают за
    амортизированное время $O(\log n)$ в этом новом представлении (а
    следовательно, и в старом тоже). Следует использовать ту же самую
    функцию потенциала, как и в расширяющихся деревьях.
  \end{enumerate}
\end{exercise}

\section{Плохая новость}
\label{sc:5.6}

Как мы могли убедиться, амортизированные структуры могут быть
чрезвычайно эффективны на практике. К сожалению, все рассуждения в
этой главе неявно предполагают, что анализируемые структуры данных
используются эфемерным образом (то есть, только одной нитью
последовательных операций). Что произойдет, если мы попытаемся с теми же
самыми структурами обращаться как с устойчивыми?

Рассмотрим очереди из Раздела~\ref{sc:5.2}. Пусть $q$ будет очередь,
получаемая вставкой $n$ элементов в изначально пустую очередь, так что
головной список $q$ содержит один элемент, а хвостовой $n - 1$
элементов. Теперь предположим, что мы считаем очередь устойчивой и $n$
раз удаляем первый элемент. Каждый из этих вызовов отнимет $n$
реальных шагов.  Общая реальная стоимость этой последовательности
операций, включая изначальное построение $q$, равна $n^2 + n$. Если бы
операции на самом деле отнимали только по $O(1)$ амортизированного
времени, общая реальная стоимость была бы всего $O(n)$. Таким образом,
ясно, что использование наших очередей как устойчивой структуры
нарушает установленные в Разделе~\ref{sc:5.2} амортизированные
ограничения стоимости $O(1)$. Где же ошибка в доказательствах?

В обоих случаях одно из основных предположений доказательства
оказывается нарушенным при рассмотрении структуры как устойчивой. В
методе банкира требуется, чтобы каждая единица кредита тратилась не
более одного раза, а метод физика требует, чтобы результат одной
операции служил аргументом следующей (или, в более общей формулировке,
чтобы всякий результат операции использовался как аргумент другой не
более одного раза).  Рассмотрим второе обращение к \lstinline!tail q!
в вышеописанном примере. Первое обращение тратит весь кредит,
накопленный в хвостовом списке $q$, и оказывается нечем оплатить
второй и последующие вызовы, так что метод банкира терпит
неудачу. Кроме того, второе обращение к \lstinline!tail q! повторно
использует $q$, а не результат первого вызова, так что метод физика
тоже не работает.

Обе неудачных попытки доказательства отражают слабость всякой
системы подсчета, основанной на накоплениях~--- то, что эти накопления
можно потратить лишь один раз. Традиционные методы амортизации
работают путем накопления единиц работы (либо кредита, либо
потенциала) для дальнейшего использования. Это отлично работает при
эфемерном использовании, когда у каждой операции лишь одно логическое
будущее. Но у операции над устойчивой структурой может быть сколько угодно
логических будущих, и в каждом из них структура может пытаться потратить
одни и те же накопления.

В следующей главе мы разъясним, что имеется в виду под <<логическим
будущим>> операции, и как можно совместить амортизацию и устойчивость
через ленивое вычисление.

\begin{exercise}\label{ex:5.9}
  Приведите примеры последовательности операций, где биномиальные
  кучи, расширяющиеся кучи и парные кучи отнимают намного больше
  времени, чем указывают амортизированные границы их стоимости.
\end{exercise}

\section{Примечания}

Методы амортизации, обсуждаемые в этой главе, были разработаны
Слитором и Тарджаном \cite{SleatorTarjan1985, SleatorTarjan1986b}. Они
стали популярны благодаря Тарджану \cite{Tarjan1985}. Схунмакерс
\cite{Schoenmakers1992} показывает, как систематическим образом
получать амортизированные оценки стоимости при функциональном
программировании без использования устойчивости.

Кучи из Раздела~\ref{sc:5.2} были предложены Грисом
\cite[с.~250-251]{Gries1981}, а также Худом и Мелвиллом
\cite{HoodMelville1982}. Бёртон \cite{Burton1982} предложил похожую
реализацию, однако без ограничения, чтобы у непустой кучи головной список всегда был
непуст. У Бёртона \lstinline!head! и \lstinline!tail! объединены в
одну функцию, и, таким образом, нет требования, чтобы \lstinline!head!
по отдельности была эффективна.

В нескольких экспериментальных исследованиях было показано, что
расширяющиеся кучи \cite{Jones1986} и парные кучи
\cite{MoretShapiro1991,Liao1992}~--- одни из самых быстрых
реализаций для этой абстракции. Стаско и Виттер
\cite{StaskoVitter1987} подтвердили для варианта парных куч
предполагаемое амортизированное ограничение $O(1)$ на вставку.

%%% Local Variables: 
%%% mode: latex
%%% TeX-master: "pfds"
%%% End: 

\chapter{Сочетание амортизации и устойчивости через ленивое
  вычисление}
\label{ch:6}

В предыдущей главе мы представили понятие амортизации и привели
несколько примеров структур данных с хорошими амортизированными
показателями производительности. Однако все эти показатели для всех этих
структур перестают быть применимы, если их использовать как
устойчивые. В этой главе мы покажем, как ленивое вычисление может
разрешить конфликт между амортизацией и устойчивостью, и модифицируем
методы банкира и физика, чтобы они были применимы в условиях ленивого
вычисления. Затем мы демонстрируем применение наших методов к
нескольким амортизированным структурам данных, использующим ленивое
вычисление в своей реализации.

\section{Трассировка вычисления и логическое время}
\label{sc:6.1}

В предыдущей главе мы заметили, что традиционные методы амортизации
ломаются при наличии устойчивости, поскольку они предполагают наличие
у структуры единственного будущего, где накопленные сбережения будут
потрачены только один раз. Однако в устойчивой структуре несколько
будущих логических историй могут одновременно пытаться
использовать одни и те же сбережения. Однако что же мы имеем в виду,
говоря о <<логическом будущем>> операции?

Мы моделируем логическое время при помощи \term{трассировок
  вычисления}{execution traces}, которые представляют абстракцию
истории выполнения программы. Трассировка вычисления представляет собой
направленный граф, вершины которого соответствуют операциям, которые
нас интересуют; как правило, это только операции модификации над
рассматриваемым типом данных. Дуга от вершины $v$ к вершине $v'$
означает, что операция $v'$ использует результат операции
$v$. \term{Логической историей}{logical history} операции $v$
(обозначается $\hat{v}$) называется
множество всех операций, от которых зависит её результат (включая и
саму операцию $v$). Другими словами, $\hat{v}$~--- множество вершин
$w$, таких, что существует путь (возможно, длины 0) от $w$ до $v$. 
\term{Логическим будущим}{logical future} вершины $v$ называется любой
путь от $v$ до конечной вершины (т.~е., вершины с числом исходящих дуг
0). Если таких путей больше одного, значит, вершина $v$ имеет
несколько логических будущих. Иногда мы говорим о логической истории
или логическом будущем объекта, имея при этом в виду логическую
историю или будущее операции, создавшей этот объект.

\begin{exercise}\label{ex:6.1}
  Нарисуйте трассировку вычисления для следующей последовательности
  операций. Пометьте каждую вершину в графе количеством ее логических
  будущих.
  \begin{lstlisting}
    val a = snoc (empty, 0)
    val b = snoc (a, 1)
    val c = tail b
    val d = snoc (b, 2)
    val e = c $\concat$ d
    val f = tail c
    val g = snoc (d, 3)
  \end{lstlisting}
\end{exercise}
Понятие трассировки вычисления обобщает \term{графы версий}{version
  graphs} \cite{Driscoll-etal1989}, часто используемые для
моделирования историй устойчивых структур данных. В графе версий
вершины представляют различные версии единой устойчивой структуры, а
дуги соответствуют зависимостям между этими версиями.  Таким образом,
графы версий моделируют результаты операций, а трассировки
вычисления~--- операции сами по себе. Трассировки вычисления часто
оказываются удобнее, если надо совместить истории нескольких
устойчивых объектов (возможно, даже разных типов), а также для
рассуждений об операциях, не изменяющих версию объекта (например, о
запросах) либо возвращающих несколько результатов (скажем, разбивающих
список на два подсписка).

Для эфемерных структур данных, как правило, число исходящих дуг в
графе версий или в трассировке вычисления должно быть не более
единицы; это отражает ограничение, что каждая структура может
модифицироваться не более одного раза.  Для моделирования различных
вариантов устойчивости графы версий могут позволять числу исходящих
дуг вершины быть каким угодно, но вводить другие
ограничения. Например, часто требуют, чтобы графы версий были
деревьями (или лесами), говоря, что число входящих дуг для каждой
вершины не может превышать 1. Или же разрешается больше одной входящей
дуги у вершины, но запрещаются циклы, и таким образом, граф
оказывается направленным ациклическим графом. Мы никаких таких
ограничений для трассировок выполнения устойчивых структур данных не
накладываем. Вершины с числом входящих дуг более одной соответствуют
операциям, принимающим более одного аргумента, например, конкатенации
списков или объединению множеств. Циклы возникают для рекурсивно
определенных объектов, которые поддерживаются во многих ленивых
языках. Разрешено даже иметь несколько дуг между одними и теми же
вершинами, например, когда список конкатенируется сам с собой.

Трассировки вычисления будут использоваться в Разделе~\ref{sc:6.3.1},
где мы расширяем метод банкира для работы с устойчивыми структурами.

\section{Сочетание амортизации и устойчивости}
\label{sc:6.2}

В этом разделе мы показываем, как можно исправить методы банкира и
физика, заменив понятие текущих накоплений понятием текущего долга,
который представляет стоимость невыполненных ленивых
вычислений. Интуиция здесь состоит в том, что в то время, как
накопления можно тратить только один раз, нет никакого вреда в
многократном выплачивании долга.

\subsection{Роль ленивого вычисления}
\label{sc:6.2.1}

Напомним, что \term{дорогой}{expensive} называется операция, чья
реальная стоимость превышает ее (желательную) амортизированную
стоимость. Предположим, к примеру, что некоторый вызов функции
\lstinline!f x!
является дорогим. При наличии устойчивости вредоносный противник может
вызывать \lstinline!f x! сколь угодно часто. (Заметим, что каждый
такой вызов образует новое логическое будущее \lstinline!x!.) Если
каждая такая операция занимает одно и то же время, амортизированные
ограничения на время вычисления деградируют до наихудших
ограничений. Следовательно, нам надо добиться того, чтобы даже если
первое вычисление \lstinline!f x! окажется дорогим, последующие вызовы
таковыми не были.

При программировании без побочных эффектов такая цель недостижима ни
при вызове по значению (т.~е., при аппликативном порядке вычислений),
ни при вызове по имени (т.~е., при ленивом вычислении без мемоизации),
поскольку всякое применение функции \lstinline!f! к аргументу
\lstinline!x! занимает одно и то же время. Следовательно, амортизацию
невозможно выгодно совместить с устойчивостью в языках, поддерживающих
только эти два порядка вычисления.

Однако рассмотрим теперь вызов по необходимости (т.~е., ленивое
вычисление с мемоизацией). Если \lstinline!x! содержит задержанный
компонент, необходимый для вычисления \lstinline!f!, первое применение
\lstinline!f! к \lstinline!x! вынудит (возможно, дорогое) вычисление
этого компонента и запомнит результат. Последующие операции смогут
обращаться к результату напрямую. Ровно это нам и требовалось!

\begin{remark}
  Будучи однажды обнаруженной, связь ленивого вычисления и амортизации
  кажется естественной. Ленивое вычисление можно рассматривать как
  разновидность самомодификации, а самомодификация часто используется
  при амортизации \cite{SleatorTarjan1985, SleatorTarjan1986b}. Однако
  ленивое вычисление является особым образом ограниченной
  разновидностью самомодификации~--- не все виды самомодификации,
  используемые в амортизированных эфемерных структурах данных, могут
  быть выражены при помощи ленивого вычисления. В частности,
  расширяющиеся деревья, по-видимому, этому методу неподвластны.
\end{remark}
\subsection{Общая методика анализа ленивых структур данных}
\label{sc:6.2.2}

Как мы только что показали, ленивое вычисление необходимо для чисто
функциональной реализации амортизированных структур данных. Но
программы с ленивым вычислением знамениты тем, что анализ времени их
работы чрезвычайно сложен. Наиболее обычный способ анализа ленивых
программ состоит в том, чтобы притвориться, что они на самом деле
используют аппликативный порядок. Однако для анализа амортизированных
структур данных этот способ совершенно непригоден. Ниже мы описываем
базовую методику, позволяющую проводить такой анализ. В оставшейся
части главы мы с помощью этой методики модифицируем методы банкира и
физика. В результате мы получаем первые в истории методы анализа
устойчивых амортизированных структур данных и первые практически применимые
методы анализа нетривиальных ленивых программ.

Стоимость каждой операции мы разбиваем на несколько категорий. Во-первых,
\term{нераздельная}{unshared} стоимость операции~--- это время,
требуемое операции в предположении, что все задержки в системе уже
вынуждены и мемоизированы ко времени ее начала (т.~е., в
предположении, что \lstinline!force! всегда занимает время $O(1)$, за
исключением тех задержек, которые создаются и вынуждаются в процессе
выполнения самой операции). \term{Разделяемая}{shared} стоимость
операции~--- это время, требуемое для выполнения всех задержек,
создаваемых, но не вынуждаемых операцией (при тех же предположениях,
что и раньше). Наконец, \term{полная}{complete} стоимость операции
есть сумма её нераздельной и разделяемой стоимости. Заметим, что
полная стоимость операции равна ее стоимости, если бы ленивое
вычисление было заменено на аппликативное.

Кроме того, мы разбиваем разделяемую стоимость последовательности
операций на реализованную и
нереализованную. \term{Реализованная}{realized} стоимость есть
стоимость задержек, которые вынуждаются в процессе полного
вычисления. \term{Нереализованная}{unrealized} стоимость~--- стоимость
задержек, которые так и остаются невыполненными. \term{Общая
  реальная}{total actual} стоимость последовательности операций
равняется сумме общей нераздельной стоимости и реализованной
разделяемой стоимости~--- нереализованные вычисления не влияют на
общую стоимость. Заметим, что доля каждой операции в общей реальной
стоимости не меньше ее нераздельной стоимости и не больше ее полной
стоимости, в зависимости от того, какая доля разделяемой стоимости
реализуется.

Мы будем учитывать разделяемую стоимость с помощью понятия
\term{текущего долга}{accumulated debt}.  В начале вычисления долг
равен нулю, но каждый раз, когда создается задержка, он увеличивается
на разделяемую стоимость этой задержки (а также вложенных в неё
задержек). Впоследствии каждая операция выплачивает часть текущего
долга. \term{Амортизированная стоимость}{amortized cost} операции
равна сумме её нераздельной стоимости и количества выплаченного этой
операцией долга. Нам запрещается вынуждать задержку, прежде чем
полностью выплачен связанный с ней долг.

\begin{remark}
  Амортизированный анализ на основе понятия текущего долга во многом
  работает как \term{отложенная покупка}{layaway plan}. В случае
  отложенной покупки вы находите в магазине некоторый товар~---
  например, кольцо с бриллиантом,~--- который вы не можете позволить себе
  немедленно. Вы договариваетесь с магазином о цене и просите персонал
  отложить для вас кольцо. Затем вы производите регулярные платежи, и
  получаете кольцо только тогда, когда его цена полностью выплачена.

  При анализе ленивой структуры данных вы имеете вычисление, которое
  пока что не можете позволить себе немедленно. Вы создаете для этого
  вычисления задержку и присваиваете ей размер долга, пропорциональный
  её разделяемой стоимости. Затем вы выплачиваете долг небольшими
  порциями. Наконец, когда долг полностью выплачен, вам позволено
  произвести вычисление задержки.
\end{remark}

В жизненном цикле задержки есть три важных момента: когда она
создается, когда ее стоимость полностью оплачена, и когда она
выполняется. Мы обязаны доказать, что второй из этих моментов
предшествует третьему.  Если каждая задержка до своего вынуждения
полностью оплачена, то общее количество выплаченного долга является
верхней границей для реализованной разделяемой стоимости, а
следовательно, общая амортизированная стоимость (т.~е., общая
нераздельная стоимость плюс общее количество выплаченного долга)
является верхней границей для общей реальной стоимости (т.~е., общей
нераздельной стоимости плюс реализованная разделяемая стоимость). Мы
сделаем этот аргумент формальным в Разделе~\ref{sc:6.3.1}.

Одна из наиболее трудных проблем при анализе времени выполнения
ленивых программ~--- взаимодействие множественных логических
будущих. Мы избегаем этой проблемы, рассуждая о каждом из этих будущих
\emph{как если бы оно было единственным}. С точки зрения операции,
создающей задержку, каждое логическое будущее, эту задержку
вынуждающее, обязано само ее оплатить. Если два логических будущих
желают вынудить одну и ту же задержку, каждое из них платит за неё по
отдельности. Сговориться и выплатить долг по частям не
разрешается. Альтернативный взгляд на это ограничение состоит в том,
что задержку разрешается вынуждать \emph{только тогда, когда ее
  стоимость оплачена в рамках логической истории текущей операции}.
При использовании этого метода иногда мы будем выплачивать долг более
одного раза, и следовательно, переоценивать общее время, необходимое
для некоторых вычислений. Однако такая переоценка безвредна, и её цена
невелика по сравнению с простотой получаемого анализа.

\section{Метод банкира}
\label{sc:6.3}

Чтобы приспособить метод банкира к использованию понятия текущего
долга вместо текущих накоплений, мы заменяем кредит дебетом. Каждая
единица долга представляет определенное количество отложенной
работы. Когда мы вначале задерживаем какое-то вычисление, мы создаем
дебет, равный разделяемой стоимости этого вычисления, и распределяем
долг по узлам созданного объекта.  Выбор места, с которым связывается
каждая единица долга, зависит от природы вычисления. Если оно
\term{монолитно}{monolithic} (то есть, будучи однажды запущено,
сработает до завершения), обычно весь долг присваивается корневому
узлу результата. С другой стороны, если мы имеем дело с
\term{пошаговым}{incremental} вычислением (то есть, оно разбивается
на фрагменты, которые можно выполнить независимо друг от друга), то
долг может распределяться по корневым узлам частичных результатов.

Амортизированная стоимость операции равна ее нераздельной стоимости
плюс количество единиц долга, освобождаемых этой операцией. Обратите
внимание, что единицы долга, создаваемые операцией, в ее
амортизированную стоимость \emph{не включаются}. Порядок, в котором
высвобождаются единицы долга, зависит от того, как предполагается в
будущем обращаться к узлам объекта; долг на тех узлах, к которым мы
обратимся раньше, следует выплачивать в первую очередь. Чтобы
доказать ограничение амортизированной стоимости операции, нам нужно
показать, что всякий раз, как мы обращаемся к некоторой ячейке
(возможно, при этом вынуждая некоторую задержку), все единицы долга,
привязанные к этой ячейке, уже высвобождены (а следовательно,
задержанная операция полностью оплачена). Таким образом, мы
гарантируем, что общее количество долга, высвобожденное
последовательностью операций, является верхней границей реализованной
разделяемой стоимости этих операций. Следовательно, общая
амортизированная стоимость является верхней границей общей реальной
стоимости. Долг, сохраняющийся в конце вычислений, соответствует
нереализованной разделяемой стоимости, и не влияет на общую реальную
стоимость.

Пошаговые функции играют важную роль в методе банкира, поскольку они
позволяют распределить долг по различным ячейкам структуры данных,
каждая из которых соответствует вложенной задержке.  Впоследствии
доступ к каждой ячейке может быть открыт по мере высвобождения долга,
не ожидая выплаты долга по другим ячейкам.  На практике это означает,
что можно очень быстро оплатить начальные частичные результаты
пошагового вычисления, а последующие частичные результаты оплачиваются
по мере необходимости.  Однако монолитные функции дают намного меньшую
гибкость.  Программист вынужден предсказывать, когда понадобится
результат дорогого монолитного вычисления, и обустроить высвобождение
долга достаточно рано, чтобы ко времени, когда результат понадобится,
он был полностью выплачен.

\subsection{Обоснование метода банкира}
\label{sc:6.3.1}

В этом разделе мы доказываем утверждение, что общая амортизированная
стоимость является верхней границей общей реальной стоимости. Общая
амортизированная стоимость равна сумме общей нераздельной стоимости и
общего количества высвобожденного долга (включая повторные
высвобождения); общая реальная стоимость равна общей нераздельной
стоимости плюс реализованная разделяемая стоимость. Следовательно, нам
надо показать, что общее количество высвобожденного долга является
верхней границей для реализованной разделяемой стоимости.

Можно абстрактно рассматривать метод банкира как задачу разметки графа
трассировки вычисления из Раздела~\ref{sc:6.1}. Задача состоит в том,
чтобы каждую вершину графа пометить тремя (мульти)множествами $s(v)$,
$a(v)$ и $r(v)$, так что
$$
\begin{array}{cl}
  (I) & v \ne v' \Rightarrow s(v) \cap s(v') = \emptyset \\
  (II) & a(v) \subseteq \bigcup_{w \in \hat{v}} s(w) \\
  (III) & r(v) \subseteq \bigcup_{w \in \hat{v}} a(w) \\
\end{array}
$$
$s(v)$ является множеством, но $a(v)$ и $r(v)$ могут быть
мультимножествами (т.~е., могут содержать повторения). Условия II и
III не учитывают эти повторения.

$s(v)$~--- это множество дебетов, выделяемых операцией $v$. Условие I
утверждает, что никакая единица долга не выделяется более одного
раза. $a(v)$~--- множество дебетов, высвобождаемых операцией
$v$. Условие II требует, чтобы всякая высвобождаемая единица долга
была заранее выделена; точнее, операция может высвобождать только
единицы долга, которые были выделены в ее логической
истории. Наконец, $r(v)$~--- это единицы долга,
\term{реализуемые}{realized} операцией $v$, то есть, мультимножество
единиц долга, соответствующее задержкам, которые вынуждаются операцией
$v$. Условие III требует, чтобы всякая единица долга была
высвобождена, прежде чем она реализована, или, точнее, что нельзя
реализовать единицу долга, если она не высвобождена в логической
истории текущей операции.

Почему $a(v)$ и $r(v)$ являются не просто множествами, а
мультимножествами? Потому что одна и та же операция может высвободить
одни и те же единицы долга более одного раза или реализовать их более
одного раза (многократно вынудив одни и те же задержки). Несмотря на
то, что мы никогда не имеем намерения высвободить одни и те же единицы долга
многократно, при сочетании объекта с самим собой это может произойти.
Предположим, например, что в анализе функции конкатенации списков мы
высвобождаем какое-то количество единиц долга из первого аргумента и
какое-то из второго. Если мы будем строить конкатенацию списка с самим
собой, возможно, одни и те же единицы долга окажутся высвобожденными
дважды.

При таком абстрактном взгляде на метод банкира мы легко можем измерить
различные показатели стоимости вычисления. Пусть $V$ будет множество
всех вершин в трассировке вычисления. В таком случае общая разделяемая
стоимость равна $\sum_{v \in V}|s(v)|$, а общее количество
высвобожденного долга равно $\sum_{v \in V}|a(v)|$. Поскольку имеется
мемоизация, реализованная разделяемая стоимость равна не 
$\sum_{v \in V}|r(v)|$, а $\bigcup_{v \in V}|r(v)|$, где операция $\bigcup$
отбрасывает повторения. Таким образом, многократно вынужденная
задержка участвует в реальной стоимости только один раз. Согласно
Условию III, мы знаем, что 
$\bigcup_{v \in V}r(v) \subseteq \bigcup_{v \in V}
a(v)$. Следовательно,
$$
|\bigcup_{v \in V} r(v)| \le |\bigcup_{v \in V} a(v)| \le \sum_{v \in V} |a(v)|
$$
Таким образом, реализованная разделяемая стоимость ограничена сверху
общим количеством высвобожденного долга, а общая реальная стоимость
ограничена общей амортизированной стоимостью, что и требовалось
доказать.

\begin{remark}
  В этом рассуждении еще раз подчеркивается важность мемоизации. Без
  мемоизации (то есть, при вызове по имени вместо вызова по
  необходимости) общая реализованная стоимость была бы равна 
  $\sum_{v \in V} |r(v)|$, и не было бы никаких причин считать, что
  она меньше, чем $\sum_{v \in V} |a(v)|$.
\end{remark}

\subsection{Пример: очереди}
\label{sc:6.3.2}

В этом подразделе мы разрабатываем эффективную устойчивую реализацию
очередей и доказываем методом банкира, что все операции занимают амортизированное время
$O(1)$.

Как видно из обсуждения в предыдущем разделе, мы должны каким-то
образом внести в устройство нашей структуры данных ленивое вычисление,
так что мы заменяем пару списков из простых очередей
(Раздел~\ref{sc:5.2}) на пару потоков\footnote{Было
  бы достаточно заменить потоком только список-голову, однако для
  простоты мы заменяем оба списка.%
}.
Для упрощения дальнейшей работы мы также явно отслеживаем длину обоих
этих потоков.
\begin{lstlisting}
  type $\alpha$ Queue = int $\times$ $\alpha$ Stream $\times$ int  $\times$ $\alpha$ Stream
\end{lstlisting}
Первое целое число здесь~--- длина головного потока, а второе~---
длина хвостового потока. Заметим, что, в качестве приятного побочного
эффекта, благодаря явному хранению информации о длине мы тривиальным
образом можем
поддержать функцию \lstinline!size! (размер) с константным временем выполнения.

Если мы будем, прежде чем обратить хвостовой список, ждать, пока
головной опустеет, у нас не окажется достаточно времени, чтобы
заплатить за обращение. Вместо этого мы периодически
\term{проворачиваем}{rotate} очередь, заменяя \lstinline!f! на
\lstinline!f $\concat$ reverse r!, и делая хвостовой поток
пустым. Заметим, что это преобразование не меняет относительный порядок
элементов.

Когда следует проворачивать очередь? Напомним, что функция
\lstinline!reverse! монолитна. Следовательно, ее вычисление должно
быть спланировано достаточно сильно заранее, чтобы ко времени, когда
оно понадобится, весь ее долг был высвобожден. Вычисление
\lstinline!reverse! занимает $|r|$ шагов, так что, чтобы учесть её
цену, мы выделяем $|r|$ единиц долга (пока что игнорируя цену
операции $\concat$). Задержка \lstinline!reverse! может быть вынуждена
самое раннее после $|f|$ применений \lstinline!tail!, так что, если мы
провернем очередь в момент, когда $|r| \approx |f|$ и высвободим по
одной единице долга на каждое применение \lstinline!tail!, ко времени
исполнения \lstinline!reverse! долг будет выплачен.  Так что мы
проворачиваем очередь, когда $r$ становится на единицу длиннее $f$, и
поддерживаем инвариант $|f| > |r|$. Между прочим, это гарантирует нам,
что $f$ может быть пустым только при пустом $r$, как и в простых
очередях из Раздела~\ref{sc:5.2}. Основные функции работы с очередями
теперь записываются так:
\begin{lstlisting}
  fun snoc ((lenf, f, lenr, r), x) = check (lenf, f, lenr+1, $\$$Cons (x, r))
  fun head (lenf, $\$$Cons (x, f'), lenr, r) = x
  fun tail (lenf, $\$$Cons (x, f'), lenr, r) = check (lenf-1, f', lenr, r)
\end{lstlisting}
где вспомогательная функция \lstinline!check! обеспечивает $|f| \ge |r|$.
\begin{lstlisting}
  fun check (q as (lenf, f, lenr, r) =
        if lenr $\le$ lenf then q else (lenf+lenr, f $\concat$ reverse r, 0, $\$$Nil)
\end{lstlisting}
Полный программный код этой реализации приведен на Рис.~\ref{fig:6.1}
\begin{figure}
  \centering
  
  \caption{Амортизированные очереди с использованием метода банкира.}
  \label{fig:6.1}
\end{figure}

Чтобы лучше объяснить, как эта реализация эффективно справляется с
устойчивостью, рассмотрим следующий сценарий. Пусть имеется очередь
$q_0$, чьи головной и хвостовой потоки каждый имеют длину $m$, и пусть
$q_i = \lstinline!tail! q_{i-1}$ для $0 < i \le m+1$. Очередь
проворачивается при первом вызове \lstinline!tail!, а задержка
\lstinline!reverse!, созданная при повороте, вынуждается при последнем
вызове \lstinline!tail!. Обращение занимает $m$ шагов, и его стоимость
амортизируется по цепочке $q_1\ldots q_m$. (Пока что мы заботимся
только о стоимости \lstinline!reverse! и игнорируем стоимость
$\concat$.)

Выберем теперь какую-нибудь точку ветвления $k$ и повторим вычисление
от $q_k$ до $q_{m+1}$. (Заметим, что $q$ используется как устойчивая
структура.) Сделаем это $d$ раз. Как часто выполняется
\lstinline!reverse!? Зависит от того, находится ли точка ветвления $k$
до или после проворачивания очереди. Допустим, $k$ расположена после
поворота. Допустим даже, что $k = m$, так что каждая из повторяющихся
ветвей представляет собой простое взятие хвоста очереди. Каждая из
ветвей выполнения вынуждает задержку \lstinline!reverse!, но все они
вынуждают \emph{одну и ту же} задержку, так что функция
\lstinline!reverse! выполняется только один раз. Здесь ключевую роль
играет мемоизация~--- без нее вычисление \lstinline!reverse!
повторялось бы каждый раз, и общая стоимость составила бы $m(d+1)$
шагов, при том, что для ее амортизации у нас есть всего лишь $m + 1 +
d$ операций. При большом $d$ получилась бы амортизированная стоимость
операции $O(1)$, но мемоизация дает нам амортизированную стоимость
операции всего лишь $O(1)$.

Возможна, однако, и ситуация, когда вычисление \lstinline!reverse!
будет повторяться. Надо только взять $k = 0$ (т.~е., расположить точку
ветвления прямо перед проворотом очереди.) В этом случае первый вызов
\lstinline!tail! на каждой ветке выполнения повторяет проворот и
создает новую задержку \lstinline!reverse!. Эта новая задержка
вынуждается при последнем вызове \lstinline!tail! на каждой ветке, и
функция \lstinline!reverse! вычисляется заново. Поскольку все задержки
различны, мемоизация здесь не приносит никакой пользы. Полная
стоимость всех обращений равна \lstinline!m $\cdot$ d!, но теперь у нас
для амортизации этой стоимости есть $(m + 1)(d + 1)$ операций, и опять
амортизированная стоимость каждой операции получается $O(1)$. Главное,
что здесь следует заметить~--- это
 что мы повторяем работу только в том
случае, если повторяем и последовательность операций, по которым мы
распределяем амортизированную стоимость этой работы.

Это неформальное рассуждение показывает, что наши очереди требуют
амортизированное время $O(1)$ на каждую операцию, даже если они
используются как устойчивая структура. Мы переводим это рассуждение в
формальное доказательство с помощью метода банкира.

При простом просмотре кода легко убедиться, что неразделенная
стоимость каждой операции над очередью равна $O(1)$. Следовательно,
чтобы показать, что амортизированная стоимость каждой операции равна
$O(1)$, нужно доказать, что высвобождение $O(1)$ единиц долга на
каждой операции достаточно, чтобы выплатить стоимость каждой задержки
ко времени её вынуждения. Высвобождение долга происходит только на
операциях \lstinline!tail! и \lstinline!snoc!.

Пусть размер долга на $i$-ом узле головного списка будет $d(i)$, и
пусть $D(i) = \sum_{j=0}^i d(j)$~--- кумулятивный размер долга на
узлах от начала очереди до $i$ включительно. Мы будем соблюдать
следующий \emph{инвариант долга}:
$$
D(i) \le \min(2i, |f| - |r|)
$$
Подвыражение $2i$ гарантирует нам, что весь долг на первом узле
головного потока уже высвобожден (поскольку $d(0) = D(0) \le 2 \cdot 0
= 0$), так что этот узел можно спокойно вынуждать (операциями
\lstinline!head! или \lstinline!tail!). Подвыражение $|f| - |r|$
гарантирует, что весь долг во всей очереди высвобожден к моменту,
когда потоки имеют одинаковую длину, что бывает перед следующим
проворотом очереди.

\begin{theorem}\label{th:6.1}
  Операции \lstinline!snoc! и \lstinline!tail! сохраняют инвариант
  долга, высвобождая, соответственно, одну и две единицы.

  \noindent
  \textit{Доказательство.} Операция \lstinline!snoc!, не вызывающая
  проворот, просто добавляет один элемент к хвостовому потоку, и таким
  образом, увеличивает на один величину $|r|$ и уменьшает на один $|f|
  - |r|$. При этом инвариант оказывается нарушен в узлах, где до сих
  пор мы имели $D(i) = |f| - |r|$. Мы можем восстановить этот
  инвариант, высвободив первую единицу долга в очереди; при этом
  кумулятивный долг на всех последующих узлах уменьшится на единицу.
  Подобным образом, если операция \lstinline!tail! не вызывает
  поворота очереди, она просто отбрасывает элемент из головного
  потока. При этом $|f|$ уменьшается на единицу (а следовательно, и
  $|f| - |r|$), однако, что более важно, индекс $i$ всех остающихся
  узлов уменьшается на один, а следовательно, $2i$ уменьшается на
  два. Высвобождение первых двух единиц долга в очереди
  восстанавливает инвариант. Наконец, рассмотрим операцию
  \lstinline!snoc! или \lstinline!tail!, которая вызывает проворот
  очереди. Перед самым проворотом мы знаем, что весь долг в очереди
  высвобожден, так что после него единственные невысвобожденные
  единицы долга порождены самим этим проворотом. Если на момент
  проворота $|f| = m$ и $|r| = m+1$, то мы создаем $m$ единиц долга
  для конкатенации и $m+1$ для операции \lstinline!reverse!.
  Конкатенация~--- пошаговая операция, так что мы распределяем ее
  долг по единице на каждую из первых $m$ узлов. С другой стороны,
  функция \lstinline!reverse! монолитна, так что все ее $m+1$ единиц
  долга мы помещаем в узел $m$, первый узел обращенного потока. Таким
  образом, долг теперь распределен так, что
  $$
  \begin{array}{lcl}
    d(i) = \left\{
    \begin{array}[c]{ll}
      1 & \mbox{если $i < m$}\\
      m+1 & \mbox{если $i = m$}\\
      0 & \mbox{если $i > m$}\\
    \end{array}
    \right.
    & \mbox{и} &
    D(i) = \left\{
      \begin{array}[c]{ll}
        i+1 & \mbox{если $i < m$} \\
        2m+1 & \mbox{если $i \ge m$} \\
      \end{array}
    \right.
  \end{array}
  $$
  Это распределение нарушает инвариант в узлах 0 и $m$. Однако после
  высвобождения единицы долга в узле 0 инвариант в обоих этих узлах
  оказывается восстановлен.
\end{theorem}

Это рассуждение имеет стандартную структуру. Единицы долга
распределяются по нескольким узлам для пошаговых функций и
концентрируются в одном узле для монолитных. Инварианты долга
измеряют не только количество его единиц в каждой конкретной вершине,
но и количество его на пути от корневого узла к этой вершине. Это
отражает наблюдение, что для доступа к любому узлу мы должны сначала
пройти через всех его предков. Следовательно, к этому времени долг на
всех этих узлах тоже должен быть равен нулю.

Эта структура данных демонстрирует также тонкую деталь, касающуюся
вложенных задержек: долг для вложенной задержки может быть выделен и
даже высвобожден прежде, чем задержка физически создается. Рассмотрим,
например, как работает операция $\concat$. Задержка для второго узла
потока физически создается только при вынуждении первого. Однако из-за
мемоизации задержка для второго узла будет разделяться между ветвями
вычисления всегда, когда разделяется задержка для
первого. Следовательно, мы считаем, что неявно вложенная задержка
создается тогда, когда создается задержка, в которую она вложена.
Более того, когда мы рассуждаем о долге или вообще о структуре
объекта, нас не интересует, создан ли узел физически или нет. Мы
рассуждаем так, будто бы все узлы были созданы в своем
окончательном виде, т.~е., как будто все задержки в объекте уже были
вынуждены.

\begin{exercise}\label{ex:6.2}
  Допустим, мы изменим инвариант очереди с формулы $|f| > |r|$ на $2|f| >
  |r|$.
  \begin{enumerate}
  \item Докажите, что амортизированные ограничения $O(1)$ по-прежнему
    выполняются.
  \item Сравните производительность двух реализаций при
    последовательной вставке ста элементов через \lstinline!snoc!, а
    затем их последовательном удалении через \lstinline!tail!.
  \end{enumerate}
\end{exercise}

\subsection{Наследование долга}
\label{sc:6.3.3}

Часто мы создаем задержки, тела которых вынуждают другие, существующие
задержки. В таких случаях мы говорим, что новая задержка
\term{зависит}{depends} от старой. В примере с очередью задержка,
создаваемая операцией \lstinline!reverse r!, зависит от \lstinline!r!,
а задержка, создаваемая \lstinline!f $\concat$ reverse r! зависит от
\lstinline!f!.  Каждый раз, когда мы вынуждаем задержку, следует быть
уверенными, что высвобожден не только долг, относящийся к ней самой,
но и долг всех задержек, от которых она зависит. В примере с очередью
инвариант долга гарантирует, что мы создаем задержки через $\concat$ и
\lstinline!reverse! только в случаях, когда ранее существующие
задержки уже полностью оплачены. Однако так будет не всегда.

Когда мы создаем задержку, зависящую от существующей задержки с
невысвобожденным долгом, мы переносим этот долг на новую задержку и
говорим, что новая задержка \term{наследует}{inherits} долги
старой. Мы не имеем права вынуждать новую задержку, пока мы не
высвободили как ее собственные долги, так и унаследованные от старой
задержки. В методе банкира не делается никакого различия между этими
двумя разновидностями долга; считается, что весь долг принадлежит
новой задержке. Наследование долга будет применяться при анализе
структур данных из Глав~\ref{ch:9}, \ref{ch:10} и \ref{ch:11}.

\begin{remark}
  Наследование долга предполагает, что не существует способа доступа к
  более старой задержке внутри текущего объекта в обход новой. К
  примеру, наследование долга нельзя применить при анализе следующей
  функции, применяемой к паре потоков:
  \begin{lstlisting}
    fun reverseSnd (xs, ys) = (reverse ys, ys)
  \end{lstlisting}
  Здесь \lstinline!ys! может быть вынуждена как через первый компонент
  пары, так и через второй. В таких случаях мы либо удваиваем долг,
  связанный с \lstinline!ys!, и новая задержка наследует копию долга,
  либо сохраняем одну копию каждой единицы долга и отслеживаем
  зависимости явно.
\end{remark}

\section{Метод физика}
\label{sc:6.4}

Подобно методу банкира, метод физика тоже можно адаптировать для работы
с понятием текущего долга вместо текущих накоплений.  В традиционном
методе физика описывается функция потенциала $\Phi$, представляющая
нижнюю границу текущих накоплений. Чтобы работать с долгом вместо
накоплений, мы заменяем $\Phi$ на функцию $\Psi$, которая сопоставляет
объектам потенциалы, представляющие верхнюю границу текущего долга
(или, по крайней мере, долю общего долга, относящуюся к
рассматриваемому объекту).  Грубо говоря, после этого амортизированная
стоимость операции определяется как ее полная стоимость (т.~е.,
разделенная стоимость плюс нераздельная) минус изменение
потенциала. Напомним, что простейший способ рассчитать полную
стоимость операции~--- притвориться, что все вычисление работает
аппликативно. 

Всякое изменение текущего долга отражается в изменении потенциала.
Если операция не выплачивает никакую долю разделенной стоимости, то
изменение потенциала равно ее разделенной стоимости, и, следовательно,
амортизированная стоимость операции равна ее нераздельной стоимости. С
другой стороны, если операция выплачивает часть своей разделяемой
стоимости, или разделяемой стоимости предыдущих операций, тогда
изменение потенциала будет меньше, чем её разделяемая стоимость
(т.~е., текущий долг увеличивается меньше, чем на её разделяемую
стоимость), и амортизированная стоимость операции окажется больше, чем
нераздельная. Однако амортизированная стоимость операции не может быть
меньше, чем её нераздельная стоимость, так что мы не позволяем
потенциалу изменяться больше, чем на разделяемую стоимость операции.

Обосновать метод физика можно путём сведения его к методу
банкира. Напомним, что в методе банкира амортизированная стоимость
операции равна её нераздельной стоимости плюс размер
высвобождаемого долга. В методе физика амортизированная стоимость
равна полной стоимости минус изменение потенциала или, другими
словами, нераздельной стоимости плюс разница между разделяемой
стоимостью и изменением потенциала.  Если единицу потенциала мы
считаем равной единице долга, то разделяемая стоимость равна
количеству единиц, на которое мог бы увеличиться текущий долг, а
изменение потенциала равно количеству единиц, на которое текущий долг
увеличился на самом деле. Разница должна была быть покрыта путем
высвобождения части долга.  Следовательно, амортизированная стоимость
в методе физика также может рассматриваться как нераздельная стоимость
плюс количество высвобождаемых единиц долга.

Мы иногда хотим вынудить задержку в объекте, чей потенциал не равен
нулю. В таком случае мы добавляем потенциал этого объекта к
амортизированной стоимости операции. Как правило, такое случается в
операциях-запросах, поскольку там стоимость вынуждения задержки нельзя
отразить как изменение потенциала: такая операция не возвращает нового
объекта. 

Главное различие между методами банкира и физика состоит в том, что
при использовании метода банкира мы можем вынудить задержку, как только
её собственный долг выплачен, не ожидая выплаты долга по другим
задержкам, в то время как в методе физика разделяемая задержка может
быть вынуждена только после того, как весь текущий долг объекта,
измеряемый его потенциалом, обращен в ноль.  Поскольку потенциал
измеряет только накопившийся долг всего объекта и не делает различия
между его ячейками, нам приходится делать пессимистическое
предположение, что весь текущий долг привязан к той конкретной
задержке, которую мы сейчас хотим вынудить. Из-за этого метод физика
кажется менее мощным, чем метод банкира. Однако когда он применим, как
правило, метод физика значительно упрощает рассуждения.

Поскольку метод физика не может воспользоваться частичным выполнением
вложенных задержек, нет никаких причин предпочитать пошаговые функции
монолитным. В сущности, если все или большинство задержек монолитны,
это может служить подсказкой о применимости метода физика.

\subsection{Пример: биномиальные кучи}
\label{sc:6.4.1}

В Главе~\ref{ch:5} мы показали, что биномиальные кучи из
Раздела~\ref{sc:3.2} поддерживают операцию \lstinline!insert! за
амортизированное время $O(1)$. Однако если кучи используются как
устойчивая структура, этот показатель для худшего случая деградирует
до $O(\log n)$.  С помощью ленивого вычисления мы можем восстановить
амортизированное ограничение по времени $O(1)$ вне зависимости от
того, используются ли кучи как устойчивая структура.

Основная идея состоит в том, чтобы заменить в представлении кучи
список деревьев на задержанный список деревьев.
\begin{lstlisting}
  type Heap = Tree list susp
\end{lstlisting}
При этом мы можем переписать \lstinline!insert! в виде
\begin{lstlisting}
  fun lazy insert (x, $\$$ts) = $\$$insTree (Node (0, x, []), ts)
\end{lstlisting}
или, эквивалентным образом, в виде
\begin{lstlisting}
  fun insert (x, h) = $\$$insTree (Node (0, x, []), force h)
\end{lstlisting}
Остальные функции столь же просты в написании; они показаны на
Рис.~\ref{fig:6.2}.

\begin{figure}
  \centering
  
  \caption{Ленивые биномиальные кучи}
  \label{fig:6.2}
\end{figure}

Проанализируем амортизированное время работы
\lstinline!insert!. Поскольку это монолитная операция, мы можем
использовать метод физика. Сначала определяем функцию потенциала как 
$\Psi(h) = Z(|h|)$, где $Z(n)$~--- число нулей в двоичном
представлении $n$ (минимальной длины). Затем мы покажем, что
амортизированная стоимость вставки элемента в биномиальную кучу
размера $n$ равна двум.  Допустим, что $k$ младших разрядов в двоичном
представлении $n$ равны единице. Тогда полная стоимость операции
\lstinline!insert! пропорциональна $k + 1$, поскольку включает $k$
вызовов операции \lstinline!link!.  Рассмотрим теперь изменение
потенциала. Младшие $k$ разрядов изменяются с единиц на нули, а одна
следующая цифра изменяется с нуля на единицу, так что изменение
потенциала равно $k - 1$. Амортизированная стоимость получается 
$(k + 1)  - (k - 1) = 2$.

\begin{remark}
  Заметим, что наше доказательство двойственно по отношению к
  доказательству из Раздела~\ref{sc:5.3}. Тогда
  потенциал равнялся количеству единиц в двоичном представлении $n$,
  теперь это количество нулей. Такая зеркальность отражает двойственность
  между понятиями текущих накоплений и текущего долга.
\end{remark}

\begin{exercise}\label{ex:6.3}
  Докажите, что \lstinline!findMin!, \lstinline!deleteMin! и
  \lstinline!merge! также выполняются за амортизированное время
  $O(\log n)$.
\end{exercise}

\begin{exercise}\label{ex:6.4}
  Допустим, мы уберем ключевое слово \lstinline!lazy! из определений
  функций \lstinline!merge! и \lstinline!deleteMin!, так что эти
  функции будут вычислять свои аргументы немедленно. Покажите, что обе
  они по-прежнему выполняются за время $O(\log n)$.
\end{exercise}

\begin{exercise}\label{ex:6.5}
  Задержка списка деревьев имеет неприятное последствие: время работы
  \lstinline!isEmpty! деградирует от $O(1)$ в худшем случае до
  амортизированного $O(\log n)$. Восстановите время работы $O(1)$ для
  \lstinline!isEmpty! путем явного хранения размера каждой
  кучи.  Вместо того, чтобы явно модифицировать нашу теперешнюю
  реализацию, напишите функтор \lstinline!SizedHeap!, подобный
  \lstinline!ExplicitMin! из Упражнения~\ref{ex:3.7}. Он должен
  преобразовывать произвольную реализацию кучи в реализацию, которая
  явно хранит размер.
\end{exercise}

\subsection{Пример: очереди}
\label{sc:6.4.2}

В этом разделе мы приспосабливаем под метод физика нашу реализацию
очередей. Как и раньше, мы показываем, что все операции занимают
амортизированное время $O(1)$.

Поскольку теперь нет никакого смысла предпочитать пошаговые задержки
монолитным, вместо потоков мы используем задержанные списки. В
сущности, хвостовой список даже задерживать не надо, поэтому его мы
представляем как обыкновенный список.  Как и раньше, мы явно храним
длины списков и гарантируем, что головной список имеет длину не меньше
хвостового. 

Поскольку головной список задержан, мы не можем получить доступ к его
первому элементу, не выполнив всю задержку целиком.  Поэтому для
ответов на запросы \lstinline!head! мы держим рабочую копию некоторого
префикса головного списка. Ради эффективности доступа эта рабочая
копия хранится в виде обычного списка. Если головной список непуст,
эта копия также непуста. Итоговый тип выглядит так:
\begin{lstlisting}
  type $\alpha$ Queue = $\alpha$ list $\times$ int $\times$ $\alpha$ list $\times$ int $\times$ $\alpha$ list
\end{lstlisting}
Теперь мы можем записать основные функции над очередями:
\begin{lstlisting}
  fun snoc ((w, lenf, f, lenr, r), x) = check (w, lenf, f, lenr+1, x :: r)
  fun head (x :: w, lenf, f, lenr, r) = x
  fun tail (x :: w, lenf, f, lenr, r) = check (w, lenf-1, $\$$tl (force f), lenr, r)
\end{lstlisting}
Вспомогательная функция \lstinline!check! обеспечивает два инварианта:
что \lstinline!r! не может быть длиннее, чем \lstinline!f!, и что при
непустом \lstinline!f! не может быть пуст \lstinline!w!.
\begin{lstlisting}
  fun checkw ([], lenf, f, lenr, r) = (force f, lenf, f, lenr, r)
    | checkw q = q
  fun check (q as (w, lenf, f, lenr, r)) =
        if lenr $\le$ lenf then checkw q
        else let val f' = force f
             in checkw (f', lenf+lenr, $\$$(f' @ rev r), 0, []) end
\end{lstlisting}
Полная реализация очередей приведена на Рис.~\ref{fig:6.3}.

\begin{figure}
  \centering
  
  \caption{Амортизированные кучи с использованием метода физика}
  \label{fig:6.3}
\end{figure}

Для анализа очередей методом физика мы выбираем функцию потенциала
$\Psi$ так, чтобы при вынуждении задержанного списка потенциал всегда был
равен нулю. Такое может произойти в двух ситуациях: когда
\lstinline!w! оказывается пустым, и когда \lstinline!r! оказывается
длиннее \lstinline!f!. Поэтому мы выбираем такое $\Psi$:
$$
\Psi(\lstinline!q!) = \min(2|\lstinline!w!|, |\lstinline!f!| - |\lstinline!r!|)
$$
\begin{theorem}\label{th:6.2}
  Амортизированная стоимость операций \lstinline!snoc! и
  \lstinline!tail! равна, соответственно, двум и четырем.

  \emph{Доказательство.} \lstinline!snoc!, не вызывающий проворота,
  просто добавляет новый элемент к хвостовому списку. При этом
  $|\lstinline!r!|$ увеличивается на единицу, а $|\lstinline!f!| -
  |\lstinline!r!|$ уменьшается на единицу. Полная стоимость
  \lstinline!snoc! равна одному, а уменьшение потенциала не больше
  одного, так что амортизированная стоимость равна максимум $1 - (-1)
  = 2$. Вызов \lstinline!tail!, не приводящий к провороту очереди,
  убирает элемент из рабочего списка и лениво убирает тот же самый
  элемент из головного списка. При этом $|\lstinline!w!|$ уменьшается
  на единицу, и на столько же уменьшается $|\lstinline!f!| -
  |\lstinline!r!|$, так что потенциал уменьшается максимум на два.
  Полная стоимость \lstinline!tail! равна двум~--- один как
  нераздельная стоимость (включая отбрасывание первого элемента
  \lstinline!w!) и один как разделяемая стоимость ленивого
  отбрасывания головы \lstinline!f!. Амортизированная стоимость
  получается $2 - (-2) = 4$.

  Наконец, рассмотрим вызов \lstinline!snoc! или \lstinline!tail!,
  приводящий к провороту очереди. При начале операции $|\lstinline!f!|
  = |\lstinline!r!|$, так что $\Psi = 0$. Перед самым проворотом
  $|\lstinline!f!| = m$, а $|\lstinline!r!| = m+1$. Разделяемая
  стоимость проворота равна $2m+1$, а потенциал получающейся очереди
  $2m$. Таким образом, амортизированная стоимость \lstinline!snoc!
  равна $1 + (2m + 1) - 2m = 2$. Амортизированная стоимость
  \lstinline!tail!  равна $2 + (2m + 1) - 2m = 3$. (Разница получается
  потому, что в случае \lstinline!tail! нам нужно ещё учесть стоимость
  удаления первого элемента \lstinline!f!.)
\end{theorem}

\begin{exercise}\label{ex:6.6}
  Покажите, почему каждая из следующих <<оптимизаций>> уничтожает
  амортизированное ограничение времени $O(1)$. Эти примеры показывают
  типичные ошибки при проектировании устойчивых амортизированных
  структур данных.
  \begin{enumerate}
  \item Заметим, что \lstinline!check! при провороте вынуждает
    \lstinline!f!, а затем записывает результат в \lstinline!w!. Разве не
    было бы более ленивым поведением, а следовательно, более выгодным,
    не вынуждать \lstinline!f!, пока \lstinline!w! не окажется пустым?
  \item Заметим, что при операции \lstinline!tail! мы заменяем
    \lstinline!f! на \lstinline!$\$$tl (force f)!. Создание и
    вынуждение задержек приводит к заметным расходам, которые, хотя и
    сохраняют стоимость константной, могут сделать константу слишком
    большой. Разве не было бы ленивее, а следовательно, лучше, не изменять
    \lstinline!f!, а просто уменьшать \lstinline!lenf!, показывая
    таким образом, что элемент удален?
  \end{enumerate}
\end{exercise}

\subsection{Сортировка слиянием снизу вверх с совместным
  использованием}
\label{sc:6.4.3}

Большинство примеров в оставшихся главах использует метод банкира, а
не физика. Поэтому здесь мы приводим ещё один пример на метод физика.

Допустим, что вы хотите отсортировать несколько похожих списков,
например, \lstinline!x! и \lstinline!x :: xs!, или \lstinline!xs @ zs! и
\lstinline!ys @ zs!. Из соображений эффективности вам хотелось бы
использовать то, что хвосты списков совпадают, чтобы не повторять
работу по сортировке хвостов.  Назовем абстрактный тип данных для
решения этой задачи \term{сортируемая коллекция}{sortable
  collection}. Сигнатура сортируемых коллекций приведена на
Рис.~\ref{fig:6.4}.

\begin{figure}
  \centering
  
  \caption{Сигнатура сортируемых коллекций}
  \label{fig:6.4}
\end{figure}

Теперь если мы из списка \lstinline!xs! сделаем сортируемую коллекцию
\lstinline!xs'!, добавив к пустой коллекции все элементы
\lstinline!xs! по очереди, то сможем отсортировать \lstinline!xs! и
\lstinline!x :: xs!, вызвав, соответственно, \lstinline!sort xs'! и
\lstinline!sort (add (x, xs'))!.

Сортируемые коллекции можно реализовать как сбалансированные двоичные
деревья поиска. Тогда \lstinline!add! и \lstinline!sort! будут иметь,
соответственно, ограничения по времени в худшем случае $O(\log n)$ и
$O(n)$. Здесь мы достигаем тех же самых ограничений, но только
амортизированных, используя \term{сортировку слиянием снизу
  вверх}{bottom-up mergesort}.

Сортировка слиянием снизу вверх сначала разбивает список на $n$
упорядоченных сегментов (на первом этапе каждый сегмент содержит по
одному элементу). Затем она попарно сливает сегменты одинакового
размера, пока для каждого размера не останется только один. Наконец,
сливаются сегменты неодинакового размера.

Возьмем состояние данных непосредственно перед последним
шагом. Размеры сегментов в этот момент равны степеням двойки,
соответствующим единичным битам в $n$.  Именно это представление мы
будем использовать для наших сортируемых коллекций.  Похожие коллекции
будут совместно использовать работу сортировки снизу вверх с точностью
до последней фазы, когда сливаются сегменты разного размера.
Полностью данные будут представлены в виде задержанного списка
сегментов, каждый из которых является списком элементов, плюс целое
число~--- размер коллекции.
\begin{lstlisting}
  type Sortable = int $\times$ Elem.T list list susp
\end{lstlisting}
Отдельные сегменты хранятся в порядке возрастания размера, а элементы
каждого сегмента хранятся в порядке возрастания согласно функциям
сравнения структуры \lstinline!Elem!.

Основная операция над сегментами~--- слияние двух упорядоченных
списков, \lstinline{mrg}.
\begin{lstlisting}
  fun mrg ([], ys) = ys
    | mrg (xs, []) = xs
    | mrg (xs as x :: xs', ys as y :: ys') =
       if Elem.leq (x, y) then x :: mrg (xs', ys) else y :: mrg (xs, ys')
\end{lstlisting}
При добавлении нового элемента мы создаем одноэлементный сегмент. Если
наименьший из существующих сегментов тоже одноэлементен, мы эти два
сегмента сливаем, и продолжаем слияние до тех пор, пока новый сегмент
не окажется меньше наименьшего существующего. Это слияние управляется
битами в поле размера. Если младший бит \lstinline!size! равен нулю, то
мы просто прицепляем новый сегмент к списку сегментов. Если бит равен
единице, мы сливаем два сегмента и повторяем операцию. Разумеется, все
это происходит в ленивом режиме.
\begin{lstlisting}
  fun add (x, (size, segs)) =
       let fun addSeg (seg, segs, size) =
                if size mod 2 = 0 then seg :: segs
                else addSeg (mrg (seg, hd segs), tl segs, size div 2)
       in (size+1, $\$$addSeg([x], force segs, size)) end
\end{lstlisting}
Наконец, чтобы отсортировать коллекцию, мы сливаем сегменты от
меньшего к большему.
\begin{lstlisting}
  fun sort (size, segs) =
       let fun mrgAll (xs, []) = xs
             | mrgAll (xs, seg :: segs) = mrgAll (mrg (xs, seg), segs)
       in mrgAll ([], force segs) end
\end{lstlisting}

\begin{remark}
  Можно рассматривать \lstinline!mrgAll! как вычисление
  $$
  [] \bowtie s_1 \bowtie \ldots \bowtie s_m
  $$
  где $s_i$~--- $i$-й сегмент, а $\bowtie$~--- инфиксное
  лево-ассоциативное обозначение для операции \lstinline!mrg!. Это частный случай весьма
  распространенного программного шаблона, который можно записать как
  $$
  c \oplus x_1 \oplus \ldots \oplus x_m
  $$
  для любого $c$ и лево-ассоциативной $\oplus$. В качестве других
  примеров этого шаблона можно привести суммирование списка целых ($c
  = 0$ и $\oplus = +$) или нахождение максимума в списке натуральных
  чисел ($c = 0$ и $\oplus = \max$). Одна из самых сильных черт
  функциональных языков~--- способность определять шаблоны подобного
  рода в виде \term{функций высших порядков}{higher-order functions}
  (т.~е., функций, которые принимают другие функции в качестве
  аргументов или возвращают функции как результат). Например,
  вышеприведенный шаблон можно записать как
  \begin{lstlisting}
    fun foldl (f, c, []) = c
      | foldl (f, c, x :: xs) = foldl (f, f (c, x), xs)
  \end{lstlisting}
  Тогда \lstinline!sort! выглядит как
  \begin{lstlisting}
    fun sort (size, segs) = foldl (mrg, [], force segs)
  \end{lstlisting}
\end{remark}
Полный программный код к нашей реализации сортируемых коллекций
приведен на Рис.~\ref{fig:6.5}.

\begin{figure}
  \centering
  
  \caption{Сортируемые коллекции на основе сортировки слиянием снизу вверх}
  \label{fig:6.5}
\end{figure}

Покажем, используя метод физика, что операция \lstinline!add! занимает
амортизированное время $O(\log n)$, а операция \lstinline!sort!
амортизированное время $O(n)$.  Вначале зададим функцию потенциала
$\Psi$, которая полностью определяется размером коллекции.
$$
\Psi(n) = 2n - 2 \sum_{i=0}^{\infty} b_i (n \mod 2^i+1)
$$
где $b_i$~--- $i$-й бит $n$. Заметим, что $\Psi(n)$ ограничен сверху
величиной $2n$, и что $\Psi(n) = 0$ в точности тогда, когда $n = 2^k -
1$ для некоторого $k$.

\begin{remark}
  Наша функция потенциала может показаться немного сложноватой. Она
  возникает из желания считать, что каждый сегмент имеет потенциал,
  пропорциональный его собственному размеру минус размер всех более
  мелких сегментов. Интуиция здесь заключается в том, что у всякого
  сегмента потенциал сначала велик, но он уменьшается по мере
  добавления новых элементов в коллекцию, и обращается в ноль
  непосредственно перед тем, как наш сегмент сливается с другим
  сегментом. Однако для того, чтобы проводить вычисления с функцией,
  необязательно знать, какими соображениями мотивировано ее
  определение. 
\end{remark}

Сначала вычислим полную стоимость операции \lstinline!add!. Её
нераздельная стоимость равна единице, а разделяемая равна
стоимости слияний, проводимых внутри \lstinline!addSeg!. Допустим, что
младшие $k$ бит числа $n$ равны единице (т.~е., $b_i =1$ для $i < k$ и
$b_k = 0$). В этом случае \lstinline!addSeg! проводит $k$ слияний.
Первое из них сливает два списка длиной 1, второе два списка длиной 2,
и так далее. Поскольку слияние двух списков размера $m$ занимает $2m$
шагов, \lstinline!addSeg! занимает
$$
(1+1) + (2+2) + \cdots + (2^{k-1} + 2^{k-1}) = 2(\sum_{i=0}^{k-1} 2^i)
= 2 (2^k - 1)
$$
шагов. Следовательно, полная стоимость \lstinline!add! равна $2(2^k -
1) + 1 = 2^{k+1} - 1$.

Вычислим теперь изменение потенциала. Пусть $n' = n+1$, а $b'_i$~---
$i$-й бит числа $n'$. Тогда
$$
\begin{array}{l}
\Psi(n') - \Psi(n) \\
= 2n' - 2\sum_{i=0}^\infty b'_i (n \mod 2^i + 1) - (2n - 2\sum_{i=0}^\infty (n \mod 2^i + 1) \\
= 2 + 2\sum_{i=0}^\infty (b_i(n \mod 2^i + 1) - b'_i(n' \mod 2^i + 1)) \\
= 2 + 2\sum_{i=0}^\infty \delta(i)
\end{array}
$$
где $\delta(i) = b_i(n \mod 2^i + 1) - b'_i(n \mod 2^i +
1)$. Рассмотрим три случая: $i < k$, $i = k$ и $i > k$.
\begin{itemize}
\item $(i < k)$: поскольку $b_i = 1$, а $b'_i = 0$, $\delta(i) = n
  \mod 2^i + 1$. Но $n \mod 2^i = 2^i - 1$, так что $\delta(i) - 2^i$.
\item $(i = k)$: поскольку $b_k = 0$, а $b'_k = 1$, $\delta(k) = -(n'
  \mod 2^k + 1)$. Но $n' \mod 2^k = 0$, так что $\delta(k) = -1 = -b'_k$.
\item $(i > k)$: поскольку $b'_i = b_i$, $\delta(i) = b'_i (n \mod 2^i
  - n' \mod 2^i)$. Но $n' \mod 2^i = (n+1) \mod 2^i = n \mod 2^i + 1$,
  так что $\delta(i) = b'_i(-1) = -b'_i$. 
\end{itemize}
Следовательно,
$$
\begin{array}{lcl}
\Psi(n') - \Psi(n) & = & 2 + 2\sum_{i=0}^\infty \delta(i) \\
& = & 2 + 2\sum_{i=0}^{k-1} 2^i + 2 \sum_{i=k}^\infty (-b'_i) \\
& = & 2 + 2(2^k - 1) - 2\sum_{i=k}^\infty b'_i \\
& = & 2^{k+1} - 2B'
\end{array}
$$
где $B'$~--- число единичных битов в $n'$. Тогда амортизированная
стоимость операции \lstinline!add! равна
$$
(2^{k+1} - 1) - (2^{k+1} - 2B') = 2B' -1
$$
Поскольку $B'$ пропорционален $O(\log n)$, такую же оценку имеет и
амортизированная стоимость \lstinline!add!.

Наконец, вычисляем амортизированную стоимость операции
\lstinline!sort!. Первое её действие~--- вынудить задержанный список
сегментов.  Поскольку потенциал не обязательно равен нулю, это
добавляет $\Psi(n)$ к амортизированной стоимости операции. Затем
\lstinline!sort! сливает сегменты, двигаясь от меньших к большим. В
худшем случае $n = 2^k -1$, так что есть по сегменту каждого размера
от 1 до $2^{k-1}$. Слияние сегментов занимает в общей сложности
$$
\begin{array}{l}
(1+2) + (1+2+4) + (1+2+4+8) + \cdots + (1 + 2 + \cdots + 2^{k-1}) \\
= \sum_{i=1}^{k-1}\sum_{j=0}^i 2^j = \sum_{i=1}^{k-1}(2^{i+1} - 1)
= (2^{k+1} - 4) - (k - 1) = 2n - k - 1
\end{array}
$$
шагов. Следовательно, амортизированная стоимость равна 
$O(n) + \Psi(n) = O(n)$.

\begin{exercise}\label{ex:6.7}
Замените в нашей реализации задержанный список списков на список
потоков.
\begin{enumerate}
\item докажите ограничения стоимости для \lstinline!add! и
  \lstinline!sort! с помощью метода банкира.
\item Напишите функцию для извлечения наименьших $k$ элементов из
  сортируемой коллекции. Докажите, что ваша функция работает за
  амортизированное время не хуже $O(k \log n)$.
\end{enumerate}
\end{exercise}

\section{Ленивые парные кучи}
\label{sc:6.5}

В завершение этой главы мы модифицируем парные кучи из
Раздела~\ref{sc:5.5} для работы в условиях устойчивости. К сожалению,
анализ получающейся структуры данных оказывается столь же сложен, как
и для исходной. Однако мы предполагаем, что асимптотически наша новая
реализация столь же эффективна в условиях устойчивости, как исходная
реализация эффективна в эфемерных условиях.

Напомним, что в предыдущей реализации парных куч дети каждого узла
представлялись как список структур \lstinline!Heap!. При уничтожении
минимального элемента корень отбрасывался, а затем дети сливались
попарно при помощи функции
\begin{lstlisting}
  fun mergePairs [] = E
    | mergePairs [h] = h
    | mergePairs (h$_1$ :: h$_2$ :: hs) = merge (merge (h$_1$, h$_2$), mergePairs hs)
\end{lstlisting}
Если уничтожить корневой элемент одной и той же кучи дважды,
\lstinline!mergePairs!  будет также вызвана дважды. При этом работа
будет повторяться, а всякая надежда на эффективное амортизированное
использование будет потеряна. Чтобы справиться с задачей устойчивости,
нужно предотвратить повторение этой работы.  Очередной раз мы
используем для этого ленивое вычисление. Вместо списка куч
\lstinline!Heap list!, мы представляем детей узла как задержанную кучу
\lstinline!Heap susp!. Значение этой задержки равно
\lstinline!$\$$mergePairs cs!. Поскольку \lstinline!mergePairs!
работает с парами элементов списка детей, мы будем расширять нашу
задержку двумя элементами сразу. Следовательно, нам понадобится
дополнительное поле типа \lstinline!Heap! в каждом узле для хранения
непарных потомков. Если непарных потомков нет (т.~е., число детей
чётно), это дополнительное поле будет пустым. Поскольку это поле
используется только тогда, когда число детей нечетно, мы будем
называть его \term{нечётным полем}{odd field}. Таким образом, наш
новый тип данных имеет вид
\begin{lstlisting}
  datatype Heap = E | T of Elem.T $\times$ Heap $\times$ Heap susp
\end{lstlisting}
Операции \lstinline!insert! и \lstinline!findMin! почти не требуют
изменений.
\begin{lstlisting}
  fun insert (x, a) = merge (T (x, E, $\$$E), a)
  fun findMin (T (x, a, m)) = x
\end{lstlisting}
Раньше у нас операция \lstinline!merge! была простой, а операция
\lstinline!deleteMin!~--- сложной. Теперь ситуация обратная~--- вся
сложность функции \lstinline!mergePairs! оказалась перенесена в
\lstinline!merge!, которая устанавливает все необходимые
задержки. Функция \lstinline!deleteMin! просто вынуждает задержку кучи
и сливает ее с нечётным полем.
\begin{lstlisting}
  fun deleteMin (T (x, a, $\$$b)) = merge (a, b)
\end{lstlisting}
Функцию \lstinline!merge! мы определяем в два шага. Первый шаг
проверяет, что аргументы непусты, и если это так, выясняет, у которого
из двух аргументов меньше корневой элемент.
\begin{lstlisting}
  fun merge (a, E) = a
    | merge (E, a) = a
    | merge (a as T (x, _, _), b as T (y, _, _)) =
        if Elem.leq (x, y) then link (a, b) else link (b, a)
\end{lstlisting}
Второй шаг, воплощенный во вспомогательной функции \lstinline!link!,
добавляет к куче новый элемент. Если нечётное поле пусто, новый
ребёнок добавляется туда.
\begin{lstlisting}
  fun link (T (x, E, m), a) = T (x, a, m)
\end{lstlisting}
В противном случае новый ребёнок спаривается с ребёнком из нечётного
поля, и оба они добавляются к задержке. Другими словами, мы превращаем
задержку \lstinline!m = $\$$mergePairs cs! в 
\lstinline!$\$$mergePairs (a :: b :: cs)!. Заметим, что
\begin{lstlisting}
  $\$$mergePairs (a :: b :: cs)
    $\equiv$ $\$$merge (merge (a, b), mergePairs cs)
    $\equiv$ $\$$merge (merge (a, b), force ($\$$mergePairs cs))
    $\equiv$ $\$$merge (merge (a, b), force m)
\end{lstlisting}
так что, вторую ветвь функции \lstinline!link! можно записать как
\begin{lstlisting}
  fun link (T (x, b, m), a) = T (x, E, $\$$merge (merge (a, b), force m))
\end{lstlisting}
Полный код этой реализации приведен на Рис.~\ref{fig:6.6}

\begin{figure}
  \centering
  
  \caption{Устойчивые парные кучи с использованием ленивого вычисления}
  \label{fig:6.6}
\end{figure}

\begin{hint}
  Несмотря на то, что эта реализация парных куч хорошо работает в условиях
  устойчивости, на практике она оказывается довольно
  медленной из-за высокой стоимости ленивого вычисления. Однако в
  условиях активного использования устойчивости эта реализация
  ведёт себя прекрасно~--- мы получаем максимальную пользу от
  мемоизации. Кроме того, эта реализация конкурентоспособна в ленивых
  языках, где дополнительную стоимость ленивого вычисления платят все
  структуры данных, независимо от того, есть им от этого выгода или нет.
\end{hint}

\section{Примечания}
\label{sc:6.6}

\noindent
\textbf{Долг} Некоторые разновидности анализа с использованием
традиционного метода банкира, например, анализ сжатия путей Тарджаном
\cite{Tarjan1983}, работают и с понятием кредита, и с понятием
дебета.  Когда операции требуется кредит больше, чем имеется в данный
момент, она создает пару кредит-дебет и немедленно тратит
кредит. Дебет остается как обязательство, подлежащее исполнению. Позже
избыток кредита можно использовать для выплаты долга\footnote{Здесь
  напрашивается явная аналогия со спонтанным порождением и
  аннигиляцией пар частица-античастица в физике. В сущности, для этого
  дебета более подходящим названием было бы <<антикредит>>.
}.
Дебет, остающийся в конце вычисления, добавляется к общей реальной
стоимости. Несмотря на некоторое сходство между двумя понятиями долга,
есть и явные различия. Например, долг, как он введен в этой главе,
оставшийся в конце вычисления, тихо уничтожается.

Интересно заметить, что дебет введен Тарджаном при анализе сжатия
путей; ведь сжатие путей, в сущности, является применением мемоизации
к функции поиска.

\noindent
\textbf{Амортизация и устойчивость} До публикации этой работы
считалось, что амортизация несовместима с устойчивостью.  Несколько
исследователей \cite{DriscollSleatorTarjan1994, Raman1992} замечали,
что амортизированные структуры невозможно сделать эффективно
устойчивыми с помощью существующих методик добавления устойчивости к
эфемерным структурам данных, подобных описанным в
\cite{Driscolletal1989, Dietz1989}, по причинам вроде указанных нами в
Разделе~\ref{sc:5.6}. Интересно, что эти методики порождают структуры
с амортизированными показателями производительности, при том, что
показатели нижележащей структуры должны быть жесткими. (У этих методик
есть и другие ограничения. Прежде всего, они не работают для структур
данных, имеющих функции, применимые более, чем к одной
версии. Примерами таких запретных операций являются конкатенация
списков и объединение множеств.)

Идея, что ленивое вычисление может помирить амортизацию и устойчивость,
впервые, в рудиментарной форме, появилась в
\cite{Okasaki1995c}. Теория и практика этого подхода были развиты в
\cite{Okasaki1995a, Okasaki1996b}.

\noindent
\textbf{Амортизация и функциональные структуры данных} Схунмакерс
\cite{Schoenmakers1993} в своей диссертации исследует амортизированные
структуры данных в аппликативном функциональном языке, в основном
исследуя формальный вывод амортизированных ограничений с помощью
метода физика. Он избегает проблем устойчивости, настаивая, чтобы все
структуры данных использовались только в однопоточном режиме.

\noindent
\textbf{Очереди и биномиальные кучи} Очереди из
Раздела~\ref{sc:6.3.2} и ленивые биномиальные кучи из
Раздела~\ref{sc:6.4.1} впервые появились в \cite{Okasaki1996b}. Анализ
ленивых биномиальных куч применим также к реализации Кинга
\cite{King1994}.

\noindent
\textbf{Анализ времени выполнения ленивых программ} Существует несколько
теоретических формализмов для анализа временного поведения ленивых
программ \cite{BjernerHolmstrom1989, Sands1990, Sands1995,
  Wadler1988}. Однако эти формализмы недостаточно пока разработаны,
чтобы быть применимыми на практике. Одна из сложностей состоит в том,
что они, в некотором смысле, чрезмерно общие. В каждой из этих систем
стоимость программы вычисляется по отношению к некоторому контексту,
который представляет собой описание, как результат программы будет
использоваться. Однако этот подход часто неприменим в методологии
разработки программ, где структуры данных проектируются как
абстрактные типы, чье поведение, включая сложность операций,
описывается в изоляции. В противоположность этим подходам наш способ
анализа дает независимые от контекста результаты (т.~е., они верны
безотносительно того, как структуры данных будут использоваться).

%%% Local Variables: 
%%% mode: latex
%%% TeX-master: "pfds"
%%% End: 


\end{document}
