\chapter{Некоторые известные структуры данных в функциональном
  окружении}
\label{ch:3}

Хотя реализовать в функциональной среде многие императивные структуры
данных трудно или невозможно, есть и такие, которые реализуются без
особых усилий.  В этой главе мы рассматриваем три структуры данных,
которым обычно учат в императивной среде. Первая из них,
левоориентированные кучи, просто устроена и в том, и в другом
окружении. Однако две других, биномиальные очереди и красно-черные
деревья, часто считаются сложными для понимания, поскольку
их императивные реализации быстро превращаются в мешанину манипуляций
с указателями.  Напротив, функциональные реализации этих структур
данных абстрагируются от действий с указателями и прямо отражают
высокоуровневые представления. Дополнительное преимущество
функциональной реализации этих структур состоит в том, что мы
бесплатно получаем устойчивость.

\section{Левоориентированные кучи}
\label{sc:3.1}

%%% Local Variables:
%%% mode: latex
%%% TeX-master: "pfds"
%%% End:
